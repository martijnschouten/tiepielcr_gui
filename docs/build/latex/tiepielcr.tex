%% Generated by Sphinx.
\def\sphinxdocclass{report}
\documentclass[letterpaper,10pt,english]{sphinxmanual}
\ifdefined\pdfpxdimen
   \let\sphinxpxdimen\pdfpxdimen\else\newdimen\sphinxpxdimen
\fi \sphinxpxdimen=.75bp\relax
\ifdefined\pdfimageresolution
    \pdfimageresolution= \numexpr \dimexpr1in\relax/\sphinxpxdimen\relax
\fi
%% let collapsible pdf bookmarks panel have high depth per default
\PassOptionsToPackage{bookmarksdepth=5}{hyperref}

\PassOptionsToPackage{warn}{textcomp}
\usepackage[utf8]{inputenc}
\ifdefined\DeclareUnicodeCharacter
% support both utf8 and utf8x syntaxes
  \ifdefined\DeclareUnicodeCharacterAsOptional
    \def\sphinxDUC#1{\DeclareUnicodeCharacter{"#1}}
  \else
    \let\sphinxDUC\DeclareUnicodeCharacter
  \fi
  \sphinxDUC{00A0}{\nobreakspace}
  \sphinxDUC{2500}{\sphinxunichar{2500}}
  \sphinxDUC{2502}{\sphinxunichar{2502}}
  \sphinxDUC{2514}{\sphinxunichar{2514}}
  \sphinxDUC{251C}{\sphinxunichar{251C}}
  \sphinxDUC{2572}{\textbackslash}
\fi
\usepackage{cmap}
\usepackage[T1]{fontenc}
\usepackage{amsmath,amssymb,amstext}
\usepackage{babel}



\usepackage{tgtermes}
\usepackage{tgheros}
\renewcommand{\ttdefault}{txtt}



\usepackage[Bjarne]{fncychap}
\usepackage{sphinx}

\fvset{fontsize=auto}
\usepackage{geometry}


% Include hyperref last.
\usepackage{hyperref}
% Fix anchor placement for figures with captions.
\usepackage{hypcap}% it must be loaded after hyperref.
% Set up styles of URL: it should be placed after hyperref.
\urlstyle{same}

\addto\captionsenglish{\renewcommand{\contentsname}{Contents:}}

\usepackage{sphinxmessages}
\setcounter{tocdepth}{1}



\title{TiePieLCR}
\date{Jan 24, 2022}
\release{V1.0}
\author{Martijn Schouten}
\newcommand{\sphinxlogo}{\vbox{}}
\renewcommand{\releasename}{Release}
\makeindex
\begin{document}

\pagestyle{empty}
\sphinxmaketitle
\pagestyle{plain}
\sphinxtableofcontents
\pagestyle{normal}
\phantomsection\label{\detokenize{index::doc}}


\sphinxAtStartPar
This documentation documenents the code of the TiePieLCR multi frequency impedance analyser. The hardware for this device can be found \sphinxhref{github.com/martijnschouten/tiepielcr\_hardware}{here}


\chapter{App mainwindow class}
\label{\detokenize{index:module-app}}\label{\detokenize{index:app-mainwindow-class}}\index{module@\spxentry{module}!app@\spxentry{app}}\index{app@\spxentry{app}!module@\spxentry{module}}\phantomsection\label{\detokenize{index:module-MainWindow}}\index{module@\spxentry{module}!MainWindow@\spxentry{MainWindow}}\index{MainWindow@\spxentry{MainWindow}!module@\spxentry{module}}\index{MainWindow (class in app)@\spxentry{MainWindow}\spxextra{class in app}}

\begin{fulllineitems}
\phantomsection\label{\detokenize{index:app.MainWindow}}\pysiglinewithargsret{\sphinxbfcode{\sphinxupquote{class\DUrole{w}{  }}}\sphinxcode{\sphinxupquote{app.}}\sphinxbfcode{\sphinxupquote{MainWindow}}}{\emph{\DUrole{o}{*}\DUrole{n}{args}}, \emph{\DUrole{o}{**}\DUrole{n}{kwargs}}}{}
\sphinxAtStartPar
Bases: \sphinxcode{\sphinxupquote{PyQt5.QtWidgets.QMainWindow}}
\index{acquisition\_function() (app.MainWindow static method)@\spxentry{acquisition\_function()}\spxextra{app.MainWindow static method}}

\begin{fulllineitems}
\phantomsection\label{\detokenize{index:app.MainWindow.acquisition_function}}\pysiglinewithargsret{\sphinxbfcode{\sphinxupquote{static\DUrole{w}{  }}}\sphinxbfcode{\sphinxupquote{acquisition\_function}}}{\emph{\DUrole{n}{LCR\_settings\_queue}}, \emph{\DUrole{n}{displayed\_error}}, \emph{\DUrole{n}{shared\_state}}, \emph{\DUrole{n}{plot\_data\_queue}}, \emph{\DUrole{n}{stored\_data\_queue}}, \emph{\DUrole{n}{stored\_data\_requested}}}{}
\end{fulllineitems}

\index{closeEvent() (app.MainWindow method)@\spxentry{closeEvent()}\spxextra{app.MainWindow method}}

\begin{fulllineitems}
\phantomsection\label{\detokenize{index:app.MainWindow.closeEvent}}\pysiglinewithargsret{\sphinxbfcode{\sphinxupquote{closeEvent}}}{\emph{\DUrole{n}{self}}, \emph{\DUrole{n}{QCloseEvent}}}{}
\end{fulllineitems}

\index{displayed\_error (app.MainWindow attribute)@\spxentry{displayed\_error}\spxextra{app.MainWindow attribute}}

\begin{fulllineitems}
\phantomsection\label{\detokenize{index:app.MainWindow.displayed_error}}\pysigline{\sphinxbfcode{\sphinxupquote{displayed\_error}}\sphinxbfcode{\sphinxupquote{\DUrole{w}{  }\DUrole{p}{=}\DUrole{w}{  }\textless{}SynchronizedString wrapper for \textless{}multiprocessing.sharedctypes.c\_char\_Array\_300 object\textgreater{}\textgreater{}}}}
\end{fulllineitems}

\index{fs\_changed() (app.MainWindow method)@\spxentry{fs\_changed()}\spxextra{app.MainWindow method}}

\begin{fulllineitems}
\phantomsection\label{\detokenize{index:app.MainWindow.fs_changed}}\pysiglinewithargsret{\sphinxbfcode{\sphinxupquote{fs\_changed}}}{}{}
\end{fulllineitems}

\index{get\_all\_stored\_data() (app.MainWindow method)@\spxentry{get\_all\_stored\_data()}\spxextra{app.MainWindow method}}

\begin{fulllineitems}
\phantomsection\label{\detokenize{index:app.MainWindow.get_all_stored_data}}\pysiglinewithargsret{\sphinxbfcode{\sphinxupquote{get\_all\_stored\_data}}}{}{}
\end{fulllineitems}

\index{load\_settings() (app.MainWindow method)@\spxentry{load\_settings()}\spxextra{app.MainWindow method}}

\begin{fulllineitems}
\phantomsection\label{\detokenize{index:app.MainWindow.load_settings}}\pysiglinewithargsret{\sphinxbfcode{\sphinxupquote{load\_settings}}}{}{}
\end{fulllineitems}

\index{lockins (app.MainWindow attribute)@\spxentry{lockins}\spxextra{app.MainWindow attribute}}

\begin{fulllineitems}
\phantomsection\label{\detokenize{index:app.MainWindow.lockins}}\pysigline{\sphinxbfcode{\sphinxupquote{lockins}}\sphinxbfcode{\sphinxupquote{\DUrole{w}{  }\DUrole{p}{=}\DUrole{w}{  }1}}}
\end{fulllineitems}

\index{lockins\_changed() (app.MainWindow method)@\spxentry{lockins\_changed()}\spxextra{app.MainWindow method}}

\begin{fulllineitems}
\phantomsection\label{\detokenize{index:app.MainWindow.lockins_changed}}\pysiglinewithargsret{\sphinxbfcode{\sphinxupquote{lockins\_changed}}}{}{}
\end{fulllineitems}

\index{multiprocessing\_init() (app.MainWindow method)@\spxentry{multiprocessing\_init()}\spxextra{app.MainWindow method}}

\begin{fulllineitems}
\phantomsection\label{\detokenize{index:app.MainWindow.multiprocessing_init}}\pysiglinewithargsret{\sphinxbfcode{\sphinxupquote{multiprocessing\_init}}}{}{}
\end{fulllineitems}

\index{new\_data (app.MainWindow attribute)@\spxentry{new\_data}\spxextra{app.MainWindow attribute}}

\begin{fulllineitems}
\phantomsection\label{\detokenize{index:app.MainWindow.new_data}}\pysigline{\sphinxbfcode{\sphinxupquote{new\_data}}\sphinxbfcode{\sphinxupquote{\DUrole{w}{  }\DUrole{p}{=}\DUrole{w}{  }False}}}
\end{fulllineitems}

\index{old\_error (app.MainWindow attribute)@\spxentry{old\_error}\spxextra{app.MainWindow attribute}}

\begin{fulllineitems}
\phantomsection\label{\detokenize{index:app.MainWindow.old_error}}\pysigline{\sphinxbfcode{\sphinxupquote{old\_error}}\sphinxbfcode{\sphinxupquote{\DUrole{w}{  }\DUrole{p}{=}\DUrole{w}{  }\textquotesingle{}\textquotesingle{}}}}
\end{fulllineitems}

\index{plot\_data\_queue (app.MainWindow attribute)@\spxentry{plot\_data\_queue}\spxextra{app.MainWindow attribute}}

\begin{fulllineitems}
\phantomsection\label{\detokenize{index:app.MainWindow.plot_data_queue}}\pysigline{\sphinxbfcode{\sphinxupquote{plot\_data\_queue}}\sphinxbfcode{\sphinxupquote{\DUrole{w}{  }\DUrole{p}{=}\DUrole{w}{  }None}}}
\end{fulllineitems}

\index{reference\_range (app.MainWindow attribute)@\spxentry{reference\_range}\spxextra{app.MainWindow attribute}}

\begin{fulllineitems}
\phantomsection\label{\detokenize{index:app.MainWindow.reference_range}}\pysigline{\sphinxbfcode{\sphinxupquote{reference\_range}}\sphinxbfcode{\sphinxupquote{\DUrole{w}{  }\DUrole{p}{=}\DUrole{w}{  }0}}}
\end{fulllineitems}

\index{save\_data() (app.MainWindow method)@\spxentry{save\_data()}\spxextra{app.MainWindow method}}

\begin{fulllineitems}
\phantomsection\label{\detokenize{index:app.MainWindow.save_data}}\pysiglinewithargsret{\sphinxbfcode{\sphinxupquote{save\_data}}}{\emph{\DUrole{n}{save\_location}}}{}
\end{fulllineitems}

\index{save\_data\_clicked() (app.MainWindow method)@\spxentry{save\_data\_clicked()}\spxextra{app.MainWindow method}}

\begin{fulllineitems}
\phantomsection\label{\detokenize{index:app.MainWindow.save_data_clicked}}\pysiglinewithargsret{\sphinxbfcode{\sphinxupquote{save\_data\_clicked}}}{}{}
\end{fulllineitems}

\index{save\_file\_dialog() (app.MainWindow method)@\spxentry{save\_file\_dialog()}\spxextra{app.MainWindow method}}

\begin{fulllineitems}
\phantomsection\label{\detokenize{index:app.MainWindow.save_file_dialog}}\pysiglinewithargsret{\sphinxbfcode{\sphinxupquote{save\_file\_dialog}}}{}{}
\end{fulllineitems}

\index{save\_settings() (app.MainWindow method)@\spxentry{save\_settings()}\spxextra{app.MainWindow method}}

\begin{fulllineitems}
\phantomsection\label{\detokenize{index:app.MainWindow.save_settings}}\pysiglinewithargsret{\sphinxbfcode{\sphinxupquote{save\_settings}}}{}{}
\end{fulllineitems}

\index{settings\_filename (app.MainWindow attribute)@\spxentry{settings\_filename}\spxextra{app.MainWindow attribute}}

\begin{fulllineitems}
\phantomsection\label{\detokenize{index:app.MainWindow.settings_filename}}\pysigline{\sphinxbfcode{\sphinxupquote{settings\_filename}}\sphinxbfcode{\sphinxupquote{\DUrole{w}{  }\DUrole{p}{=}\DUrole{w}{  }\textquotesingle{}settings\textquotesingle{}}}}
\end{fulllineitems}

\index{settings\_folder (app.MainWindow attribute)@\spxentry{settings\_folder}\spxextra{app.MainWindow attribute}}

\begin{fulllineitems}
\phantomsection\label{\detokenize{index:app.MainWindow.settings_folder}}\pysigline{\sphinxbfcode{\sphinxupquote{settings\_folder}}\sphinxbfcode{\sphinxupquote{\DUrole{w}{  }\DUrole{p}{=}\DUrole{w}{  }\textquotesingle{}settings/\textquotesingle{}}}}
\end{fulllineitems}

\index{settings\_queue (app.MainWindow attribute)@\spxentry{settings\_queue}\spxextra{app.MainWindow attribute}}

\begin{fulllineitems}
\phantomsection\label{\detokenize{index:app.MainWindow.settings_queue}}\pysigline{\sphinxbfcode{\sphinxupquote{settings\_queue}}\sphinxbfcode{\sphinxupquote{\DUrole{w}{  }\DUrole{p}{=}\DUrole{w}{  }\textless{}multiprocessing.queues.Queue object\textgreater{}}}}
\end{fulllineitems}

\index{start\_button\_clicked() (app.MainWindow method)@\spxentry{start\_button\_clicked()}\spxextra{app.MainWindow method}}

\begin{fulllineitems}
\phantomsection\label{\detokenize{index:app.MainWindow.start_button_clicked}}\pysiglinewithargsret{\sphinxbfcode{\sphinxupquote{start\_button\_clicked}}}{}{}
\end{fulllineitems}

\index{start\_measurement() (app.MainWindow method)@\spxentry{start\_measurement()}\spxextra{app.MainWindow method}}

\begin{fulllineitems}
\phantomsection\label{\detokenize{index:app.MainWindow.start_measurement}}\pysiglinewithargsret{\sphinxbfcode{\sphinxupquote{start\_measurement}}}{}{}
\end{fulllineitems}

\index{state (app.MainWindow attribute)@\spxentry{state}\spxextra{app.MainWindow attribute}}

\begin{fulllineitems}
\phantomsection\label{\detokenize{index:app.MainWindow.state}}\pysigline{\sphinxbfcode{\sphinxupquote{state}}\sphinxbfcode{\sphinxupquote{\DUrole{w}{  }\DUrole{p}{=}\DUrole{w}{  }\sphinxhyphen{}1}}}
\end{fulllineitems}

\index{state\_timer() (app.MainWindow method)@\spxentry{state\_timer()}\spxextra{app.MainWindow method}}

\begin{fulllineitems}
\phantomsection\label{\detokenize{index:app.MainWindow.state_timer}}\pysiglinewithargsret{\sphinxbfcode{\sphinxupquote{state\_timer}}}{}{}
\end{fulllineitems}

\index{stop\_measurement() (app.MainWindow method)@\spxentry{stop\_measurement()}\spxextra{app.MainWindow method}}

\begin{fulllineitems}
\phantomsection\label{\detokenize{index:app.MainWindow.stop_measurement}}\pysiglinewithargsret{\sphinxbfcode{\sphinxupquote{stop\_measurement}}}{}{}
\end{fulllineitems}

\index{sub\_blocks\_changed() (app.MainWindow method)@\spxentry{sub\_blocks\_changed()}\spxextra{app.MainWindow method}}

\begin{fulllineitems}
\phantomsection\label{\detokenize{index:app.MainWindow.sub_blocks_changed}}\pysiglinewithargsret{\sphinxbfcode{\sphinxupquote{sub\_blocks\_changed}}}{}{}
\end{fulllineitems}

\index{sync\_settings() (app.MainWindow method)@\spxentry{sync\_settings()}\spxextra{app.MainWindow method}}

\begin{fulllineitems}
\phantomsection\label{\detokenize{index:app.MainWindow.sync_settings}}\pysiglinewithargsret{\sphinxbfcode{\sphinxupquote{sync\_settings}}}{}{}
\end{fulllineitems}

\index{tcp\_daemon() (app.MainWindow method)@\spxentry{tcp\_daemon()}\spxextra{app.MainWindow method}}

\begin{fulllineitems}
\phantomsection\label{\detokenize{index:app.MainWindow.tcp_daemon}}\pysiglinewithargsret{\sphinxbfcode{\sphinxupquote{tcp\_daemon}}}{}{}
\end{fulllineitems}

\index{update\_freq\_changed() (app.MainWindow method)@\spxentry{update\_freq\_changed()}\spxextra{app.MainWindow method}}

\begin{fulllineitems}
\phantomsection\label{\detokenize{index:app.MainWindow.update_freq_changed}}\pysiglinewithargsret{\sphinxbfcode{\sphinxupquote{update\_freq\_changed}}}{}{}
\end{fulllineitems}

\index{updating\_common\_gui (app.MainWindow attribute)@\spxentry{updating\_common\_gui}\spxextra{app.MainWindow attribute}}

\begin{fulllineitems}
\phantomsection\label{\detokenize{index:app.MainWindow.updating_common_gui}}\pysigline{\sphinxbfcode{\sphinxupquote{updating\_common\_gui}}\sphinxbfcode{\sphinxupquote{\DUrole{w}{  }\DUrole{p}{=}\DUrole{w}{  }False}}}
\end{fulllineitems}

\index{updating\_gui (app.MainWindow attribute)@\spxentry{updating\_gui}\spxextra{app.MainWindow attribute}}

\begin{fulllineitems}
\phantomsection\label{\detokenize{index:app.MainWindow.updating_gui}}\pysigline{\sphinxbfcode{\sphinxupquote{updating\_gui}}\sphinxbfcode{\sphinxupquote{\DUrole{w}{  }\DUrole{p}{=}\DUrole{w}{  }False}}}
\end{fulllineitems}


\end{fulllineitems}

\index{main() (in module app)@\spxentry{main()}\spxextra{in module app}}

\begin{fulllineitems}
\phantomsection\label{\detokenize{index:app.main}}\pysiglinewithargsret{\sphinxcode{\sphinxupquote{app.}}\sphinxbfcode{\sphinxupquote{main}}}{}{}
\end{fulllineitems}



\chapter{Lockin tab class}
\label{\detokenize{index:module-lockin_tab}}\label{\detokenize{index:lockin-tab-class}}\index{module@\spxentry{module}!lockin\_tab@\spxentry{lockin\_tab}}\index{lockin\_tab@\spxentry{lockin\_tab}!module@\spxentry{module}}\phantomsection\label{\detokenize{index:module-0}}\index{module@\spxentry{module}!lockin\_tab@\spxentry{lockin\_tab}}\index{lockin\_tab@\spxentry{lockin\_tab}!module@\spxentry{module}}\index{lockin\_tab (class in lockin\_tab)@\spxentry{lockin\_tab}\spxextra{class in lockin\_tab}}

\begin{fulllineitems}
\phantomsection\label{\detokenize{index:lockin_tab.lockin_tab}}\pysiglinewithargsret{\sphinxbfcode{\sphinxupquote{class\DUrole{w}{  }}}\sphinxcode{\sphinxupquote{lockin\_tab.}}\sphinxbfcode{\sphinxupquote{lockin\_tab}}}{\emph{\DUrole{n}{mainwindow}}, \emph{\DUrole{n}{instance}}, \emph{\DUrole{o}{*}\DUrole{n}{args}}, \emph{\DUrole{o}{**}\DUrole{n}{kwargs}}}{}
\sphinxAtStartPar
Bases: \sphinxcode{\sphinxupquote{PyQt5.QtWidgets.QWidget}}
\index{amplitude\_changed() (lockin\_tab.lockin\_tab method)@\spxentry{amplitude\_changed()}\spxextra{lockin\_tab.lockin\_tab method}}

\begin{fulllineitems}
\phantomsection\label{\detokenize{index:lockin_tab.lockin_tab.amplitude_changed}}\pysiglinewithargsret{\sphinxbfcode{\sphinxupquote{amplitude\_changed}}}{}{}
\end{fulllineitems}

\index{bandwidth\_changed() (lockin\_tab.lockin\_tab method)@\spxentry{bandwidth\_changed()}\spxextra{lockin\_tab.lockin\_tab method}}

\begin{fulllineitems}
\phantomsection\label{\detokenize{index:lockin_tab.lockin_tab.bandwidth_changed}}\pysiglinewithargsret{\sphinxbfcode{\sphinxupquote{bandwidth\_changed}}}{}{}
\end{fulllineitems}

\index{buffer\_size (lockin\_tab.lockin\_tab attribute)@\spxentry{buffer\_size}\spxextra{lockin\_tab.lockin\_tab attribute}}

\begin{fulllineitems}
\phantomsection\label{\detokenize{index:lockin_tab.lockin_tab.buffer_size}}\pysigline{\sphinxbfcode{\sphinxupquote{buffer\_size}}\sphinxbfcode{\sphinxupquote{\DUrole{w}{  }\DUrole{p}{=}\DUrole{w}{  }5}}}
\end{fulllineitems}

\index{build\_demodulate\_plots() (lockin\_tab.lockin\_tab method)@\spxentry{build\_demodulate\_plots()}\spxextra{lockin\_tab.lockin\_tab method}}

\begin{fulllineitems}
\phantomsection\label{\detokenize{index:lockin_tab.lockin_tab.build_demodulate_plots}}\pysiglinewithargsret{\sphinxbfcode{\sphinxupquote{build\_demodulate\_plots}}}{}{}
\end{fulllineitems}

\index{build\_ref\_fft\_plot() (lockin\_tab.lockin\_tab method)@\spxentry{build\_ref\_fft\_plot()}\spxextra{lockin\_tab.lockin\_tab method}}

\begin{fulllineitems}
\phantomsection\label{\detokenize{index:lockin_tab.lockin_tab.build_ref_fft_plot}}\pysiglinewithargsret{\sphinxbfcode{\sphinxupquote{build\_ref\_fft\_plot}}}{}{}
\end{fulllineitems}

\index{build\_sig\_fft\_plot() (lockin\_tab.lockin\_tab method)@\spxentry{build\_sig\_fft\_plot()}\spxextra{lockin\_tab.lockin\_tab method}}

\begin{fulllineitems}
\phantomsection\label{\detokenize{index:lockin_tab.lockin_tab.build_sig_fft_plot}}\pysiglinewithargsret{\sphinxbfcode{\sphinxupquote{build\_sig\_fft\_plot}}}{}{}
\end{fulllineitems}

\index{build\_signal\_plot() (lockin\_tab.lockin\_tab method)@\spxentry{build\_signal\_plot()}\spxextra{lockin\_tab.lockin\_tab method}}

\begin{fulllineitems}
\phantomsection\label{\detokenize{index:lockin_tab.lockin_tab.build_signal_plot}}\pysiglinewithargsret{\sphinxbfcode{\sphinxupquote{build\_signal\_plot}}}{}{}
\end{fulllineitems}

\index{color\_list (lockin\_tab.lockin\_tab attribute)@\spxentry{color\_list}\spxextra{lockin\_tab.lockin\_tab attribute}}

\begin{fulllineitems}
\phantomsection\label{\detokenize{index:lockin_tab.lockin_tab.color_list}}\pysigline{\sphinxbfcode{\sphinxupquote{color\_list}}\sphinxbfcode{\sphinxupquote{\DUrole{w}{  }\DUrole{p}{=}\DUrole{w}{  }((230, 25, 75), (60, 180, 75), (255, 225, 25), (0, 130, 200), (245, 130, 48), (145, 30, 180), (70, 240, 240), (240, 50, 230), (210, 245, 60), (250, 190, 212), (0, 128, 128), (220, 190, 255), (170, 110, 40), (255, 250, 200), (128, 0, 0), (170, 255, 195), (128, 128, 0), (255, 215, 180), (0, 0, 128), (128, 128, 128), (0, 0, 0))}}}
\end{fulllineitems}

\index{delete\_dem\_freq\_pressed() (lockin\_tab.lockin\_tab method)@\spxentry{delete\_dem\_freq\_pressed()}\spxextra{lockin\_tab.lockin\_tab method}}

\begin{fulllineitems}
\phantomsection\label{\detokenize{index:lockin_tab.lockin_tab.delete_dem_freq_pressed}}\pysiglinewithargsret{\sphinxbfcode{\sphinxupquote{delete\_dem\_freq\_pressed}}}{}{}
\end{fulllineitems}

\index{delete\_gen\_freq\_pressed() (lockin\_tab.lockin\_tab method)@\spxentry{delete\_gen\_freq\_pressed()}\spxextra{lockin\_tab.lockin\_tab method}}

\begin{fulllineitems}
\phantomsection\label{\detokenize{index:lockin_tab.lockin_tab.delete_gen_freq_pressed}}\pysiglinewithargsret{\sphinxbfcode{\sphinxupquote{delete\_gen\_freq\_pressed}}}{}{}
\end{fulllineitems}

\index{dem\_frequency\_table\_changed() (lockin\_tab.lockin\_tab method)@\spxentry{dem\_frequency\_table\_changed()}\spxextra{lockin\_tab.lockin\_tab method}}

\begin{fulllineitems}
\phantomsection\label{\detokenize{index:lockin_tab.lockin_tab.dem_frequency_table_changed}}\pysiglinewithargsret{\sphinxbfcode{\sphinxupquote{dem\_frequency\_table\_changed}}}{}{}
\end{fulllineitems}

\index{dem\_is\_gen\_changed() (lockin\_tab.lockin\_tab method)@\spxentry{dem\_is\_gen\_changed()}\spxextra{lockin\_tab.lockin\_tab method}}

\begin{fulllineitems}
\phantomsection\label{\detokenize{index:lockin_tab.lockin_tab.dem_is_gen_changed}}\pysiglinewithargsret{\sphinxbfcode{\sphinxupquote{dem\_is\_gen\_changed}}}{}{}
\end{fulllineitems}

\index{dem\_time\_vec (lockin\_tab.lockin\_tab attribute)@\spxentry{dem\_time\_vec}\spxextra{lockin\_tab.lockin\_tab attribute}}

\begin{fulllineitems}
\phantomsection\label{\detokenize{index:lockin_tab.lockin_tab.dem_time_vec}}\pysigline{\sphinxbfcode{\sphinxupquote{dem\_time\_vec}}\sphinxbfcode{\sphinxupquote{\DUrole{w}{  }\DUrole{p}{=}\DUrole{w}{  }{[}{]}}}}
\end{fulllineitems}

\index{dynamic\_update\_gui() (lockin\_tab.lockin\_tab method)@\spxentry{dynamic\_update\_gui()}\spxextra{lockin\_tab.lockin\_tab method}}

\begin{fulllineitems}
\phantomsection\label{\detokenize{index:lockin_tab.lockin_tab.dynamic_update_gui}}\pysiglinewithargsret{\sphinxbfcode{\sphinxupquote{dynamic\_update\_gui}}}{}{}
\end{fulllineitems}

\index{first\_time\_reference\_changed (lockin\_tab.lockin\_tab attribute)@\spxentry{first\_time\_reference\_changed}\spxextra{lockin\_tab.lockin\_tab attribute}}

\begin{fulllineitems}
\phantomsection\label{\detokenize{index:lockin_tab.lockin_tab.first_time_reference_changed}}\pysigline{\sphinxbfcode{\sphinxupquote{first\_time\_reference\_changed}}\sphinxbfcode{\sphinxupquote{\DUrole{w}{  }\DUrole{p}{=}\DUrole{w}{  }True}}}
\end{fulllineitems}

\index{fmax\_changed() (lockin\_tab.lockin\_tab method)@\spxentry{fmax\_changed()}\spxextra{lockin\_tab.lockin\_tab method}}

\begin{fulllineitems}
\phantomsection\label{\detokenize{index:lockin_tab.lockin_tab.fmax_changed}}\pysiglinewithargsret{\sphinxbfcode{\sphinxupquote{fmax\_changed}}}{}{}
\end{fulllineitems}

\index{fmin\_changed() (lockin\_tab.lockin\_tab method)@\spxentry{fmin\_changed()}\spxextra{lockin\_tab.lockin\_tab method}}

\begin{fulllineitems}
\phantomsection\label{\detokenize{index:lockin_tab.lockin_tab.fmin_changed}}\pysiglinewithargsret{\sphinxbfcode{\sphinxupquote{fmin\_changed}}}{}{}
\end{fulllineitems}

\index{format\_changed() (lockin\_tab.lockin\_tab method)@\spxentry{format\_changed()}\spxextra{lockin\_tab.lockin\_tab method}}

\begin{fulllineitems}
\phantomsection\label{\detokenize{index:lockin_tab.lockin_tab.format_changed}}\pysiglinewithargsret{\sphinxbfcode{\sphinxupquote{format\_changed}}}{}{}
\end{fulllineitems}

\index{format\_demodulate\_labels() (lockin\_tab.lockin\_tab method)@\spxentry{format\_demodulate\_labels()}\spxextra{lockin\_tab.lockin\_tab method}}

\begin{fulllineitems}
\phantomsection\label{\detokenize{index:lockin_tab.lockin_tab.format_demodulate_labels}}\pysiglinewithargsret{\sphinxbfcode{\sphinxupquote{format\_demodulate\_labels}}}{}{}
\end{fulllineitems}

\index{format\_ref\_fft\_label() (lockin\_tab.lockin\_tab method)@\spxentry{format\_ref\_fft\_label()}\spxextra{lockin\_tab.lockin\_tab method}}

\begin{fulllineitems}
\phantomsection\label{\detokenize{index:lockin_tab.lockin_tab.format_ref_fft_label}}\pysiglinewithargsret{\sphinxbfcode{\sphinxupquote{format\_ref\_fft\_label}}}{}{}
\end{fulllineitems}

\index{format\_ref\_label() (lockin\_tab.lockin\_tab method)@\spxentry{format\_ref\_label()}\spxextra{lockin\_tab.lockin\_tab method}}

\begin{fulllineitems}
\phantomsection\label{\detokenize{index:lockin_tab.lockin_tab.format_ref_label}}\pysiglinewithargsret{\sphinxbfcode{\sphinxupquote{format\_ref\_label}}}{}{}
\end{fulllineitems}

\index{format\_sig\_fft\_label() (lockin\_tab.lockin\_tab method)@\spxentry{format\_sig\_fft\_label()}\spxextra{lockin\_tab.lockin\_tab method}}

\begin{fulllineitems}
\phantomsection\label{\detokenize{index:lockin_tab.lockin_tab.format_sig_fft_label}}\pysiglinewithargsret{\sphinxbfcode{\sphinxupquote{format\_sig\_fft\_label}}}{}{}
\end{fulllineitems}

\index{format\_sig\_label() (lockin\_tab.lockin\_tab method)@\spxentry{format\_sig\_label()}\spxextra{lockin\_tab.lockin\_tab method}}

\begin{fulllineitems}
\phantomsection\label{\detokenize{index:lockin_tab.lockin_tab.format_sig_label}}\pysiglinewithargsret{\sphinxbfcode{\sphinxupquote{format\_sig\_label}}}{}{}
\end{fulllineitems}

\index{gain\_changed() (lockin\_tab.lockin\_tab method)@\spxentry{gain\_changed()}\spxextra{lockin\_tab.lockin\_tab method}}

\begin{fulllineitems}
\phantomsection\label{\detokenize{index:lockin_tab.lockin_tab.gain_changed}}\pysiglinewithargsret{\sphinxbfcode{\sphinxupquote{gain\_changed}}}{}{}
\end{fulllineitems}

\index{gen\_frequency\_table\_changed() (lockin\_tab.lockin\_tab method)@\spxentry{gen\_frequency\_table\_changed()}\spxextra{lockin\_tab.lockin\_tab method}}

\begin{fulllineitems}
\phantomsection\label{\detokenize{index:lockin_tab.lockin_tab.gen_frequency_table_changed}}\pysiglinewithargsret{\sphinxbfcode{\sphinxupquote{gen\_frequency\_table\_changed}}}{}{}
\end{fulllineitems}

\index{gui\_update\_required (lockin\_tab.lockin\_tab attribute)@\spxentry{gui\_update\_required}\spxextra{lockin\_tab.lockin\_tab attribute}}

\begin{fulllineitems}
\phantomsection\label{\detokenize{index:lockin_tab.lockin_tab.gui_update_required}}\pysigline{\sphinxbfcode{\sphinxupquote{gui\_update\_required}}\sphinxbfcode{\sphinxupquote{\DUrole{w}{  }\DUrole{p}{=}\DUrole{w}{  }False}}}
\end{fulllineitems}

\index{host (lockin\_tab.lockin\_tab attribute)@\spxentry{host}\spxextra{lockin\_tab.lockin\_tab attribute}}

\begin{fulllineitems}
\phantomsection\label{\detokenize{index:lockin_tab.lockin_tab.host}}\pysigline{\sphinxbfcode{\sphinxupquote{host}}\sphinxbfcode{\sphinxupquote{\DUrole{w}{  }\DUrole{p}{=}\DUrole{w}{  }\textquotesingle{}127.0.0.1\textquotesingle{}}}}
\end{fulllineitems}

\index{insert\_dem\_freq\_pressed() (lockin\_tab.lockin\_tab method)@\spxentry{insert\_dem\_freq\_pressed()}\spxextra{lockin\_tab.lockin\_tab method}}

\begin{fulllineitems}
\phantomsection\label{\detokenize{index:lockin_tab.lockin_tab.insert_dem_freq_pressed}}\pysiglinewithargsret{\sphinxbfcode{\sphinxupquote{insert\_dem\_freq\_pressed}}}{}{}
\end{fulllineitems}

\index{insert\_gen\_freq\_pressed() (lockin\_tab.lockin\_tab method)@\spxentry{insert\_gen\_freq\_pressed()}\spxextra{lockin\_tab.lockin\_tab method}}

\begin{fulllineitems}
\phantomsection\label{\detokenize{index:lockin_tab.lockin_tab.insert_gen_freq_pressed}}\pysiglinewithargsret{\sphinxbfcode{\sphinxupquote{insert\_gen\_freq\_pressed}}}{}{}
\end{fulllineitems}

\index{int\_demodulate\_1 (lockin\_tab.lockin\_tab attribute)@\spxentry{int\_demodulate\_1}\spxextra{lockin\_tab.lockin\_tab attribute}}

\begin{fulllineitems}
\phantomsection\label{\detokenize{index:lockin_tab.lockin_tab.int_demodulate_1}}\pysigline{\sphinxbfcode{\sphinxupquote{int\_demodulate\_1}}\sphinxbfcode{\sphinxupquote{\DUrole{w}{  }\DUrole{p}{=}\DUrole{w}{  }0}}}
\end{fulllineitems}

\index{int\_demodulate\_2 (lockin\_tab.lockin\_tab attribute)@\spxentry{int\_demodulate\_2}\spxextra{lockin\_tab.lockin\_tab attribute}}

\begin{fulllineitems}
\phantomsection\label{\detokenize{index:lockin_tab.lockin_tab.int_demodulate_2}}\pysigline{\sphinxbfcode{\sphinxupquote{int\_demodulate\_2}}\sphinxbfcode{\sphinxupquote{\DUrole{w}{  }\DUrole{p}{=}\DUrole{w}{  }0}}}
\end{fulllineitems}

\index{integration\_changed() (lockin\_tab.lockin\_tab method)@\spxentry{integration\_changed()}\spxextra{lockin\_tab.lockin\_tab method}}

\begin{fulllineitems}
\phantomsection\label{\detokenize{index:lockin_tab.lockin_tab.integration_changed}}\pysiglinewithargsret{\sphinxbfcode{\sphinxupquote{integration\_changed}}}{}{}
\end{fulllineitems}

\index{offset\_bandwidth\_changed() (lockin\_tab.lockin\_tab method)@\spxentry{offset\_bandwidth\_changed()}\spxextra{lockin\_tab.lockin\_tab method}}

\begin{fulllineitems}
\phantomsection\label{\detokenize{index:lockin_tab.lockin_tab.offset_bandwidth_changed}}\pysiglinewithargsret{\sphinxbfcode{\sphinxupquote{offset\_bandwidth\_changed}}}{}{}
\end{fulllineitems}

\index{offset\_changed() (lockin\_tab.lockin\_tab method)@\spxentry{offset\_changed()}\spxextra{lockin\_tab.lockin\_tab method}}

\begin{fulllineitems}
\phantomsection\label{\detokenize{index:lockin_tab.lockin_tab.offset_changed}}\pysiglinewithargsret{\sphinxbfcode{\sphinxupquote{offset\_changed}}}{}{}
\end{fulllineitems}

\index{offset\_integration\_changed() (lockin\_tab.lockin\_tab method)@\spxentry{offset\_integration\_changed()}\spxextra{lockin\_tab.lockin\_tab method}}

\begin{fulllineitems}
\phantomsection\label{\detokenize{index:lockin_tab.lockin_tab.offset_integration_changed}}\pysiglinewithargsret{\sphinxbfcode{\sphinxupquote{offset\_integration\_changed}}}{}{}
\end{fulllineitems}

\index{optimise\_pressed() (lockin\_tab.lockin\_tab method)@\spxentry{optimise\_pressed()}\spxextra{lockin\_tab.lockin\_tab method}}

\begin{fulllineitems}
\phantomsection\label{\detokenize{index:lockin_tab.lockin_tab.optimise_pressed}}\pysiglinewithargsret{\sphinxbfcode{\sphinxupquote{optimise\_pressed}}}{}{}
\end{fulllineitems}

\index{pen\_ref (lockin\_tab.lockin\_tab attribute)@\spxentry{pen\_ref}\spxextra{lockin\_tab.lockin\_tab attribute}}

\begin{fulllineitems}
\phantomsection\label{\detokenize{index:lockin_tab.lockin_tab.pen_ref}}\pysigline{\sphinxbfcode{\sphinxupquote{pen\_ref}}\sphinxbfcode{\sphinxupquote{\DUrole{w}{  }\DUrole{p}{=}\DUrole{w}{  }\textless{}PyQt5.QtGui.QPen object\textgreater{}}}}
\end{fulllineitems}

\index{pen\_sig (lockin\_tab.lockin\_tab attribute)@\spxentry{pen\_sig}\spxextra{lockin\_tab.lockin\_tab attribute}}

\begin{fulllineitems}
\phantomsection\label{\detokenize{index:lockin_tab.lockin_tab.pen_sig}}\pysigline{\sphinxbfcode{\sphinxupquote{pen\_sig}}\sphinxbfcode{\sphinxupquote{\DUrole{w}{  }\DUrole{p}{=}\DUrole{w}{  }\textless{}PyQt5.QtGui.QPen object\textgreater{}}}}
\end{fulllineitems}

\index{periods\_changed() (lockin\_tab.lockin\_tab method)@\spxentry{periods\_changed()}\spxextra{lockin\_tab.lockin\_tab method}}

\begin{fulllineitems}
\phantomsection\label{\detokenize{index:lockin_tab.lockin_tab.periods_changed}}\pysiglinewithargsret{\sphinxbfcode{\sphinxupquote{periods\_changed}}}{}{}
\end{fulllineitems}

\index{plot\_time\_changed() (lockin\_tab.lockin\_tab method)@\spxentry{plot\_time\_changed()}\spxextra{lockin\_tab.lockin\_tab method}}

\begin{fulllineitems}
\phantomsection\label{\detokenize{index:lockin_tab.lockin_tab.plot_time_changed}}\pysiglinewithargsret{\sphinxbfcode{\sphinxupquote{plot\_time\_changed}}}{}{}
\end{fulllineitems}

\index{plotting\_timer() (lockin\_tab.lockin\_tab method)@\spxentry{plotting\_timer()}\spxextra{lockin\_tab.lockin\_tab method}}

\begin{fulllineitems}
\phantomsection\label{\detokenize{index:lockin_tab.lockin_tab.plotting_timer}}\pysiglinewithargsret{\sphinxbfcode{\sphinxupquote{plotting\_timer}}}{}{}
\end{fulllineitems}

\index{port (lockin\_tab.lockin\_tab attribute)@\spxentry{port}\spxextra{lockin\_tab.lockin\_tab attribute}}

\begin{fulllineitems}
\phantomsection\label{\detokenize{index:lockin_tab.lockin_tab.port}}\pysigline{\sphinxbfcode{\sphinxupquote{port}}\sphinxbfcode{\sphinxupquote{\DUrole{w}{  }\DUrole{p}{=}\DUrole{w}{  }65432}}}
\end{fulllineitems}

\index{ref\_color (lockin\_tab.lockin\_tab attribute)@\spxentry{ref\_color}\spxextra{lockin\_tab.lockin\_tab attribute}}

\begin{fulllineitems}
\phantomsection\label{\detokenize{index:lockin_tab.lockin_tab.ref_color}}\pysigline{\sphinxbfcode{\sphinxupquote{ref\_color}}\sphinxbfcode{\sphinxupquote{\DUrole{w}{  }\DUrole{p}{=}\DUrole{w}{  }(0, 0, 255)}}}
\end{fulllineitems}

\index{ref\_coupling\_changed() (lockin\_tab.lockin\_tab method)@\spxentry{ref\_coupling\_changed()}\spxextra{lockin\_tab.lockin\_tab method}}

\begin{fulllineitems}
\phantomsection\label{\detokenize{index:lockin_tab.lockin_tab.ref_coupling_changed}}\pysiglinewithargsret{\sphinxbfcode{\sphinxupquote{ref\_coupling\_changed}}}{}{}
\end{fulllineitems}

\index{ref\_range\_changed() (lockin\_tab.lockin\_tab method)@\spxentry{ref\_range\_changed()}\spxextra{lockin\_tab.lockin\_tab method}}

\begin{fulllineitems}
\phantomsection\label{\detokenize{index:lockin_tab.lockin_tab.ref_range_changed}}\pysiglinewithargsret{\sphinxbfcode{\sphinxupquote{ref\_range\_changed}}}{}{}
\end{fulllineitems}

\index{reference\_changed() (lockin\_tab.lockin\_tab method)@\spxentry{reference\_changed()}\spxextra{lockin\_tab.lockin\_tab method}}

\begin{fulllineitems}
\phantomsection\label{\detokenize{index:lockin_tab.lockin_tab.reference_changed}}\pysiglinewithargsret{\sphinxbfcode{\sphinxupquote{reference\_changed}}}{}{}
\end{fulllineitems}

\index{set\_demodulation\_freqs() (lockin\_tab.lockin\_tab method)@\spxentry{set\_demodulation\_freqs()}\spxextra{lockin\_tab.lockin\_tab method}}

\begin{fulllineitems}
\phantomsection\label{\detokenize{index:lockin_tab.lockin_tab.set_demodulation_freqs}}\pysiglinewithargsret{\sphinxbfcode{\sphinxupquote{set\_demodulation\_freqs}}}{}{}
\end{fulllineitems}

\index{set\_multisine() (lockin\_tab.lockin\_tab method)@\spxentry{set\_multisine()}\spxextra{lockin\_tab.lockin\_tab method}}

\begin{fulllineitems}
\phantomsection\label{\detokenize{index:lockin_tab.lockin_tab.set_multisine}}\pysiglinewithargsret{\sphinxbfcode{\sphinxupquote{set\_multisine}}}{}{}
\end{fulllineitems}

\index{set\_y\_range() (lockin\_tab.lockin\_tab method)@\spxentry{set\_y\_range()}\spxextra{lockin\_tab.lockin\_tab method}}

\begin{fulllineitems}
\phantomsection\label{\detokenize{index:lockin_tab.lockin_tab.set_y_range}}\pysiglinewithargsret{\sphinxbfcode{\sphinxupquote{set\_y\_range}}}{}{}
\end{fulllineitems}

\index{sig\_color (lockin\_tab.lockin\_tab attribute)@\spxentry{sig\_color}\spxextra{lockin\_tab.lockin\_tab attribute}}

\begin{fulllineitems}
\phantomsection\label{\detokenize{index:lockin_tab.lockin_tab.sig_color}}\pysigline{\sphinxbfcode{\sphinxupquote{sig\_color}}\sphinxbfcode{\sphinxupquote{\DUrole{w}{  }\DUrole{p}{=}\DUrole{w}{  }(255, 0, 0)}}}
\end{fulllineitems}

\index{sig\_coupling\_changed() (lockin\_tab.lockin\_tab method)@\spxentry{sig\_coupling\_changed()}\spxextra{lockin\_tab.lockin\_tab method}}

\begin{fulllineitems}
\phantomsection\label{\detokenize{index:lockin_tab.lockin_tab.sig_coupling_changed}}\pysiglinewithargsret{\sphinxbfcode{\sphinxupquote{sig\_coupling\_changed}}}{}{}
\end{fulllineitems}

\index{sig\_range\_changed() (lockin\_tab.lockin\_tab method)@\spxentry{sig\_range\_changed()}\spxextra{lockin\_tab.lockin\_tab method}}

\begin{fulllineitems}
\phantomsection\label{\detokenize{index:lockin_tab.lockin_tab.sig_range_changed}}\pysiglinewithargsret{\sphinxbfcode{\sphinxupquote{sig\_range\_changed}}}{}{}
\end{fulllineitems}

\index{sync\_tables() (lockin\_tab.lockin\_tab method)@\spxentry{sync\_tables()}\spxextra{lockin\_tab.lockin\_tab method}}

\begin{fulllineitems}
\phantomsection\label{\detokenize{index:lockin_tab.lockin_tab.sync_tables}}\pysiglinewithargsret{\sphinxbfcode{\sphinxupquote{sync\_tables}}}{}{}
\end{fulllineitems}

\index{update\_colors() (lockin\_tab.lockin\_tab method)@\spxentry{update\_colors()}\spxextra{lockin\_tab.lockin\_tab method}}

\begin{fulllineitems}
\phantomsection\label{\detokenize{index:lockin_tab.lockin_tab.update_colors}}\pysiglinewithargsret{\sphinxbfcode{\sphinxupquote{update\_colors}}}{}{}
\end{fulllineitems}

\index{update\_gui() (lockin\_tab.lockin\_tab method)@\spxentry{update\_gui()}\spxextra{lockin\_tab.lockin\_tab method}}

\begin{fulllineitems}
\phantomsection\label{\detokenize{index:lockin_tab.lockin_tab.update_gui}}\pysiglinewithargsret{\sphinxbfcode{\sphinxupquote{update\_gui}}}{}{}
\end{fulllineitems}

\index{update\_signal\_plot\_views() (lockin\_tab.lockin\_tab method)@\spxentry{update\_signal\_plot\_views()}\spxextra{lockin\_tab.lockin\_tab method}}

\begin{fulllineitems}
\phantomsection\label{\detokenize{index:lockin_tab.lockin_tab.update_signal_plot_views}}\pysiglinewithargsret{\sphinxbfcode{\sphinxupquote{update\_signal\_plot\_views}}}{}{}
\end{fulllineitems}


\end{fulllineitems}



\chapter{TiePieLCR class}
\label{\detokenize{index:module-TiePieLCR}}\label{\detokenize{index:tiepielcr-class}}\index{module@\spxentry{module}!TiePieLCR@\spxentry{TiePieLCR}}\index{TiePieLCR@\spxentry{TiePieLCR}!module@\spxentry{module}}\phantomsection\label{\detokenize{index:module-1}}\index{module@\spxentry{module}!TiePieLCR@\spxentry{TiePieLCR}}\index{TiePieLCR@\spxentry{TiePieLCR}!module@\spxentry{module}}\index{TiePieLCR (class in TiePieLCR)@\spxentry{TiePieLCR}\spxextra{class in TiePieLCR}}

\begin{fulllineitems}
\phantomsection\label{\detokenize{index:TiePieLCR.TiePieLCR}}\pysiglinewithargsret{\sphinxbfcode{\sphinxupquote{class\DUrole{w}{  }}}\sphinxcode{\sphinxupquote{TiePieLCR.}}\sphinxbfcode{\sphinxupquote{TiePieLCR}}}{\emph{\DUrole{n}{all\_settings}}, \emph{\DUrole{n}{instance}}}{}
\sphinxAtStartPar
Bases: \sphinxcode{\sphinxupquote{object}}
\index{I2C\_ADDRESS (TiePieLCR.TiePieLCR attribute)@\spxentry{I2C\_ADDRESS}\spxextra{TiePieLCR.TiePieLCR attribute}}

\begin{fulllineitems}
\phantomsection\label{\detokenize{index:TiePieLCR.TiePieLCR.I2C_ADDRESS}}\pysigline{\sphinxbfcode{\sphinxupquote{I2C\_ADDRESS}}\sphinxbfcode{\sphinxupquote{\DUrole{w}{  }\DUrole{p}{=}\DUrole{w}{  }96}}}
\sphinxAtStartPar
I$^{\text{2}}$C address of the TLC59116 chip in the TiePieLCR analog frontend

\end{fulllineitems}

\index{I2C\_CARSELECT (TiePieLCR.TiePieLCR attribute)@\spxentry{I2C\_CARSELECT}\spxextra{TiePieLCR.TiePieLCR attribute}}

\begin{fulllineitems}
\phantomsection\label{\detokenize{index:TiePieLCR.TiePieLCR.I2C_CARSELECT}}\pysigline{\sphinxbfcode{\sphinxupquote{I2C\_CARSELECT}}\sphinxbfcode{\sphinxupquote{\DUrole{w}{  }\DUrole{p}{=}\DUrole{w}{  }16}}}
\sphinxAtStartPar
Mask for selecting the signal coming directly from the function gegenerator as a reference instead a transimpedance/charge amplifier

\end{fulllineitems}

\index{I2C\_GAIN (TiePieLCR.TiePieLCR attribute)@\spxentry{I2C\_GAIN}\spxextra{TiePieLCR.TiePieLCR attribute}}

\begin{fulllineitems}
\phantomsection\label{\detokenize{index:TiePieLCR.TiePieLCR.I2C_GAIN}}\pysigline{\sphinxbfcode{\sphinxupquote{I2C\_GAIN}}\sphinxbfcode{\sphinxupquote{\DUrole{w}{  }\DUrole{p}{=}\DUrole{w}{  }65}}}
\sphinxAtStartPar
Mask for enabling the 50 times gain amplifier

\end{fulllineitems}

\index{I2C\_LEDOUT0 (TiePieLCR.TiePieLCR attribute)@\spxentry{I2C\_LEDOUT0}\spxextra{TiePieLCR.TiePieLCR attribute}}

\begin{fulllineitems}
\phantomsection\label{\detokenize{index:TiePieLCR.TiePieLCR.I2C_LEDOUT0}}\pysigline{\sphinxbfcode{\sphinxupquote{I2C\_LEDOUT0}}\sphinxbfcode{\sphinxupquote{\DUrole{w}{  }\DUrole{p}{=}\DUrole{w}{  }20}}}
\sphinxAtStartPar
Register that controls which of the four transimpedance/charge amplifiers is used

\end{fulllineitems}

\index{I2C\_LEDOUT1 (TiePieLCR.TiePieLCR attribute)@\spxentry{I2C\_LEDOUT1}\spxextra{TiePieLCR.TiePieLCR attribute}}

\begin{fulllineitems}
\phantomsection\label{\detokenize{index:TiePieLCR.TiePieLCR.I2C_LEDOUT1}}\pysigline{\sphinxbfcode{\sphinxupquote{I2C\_LEDOUT1}}\sphinxbfcode{\sphinxupquote{\DUrole{w}{  }\DUrole{p}{=}\DUrole{w}{  }22}}}
\sphinxAtStartPar
Register that controls which the gain and if the carrier used as a reference

\end{fulllineitems}

\index{I2C\_MODE1 (TiePieLCR.TiePieLCR attribute)@\spxentry{I2C\_MODE1}\spxextra{TiePieLCR.TiePieLCR attribute}}

\begin{fulllineitems}
\phantomsection\label{\detokenize{index:TiePieLCR.TiePieLCR.I2C_MODE1}}\pysigline{\sphinxbfcode{\sphinxupquote{I2C\_MODE1}}\sphinxbfcode{\sphinxupquote{\DUrole{w}{  }\DUrole{p}{=}\DUrole{w}{  }0}}}
\sphinxAtStartPar
Register that is written 0, proably to put the chip in normal mode (needs verification)

\end{fulllineitems}

\index{I2C\_NONE (TiePieLCR.TiePieLCR attribute)@\spxentry{I2C\_NONE}\spxextra{TiePieLCR.TiePieLCR attribute}}

\begin{fulllineitems}
\phantomsection\label{\detokenize{index:TiePieLCR.TiePieLCR.I2C_NONE}}\pysigline{\sphinxbfcode{\sphinxupquote{I2C\_NONE}}\sphinxbfcode{\sphinxupquote{\DUrole{w}{  }\DUrole{p}{=}\DUrole{w}{  }0}}}
\sphinxAtStartPar
Mask for selecting 1x gain or none of the transimpedance amplifiers

\end{fulllineitems}

\index{I2C\_RANGE1 (TiePieLCR.TiePieLCR attribute)@\spxentry{I2C\_RANGE1}\spxextra{TiePieLCR.TiePieLCR attribute}}

\begin{fulllineitems}
\phantomsection\label{\detokenize{index:TiePieLCR.TiePieLCR.I2C_RANGE1}}\pysigline{\sphinxbfcode{\sphinxupquote{I2C\_RANGE1}}\sphinxbfcode{\sphinxupquote{\DUrole{w}{  }\DUrole{p}{=}\DUrole{w}{  }1}}}
\sphinxAtStartPar
Mask for enabling the first (370uA/V) transimpedance amplifier

\end{fulllineitems}

\index{I2C\_RANGE2 (TiePieLCR.TiePieLCR attribute)@\spxentry{I2C\_RANGE2}\spxextra{TiePieLCR.TiePieLCR attribute}}

\begin{fulllineitems}
\phantomsection\label{\detokenize{index:TiePieLCR.TiePieLCR.I2C_RANGE2}}\pysigline{\sphinxbfcode{\sphinxupquote{I2C\_RANGE2}}\sphinxbfcode{\sphinxupquote{\DUrole{w}{  }\DUrole{p}{=}\DUrole{w}{  }4}}}
\sphinxAtStartPar
Mask for enabling the second (5uA/V) transimpedance amplifier

\end{fulllineitems}

\index{I2C\_RANGE3 (TiePieLCR.TiePieLCR attribute)@\spxentry{I2C\_RANGE3}\spxextra{TiePieLCR.TiePieLCR attribute}}

\begin{fulllineitems}
\phantomsection\label{\detokenize{index:TiePieLCR.TiePieLCR.I2C_RANGE3}}\pysigline{\sphinxbfcode{\sphinxupquote{I2C\_RANGE3}}\sphinxbfcode{\sphinxupquote{\DUrole{w}{  }\DUrole{p}{=}\DUrole{w}{  }16}}}
\sphinxAtStartPar
Mask for enabling the third (390pC/V) transimpedance/charge amplifier

\end{fulllineitems}

\index{I2C\_RANGE4 (TiePieLCR.TiePieLCR attribute)@\spxentry{I2C\_RANGE4}\spxextra{TiePieLCR.TiePieLCR attribute}}

\begin{fulllineitems}
\phantomsection\label{\detokenize{index:TiePieLCR.TiePieLCR.I2C_RANGE4}}\pysigline{\sphinxbfcode{\sphinxupquote{I2C\_RANGE4}}\sphinxbfcode{\sphinxupquote{\DUrole{w}{  }\DUrole{p}{=}\DUrole{w}{  }64}}}
\sphinxAtStartPar
Mask for enabling the third (3.9pC/V) transimpedance/charge amplifier

\end{fulllineitems}

\index{b\_bandpass (TiePieLCR.TiePieLCR attribute)@\spxentry{b\_bandpass}\spxextra{TiePieLCR.TiePieLCR attribute}}

\begin{fulllineitems}
\phantomsection\label{\detokenize{index:TiePieLCR.TiePieLCR.b_bandpass}}\pysigline{\sphinxbfcode{\sphinxupquote{b\_bandpass}}\sphinxbfcode{\sphinxupquote{\DUrole{w}{  }\DUrole{p}{=}\DUrole{w}{  }None}}}
\sphinxAtStartPar
The fir bandpass filter coefficients that will be used during the low frequent demodulation algorithm. Master tiepieLCR only.

\end{fulllineitems}

\index{bandpass\_fir\_filter() (TiePieLCR.TiePieLCR static method)@\spxentry{bandpass\_fir\_filter()}\spxextra{TiePieLCR.TiePieLCR static method}}

\begin{fulllineitems}
\phantomsection\label{\detokenize{index:TiePieLCR.TiePieLCR.bandpass_fir_filter}}\pysiglinewithargsret{\sphinxbfcode{\sphinxupquote{static\DUrole{w}{  }}}\sphinxbfcode{\sphinxupquote{bandpass\_fir\_filter}}}{\emph{\DUrole{n}{x}}, \emph{\DUrole{n}{filter\_b}}, \emph{\DUrole{n}{prev\_values}}}{}
\end{fulllineitems}

\index{base\_freq\_vector (TiePieLCR.TiePieLCR attribute)@\spxentry{base\_freq\_vector}\spxextra{TiePieLCR.TiePieLCR attribute}}

\begin{fulllineitems}
\phantomsection\label{\detokenize{index:TiePieLCR.TiePieLCR.base_freq_vector}}\pysigline{\sphinxbfcode{\sphinxupquote{base\_freq\_vector}}\sphinxbfcode{\sphinxupquote{\DUrole{w}{  }\DUrole{p}{=}\DUrole{w}{  }None}}}
\sphinxAtStartPar
A vector with the frequency of each sample in output of the fft.  Master tiepieLCR only.

\end{fulllineitems}

\index{base\_time\_vector (TiePieLCR.TiePieLCR attribute)@\spxentry{base\_time\_vector}\spxextra{TiePieLCR.TiePieLCR attribute}}

\begin{fulllineitems}
\phantomsection\label{\detokenize{index:TiePieLCR.TiePieLCR.base_time_vector}}\pysigline{\sphinxbfcode{\sphinxupquote{base\_time\_vector}}\sphinxbfcode{\sphinxupquote{\DUrole{w}{  }\DUrole{p}{=}\DUrole{w}{  }None}}}
\sphinxAtStartPar
A vector with the time at which each sample in a block of data is taken, relative to the start of the block.  Master tiepieLCR only.

\end{fulllineitems}

\index{block\_number (TiePieLCR.TiePieLCR attribute)@\spxentry{block\_number}\spxextra{TiePieLCR.TiePieLCR attribute}}

\begin{fulllineitems}
\phantomsection\label{\detokenize{index:TiePieLCR.TiePieLCR.block_number}}\pysigline{\sphinxbfcode{\sphinxupquote{block\_number}}\sphinxbfcode{\sphinxupquote{\DUrole{w}{  }\DUrole{p}{=}\DUrole{w}{  }0}}}
\sphinxAtStartPar
Number of blocks of data that have been retrieved from the TiePie oscilloscopes since the start of this measurement

\end{fulllineitems}

\index{block\_number\_process (TiePieLCR.TiePieLCR attribute)@\spxentry{block\_number\_process}\spxextra{TiePieLCR.TiePieLCR attribute}}

\begin{fulllineitems}
\phantomsection\label{\detokenize{index:TiePieLCR.TiePieLCR.block_number_process}}\pysigline{\sphinxbfcode{\sphinxupquote{block\_number\_process}}\sphinxbfcode{\sphinxupquote{\DUrole{w}{  }\DUrole{p}{=}\DUrole{w}{  }0}}}
\sphinxAtStartPar
Number of blocks of data that have been processed

\end{fulllineitems}

\index{clipping\_counter\_ref (TiePieLCR.TiePieLCR attribute)@\spxentry{clipping\_counter\_ref}\spxextra{TiePieLCR.TiePieLCR attribute}}

\begin{fulllineitems}
\phantomsection\label{\detokenize{index:TiePieLCR.TiePieLCR.clipping_counter_ref}}\pysigline{\sphinxbfcode{\sphinxupquote{clipping\_counter\_ref}}\sphinxbfcode{\sphinxupquote{\DUrole{w}{  }\DUrole{p}{=}\DUrole{w}{  }False}}}
\sphinxAtStartPar
How many blocks the reference should not be clipping before it will be considered good (and will become green). Master tiepieLCR only.

\end{fulllineitems}

\index{clipping\_counter\_sig (TiePieLCR.TiePieLCR attribute)@\spxentry{clipping\_counter\_sig}\spxextra{TiePieLCR.TiePieLCR attribute}}

\begin{fulllineitems}
\phantomsection\label{\detokenize{index:TiePieLCR.TiePieLCR.clipping_counter_sig}}\pysigline{\sphinxbfcode{\sphinxupquote{clipping\_counter\_sig}}\sphinxbfcode{\sphinxupquote{\DUrole{w}{  }\DUrole{p}{=}\DUrole{w}{  }False}}}
\sphinxAtStartPar
How many blocks the signal should not be clipping before it will be considered good (and will become green). Master tiepieLCR only.

\end{fulllineitems}

\index{close() (TiePieLCR.TiePieLCR method)@\spxentry{close()}\spxextra{TiePieLCR.TiePieLCR method}}

\begin{fulllineitems}
\phantomsection\label{\detokenize{index:TiePieLCR.TiePieLCR.close}}\pysiglinewithargsret{\sphinxbfcode{\sphinxupquote{close}}}{}{}
\end{fulllineitems}

\index{design\_bandpass\_fir\_filter() (TiePieLCR.TiePieLCR static method)@\spxentry{design\_bandpass\_fir\_filter()}\spxextra{TiePieLCR.TiePieLCR static method}}

\begin{fulllineitems}
\phantomsection\label{\detokenize{index:TiePieLCR.TiePieLCR.design_bandpass_fir_filter}}\pysiglinewithargsret{\sphinxbfcode{\sphinxupquote{static\DUrole{w}{  }}}\sphinxbfcode{\sphinxupquote{design\_bandpass\_fir\_filter}}}{\emph{\DUrole{n}{n}}, \emph{\DUrole{n}{Wn}}, \emph{\DUrole{n}{width}}}{}
\end{fulllineitems}

\index{do\_the\_ffts() (TiePieLCR.TiePieLCR method)@\spxentry{do\_the\_ffts()}\spxextra{TiePieLCR.TiePieLCR method}}

\begin{fulllineitems}
\phantomsection\label{\detokenize{index:TiePieLCR.TiePieLCR.do_the_ffts}}\pysiglinewithargsret{\sphinxbfcode{\sphinxupquote{do\_the\_ffts}}}{\emph{\DUrole{n}{input\_package}}}{}
\end{fulllineitems}

\index{filt\_win (TiePieLCR.TiePieLCR attribute)@\spxentry{filt\_win}\spxextra{TiePieLCR.TiePieLCR attribute}}

\begin{fulllineitems}
\phantomsection\label{\detokenize{index:TiePieLCR.TiePieLCR.filt_win}}\pysigline{\sphinxbfcode{\sphinxupquote{filt\_win}}\sphinxbfcode{\sphinxupquote{\DUrole{w}{  }\DUrole{p}{=}\DUrole{w}{  }None}}}
\sphinxAtStartPar
To calculate band pass filtered signals during the low frequency algorithm the inverse fourier transform of relevant fourier coefficients is taken. However because a window was used the inverse window needs to be applied to reduce noise. This vector will contain the inverse fourier transform of the fourier coeffients window used for this frequency, which can be used to apply the inverse window. Master tiepieLCR only.

\end{fulllineitems}

\index{gen (TiePieLCR.TiePieLCR attribute)@\spxentry{gen}\spxextra{TiePieLCR.TiePieLCR attribute}}

\begin{fulllineitems}
\phantomsection\label{\detokenize{index:TiePieLCR.TiePieLCR.gen}}\pysigline{\sphinxbfcode{\sphinxupquote{gen}}\sphinxbfcode{\sphinxupquote{\DUrole{w}{  }\DUrole{p}{=}\DUrole{w}{  }None}}}
\sphinxAtStartPar
A list of instances of the libtiepie Generator object. One for each tiepie. Master tiepieLCR only.

\end{fulllineitems}

\index{get\_data() (TiePieLCR.TiePieLCR method)@\spxentry{get\_data()}\spxextra{TiePieLCR.TiePieLCR method}}

\begin{fulllineitems}
\phantomsection\label{\detokenize{index:TiePieLCR.TiePieLCR.get_data}}\pysiglinewithargsret{\sphinxbfcode{\sphinxupquote{get\_data}}}{}{}
\end{fulllineitems}

\index{get\_stored\_data() (TiePieLCR.TiePieLCR method)@\spxentry{get\_stored\_data()}\spxextra{TiePieLCR.TiePieLCR method}}

\begin{fulllineitems}
\phantomsection\label{\detokenize{index:TiePieLCR.TiePieLCR.get_stored_data}}\pysiglinewithargsret{\sphinxbfcode{\sphinxupquote{get\_stored\_data}}}{}{}
\end{fulllineitems}

\index{i2c (TiePieLCR.TiePieLCR attribute)@\spxentry{i2c}\spxextra{TiePieLCR.TiePieLCR attribute}}

\begin{fulllineitems}
\phantomsection\label{\detokenize{index:TiePieLCR.TiePieLCR.i2c}}\pysigline{\sphinxbfcode{\sphinxupquote{i2c}}\sphinxbfcode{\sphinxupquote{\DUrole{w}{  }\DUrole{p}{=}\DUrole{w}{  }{[}{]}}}}
\sphinxAtStartPar
A list of instances of the libtiepie I$^{\text{2}}$C  objects. These are used to control the I2C bus that goes from the back of the tiepie to the TLC59116 chip in the TiePieLCR analog frontend. Master tiepieLCR only.

\end{fulllineitems}

\index{lockins (TiePieLCR.TiePieLCR attribute)@\spxentry{lockins}\spxextra{TiePieLCR.TiePieLCR attribute}}

\begin{fulllineitems}
\phantomsection\label{\detokenize{index:TiePieLCR.TiePieLCR.lockins}}\pysigline{\sphinxbfcode{\sphinxupquote{lockins}}\sphinxbfcode{\sphinxupquote{\DUrole{w}{  }\DUrole{p}{=}\DUrole{w}{  }0}}}
\sphinxAtStartPar
Total number of lockins that is being used, corresponds to the number of instances of the TiePieLCR class

\end{fulllineitems}

\index{low\_freq\_hilbert() (TiePieLCR.TiePieLCR static method)@\spxentry{low\_freq\_hilbert()}\spxextra{TiePieLCR.TiePieLCR static method}}

\begin{fulllineitems}
\phantomsection\label{\detokenize{index:TiePieLCR.TiePieLCR.low_freq_hilbert}}\pysiglinewithargsret{\sphinxbfcode{\sphinxupquote{static\DUrole{w}{  }}}\sphinxbfcode{\sphinxupquote{low\_freq\_hilbert}}}{\emph{\DUrole{n}{x}}, \emph{\DUrole{n}{Wn}}, \emph{\DUrole{n}{prev\_values}}}{}
\end{fulllineitems}

\index{offset\_bins (TiePieLCR.TiePieLCR attribute)@\spxentry{offset\_bins}\spxextra{TiePieLCR.TiePieLCR attribute}}

\begin{fulllineitems}
\phantomsection\label{\detokenize{index:TiePieLCR.TiePieLCR.offset_bins}}\pysigline{\sphinxbfcode{\sphinxupquote{offset\_bins}}\sphinxbfcode{\sphinxupquote{\DUrole{w}{  }\DUrole{p}{=}\DUrole{w}{  }20}}}
\sphinxAtStartPar
When the index of the lowest fft bin that contains the signal is below this value,  the low freqeuncy demodulation algorithm will be used

\end{fulllineitems}

\index{offset\_win (TiePieLCR.TiePieLCR attribute)@\spxentry{offset\_win}\spxextra{TiePieLCR.TiePieLCR attribute}}

\begin{fulllineitems}
\phantomsection\label{\detokenize{index:TiePieLCR.TiePieLCR.offset_win}}\pysigline{\sphinxbfcode{\sphinxupquote{offset\_win}}\sphinxbfcode{\sphinxupquote{\DUrole{w}{  }\DUrole{p}{=}\DUrole{w}{  }None}}}
\sphinxAtStartPar
To calculate the offset the inverse fourier transform of the lowest frequent fourier coefficients is taken. However because a window was used the inverse window needs to be applied. This vector will contain the inverse fourier transform of lowest fourier coeffients of the window, which can be used to apply the inverse window. Master tiepieLCR only.

\end{fulllineitems}

\index{old\_reference\_data (TiePieLCR.TiePieLCR attribute)@\spxentry{old\_reference\_data}\spxextra{TiePieLCR.TiePieLCR attribute}}

\begin{fulllineitems}
\phantomsection\label{\detokenize{index:TiePieLCR.TiePieLCR.old_reference_data}}\pysigline{\sphinxbfcode{\sphinxupquote{old\_reference\_data}}\sphinxbfcode{\sphinxupquote{\DUrole{w}{  }\DUrole{p}{=}\DUrole{w}{  }{[}{]}}}}
\sphinxAtStartPar
The previous package of reference data. This data will be needed to get rid of the edge effects. Master tiepieLCR only.

\end{fulllineitems}

\index{old\_signal\_data (TiePieLCR.TiePieLCR attribute)@\spxentry{old\_signal\_data}\spxextra{TiePieLCR.TiePieLCR attribute}}

\begin{fulllineitems}
\phantomsection\label{\detokenize{index:TiePieLCR.TiePieLCR.old_signal_data}}\pysigline{\sphinxbfcode{\sphinxupquote{old\_signal\_data}}\sphinxbfcode{\sphinxupquote{\DUrole{w}{  }\DUrole{p}{=}\DUrole{w}{  }{[}{]}}}}
\sphinxAtStartPar
The previous package of the signal data. This data will be needed to get rid of the edge effects. Master tiepieLCR only.

\end{fulllineitems}

\index{open\_gen() (TiePieLCR.TiePieLCR method)@\spxentry{open\_gen()}\spxextra{TiePieLCR.TiePieLCR method}}

\begin{fulllineitems}
\phantomsection\label{\detokenize{index:TiePieLCR.TiePieLCR.open_gen}}\pysiglinewithargsret{\sphinxbfcode{\sphinxupquote{open\_gen}}}{}{}
\sphinxAtStartPar
Open the connection to arbitraty waveform generators in the TiePie’s
\begin{quote}\begin{description}
\item[{Returns}] \leavevmode
\sphinxAtStartPar
True if succesfull, False if not

\item[{Return type}] \leavevmode
\sphinxAtStartPar
Boolean

\end{description}\end{quote}

\end{fulllineitems}

\index{open\_scope() (TiePieLCR.TiePieLCR method)@\spxentry{open\_scope()}\spxextra{TiePieLCR.TiePieLCR method}}

\begin{fulllineitems}
\phantomsection\label{\detokenize{index:TiePieLCR.TiePieLCR.open_scope}}\pysiglinewithargsret{\sphinxbfcode{\sphinxupquote{open\_scope}}}{}{}
\sphinxAtStartPar
Open the connection to the oscillscopes inside the TiePie’s
\begin{quote}\begin{description}
\item[{Returns}] \leavevmode
\sphinxAtStartPar
True if succesfull, False if not

\item[{Return type}] \leavevmode
\sphinxAtStartPar
Boolean

\end{description}\end{quote}

\end{fulllineitems}

\index{process\_data() (TiePieLCR.TiePieLCR method)@\spxentry{process\_data()}\spxextra{TiePieLCR.TiePieLCR method}}

\begin{fulllineitems}
\phantomsection\label{\detokenize{index:TiePieLCR.TiePieLCR.process_data}}\pysiglinewithargsret{\sphinxbfcode{\sphinxupquote{process\_data}}}{\emph{\DUrole{n}{input\_package}}}{}
\end{fulllineitems}

\index{reset\_buffers() (TiePieLCR.TiePieLCR method)@\spxentry{reset\_buffers()}\spxextra{TiePieLCR.TiePieLCR method}}

\begin{fulllineitems}
\phantomsection\label{\detokenize{index:TiePieLCR.TiePieLCR.reset_buffers}}\pysiglinewithargsret{\sphinxbfcode{\sphinxupquote{reset\_buffers}}}{}{}
\end{fulllineitems}

\index{rfft\_convolute() (TiePieLCR.TiePieLCR static method)@\spxentry{rfft\_convolute()}\spxextra{TiePieLCR.TiePieLCR static method}}

\begin{fulllineitems}
\phantomsection\label{\detokenize{index:TiePieLCR.TiePieLCR.rfft_convolute}}\pysiglinewithargsret{\sphinxbfcode{\sphinxupquote{static\DUrole{w}{  }}}\sphinxbfcode{\sphinxupquote{rfft\_convolute}}}{\emph{\DUrole{n}{sig1}}, \emph{\DUrole{n}{sig2}}}{}
\end{fulllineitems}

\index{scp (TiePieLCR.TiePieLCR attribute)@\spxentry{scp}\spxextra{TiePieLCR.TiePieLCR attribute}}

\begin{fulllineitems}
\phantomsection\label{\detokenize{index:TiePieLCR.TiePieLCR.scp}}\pysigline{\sphinxbfcode{\sphinxupquote{scp}}\sphinxbfcode{\sphinxupquote{\DUrole{w}{  }\DUrole{p}{=}\DUrole{w}{  }None}}}
\sphinxAtStartPar
The libtiepie oscilloscope object. Used to get data from the TiePie HS5’s

\end{fulllineitems}

\index{select\_LCR\_gain() (TiePieLCR.TiePieLCR method)@\spxentry{select\_LCR\_gain()}\spxextra{TiePieLCR.TiePieLCR method}}

\begin{fulllineitems}
\phantomsection\label{\detokenize{index:TiePieLCR.TiePieLCR.select_LCR_gain}}\pysiglinewithargsret{\sphinxbfcode{\sphinxupquote{select\_LCR\_gain}}}{\emph{\DUrole{n}{instance}}, \emph{\DUrole{n}{reference\_setting}}, \emph{\DUrole{n}{gain\_setting}}}{}
\end{fulllineitems}

\index{select\_reference() (TiePieLCR.TiePieLCR method)@\spxentry{select\_reference()}\spxextra{TiePieLCR.TiePieLCR method}}

\begin{fulllineitems}
\phantomsection\label{\detokenize{index:TiePieLCR.TiePieLCR.select_reference}}\pysiglinewithargsret{\sphinxbfcode{\sphinxupquote{select\_reference}}}{\emph{\DUrole{n}{instance}}, \emph{\DUrole{n}{reference\_setting}}, \emph{\DUrole{n}{gain\_setting}}}{}
\end{fulllineitems}

\index{serial\_numbers (TiePieLCR.TiePieLCR attribute)@\spxentry{serial\_numbers}\spxextra{TiePieLCR.TiePieLCR attribute}}

\begin{fulllineitems}
\phantomsection\label{\detokenize{index:TiePieLCR.TiePieLCR.serial_numbers}}\pysigline{\sphinxbfcode{\sphinxupquote{serial\_numbers}}\sphinxbfcode{\sphinxupquote{\DUrole{w}{  }\DUrole{p}{=}\DUrole{w}{  }{[}{]}}}}
\sphinxAtStartPar
After connecting this variable will contain the serial numbers of the connected TiePie HS5’s.  Master tiepieLCR only.

\end{fulllineitems}

\index{set\_settings() (TiePieLCR.TiePieLCR method)@\spxentry{set\_settings()}\spxextra{TiePieLCR.TiePieLCR method}}

\begin{fulllineitems}
\phantomsection\label{\detokenize{index:TiePieLCR.TiePieLCR.set_settings}}\pysiglinewithargsret{\sphinxbfcode{\sphinxupquote{set\_settings}}}{\emph{\DUrole{n}{all\_settings}\DUrole{p}{:}\DUrole{w}{  }\DUrole{n}{list\DUrole{p}{{[}}{\hyperref[\detokenize{index:TiePieLCR_settings.TiePieLCR_settings}]{\sphinxcrossref{TiePieLCR\_settings.TiePieLCR\_settings}}}\DUrole{p}{{]}}}}, \emph{\DUrole{n}{instance}}, \emph{\DUrole{n}{master\_lcr}\DUrole{p}{:}\DUrole{w}{  }\DUrole{n}{{\hyperref[\detokenize{index:TiePieLCR.TiePieLCR}]{\sphinxcrossref{TiePieLCR.TiePieLCR}}}}}}{}
\sphinxAtStartPar
(re)Initialise the coefficients of the band and low pass filters as well as the processed window functions that are used for the inverse window
\begin{quote}\begin{description}
\item[{Parameters}] \leavevmode\begin{itemize}
\item {} 
\sphinxAtStartPar
\sphinxstyleliteralstrong{\sphinxupquote{all\_settings}} \textendash{} A list of the TiePieLCR\_settings objects that contain the settings for this lockin

\item {} 
\sphinxAtStartPar
\sphinxstyleliteralstrong{\sphinxupquote{instance}} \textendash{} The index in the all\_settings list that belongs to this tiepieLCR’s settings

\item {} 
\sphinxAtStartPar
\sphinxstyleliteralstrong{\sphinxupquote{master\_lcr}} \textendash{} The TiePieLCR object that will be used get the data

\end{itemize}

\item[{Returns}] \leavevmode
\sphinxAtStartPar
None

\item[{Return type}] \leavevmode
\sphinxAtStartPar
None

\end{description}\end{quote}

\end{fulllineitems}

\index{settings (TiePieLCR.TiePieLCR attribute)@\spxentry{settings}\spxextra{TiePieLCR.TiePieLCR attribute}}

\begin{fulllineitems}
\phantomsection\label{\detokenize{index:TiePieLCR.TiePieLCR.settings}}\pysigline{\sphinxbfcode{\sphinxupquote{settings}}\sphinxbfcode{\sphinxupquote{\DUrole{w}{  }\DUrole{p}{=}\DUrole{w}{  }\textless{}TiePieLCR\_settings.TiePieLCR\_settings object\textgreater{}}}}
\sphinxAtStartPar
An instance of the TiePieLCR class that will contain the settings of this TiePieLCR

\end{fulllineitems}

\index{sos\_decimate (TiePieLCR.TiePieLCR attribute)@\spxentry{sos\_decimate}\spxextra{TiePieLCR.TiePieLCR attribute}}

\begin{fulllineitems}
\phantomsection\label{\detokenize{index:TiePieLCR.TiePieLCR.sos_decimate}}\pysigline{\sphinxbfcode{\sphinxupquote{sos\_decimate}}\sphinxbfcode{\sphinxupquote{\DUrole{w}{  }\DUrole{p}{=}\DUrole{w}{  }None}}}
\sphinxAtStartPar
The filter coefficients that will be used for the final low pass filter of both algorithms. Master tiepieLCR only.

\end{fulllineitems}

\index{sos\_offset (TiePieLCR.TiePieLCR attribute)@\spxentry{sos\_offset}\spxextra{TiePieLCR.TiePieLCR attribute}}

\begin{fulllineitems}
\phantomsection\label{\detokenize{index:TiePieLCR.TiePieLCR.sos_offset}}\pysigline{\sphinxbfcode{\sphinxupquote{sos\_offset}}\sphinxbfcode{\sphinxupquote{\DUrole{w}{  }\DUrole{p}{=}\DUrole{w}{  }None}}}
\sphinxAtStartPar
The lowpass filter coefficients that will be applied to the offset. Master tiepieLCR only.

\end{fulllineitems}

\index{start\_awg() (TiePieLCR.TiePieLCR method)@\spxentry{start\_awg()}\spxextra{TiePieLCR.TiePieLCR method}}

\begin{fulllineitems}
\phantomsection\label{\detokenize{index:TiePieLCR.TiePieLCR.start_awg}}\pysiglinewithargsret{\sphinxbfcode{\sphinxupquote{start\_awg}}}{\emph{\DUrole{n}{instance}}, \emph{\DUrole{n}{data}}}{}
\end{fulllineitems}

\index{start\_measurement() (TiePieLCR.TiePieLCR method)@\spxentry{start\_measurement()}\spxextra{TiePieLCR.TiePieLCR method}}

\begin{fulllineitems}
\phantomsection\label{\detokenize{index:TiePieLCR.TiePieLCR.start_measurement}}\pysiglinewithargsret{\sphinxbfcode{\sphinxupquote{start\_measurement}}}{\emph{\DUrole{n}{all\_settings}}}{}
\end{fulllineitems}

\index{start\_stream() (TiePieLCR.TiePieLCR method)@\spxentry{start\_stream()}\spxextra{TiePieLCR.TiePieLCR method}}

\begin{fulllineitems}
\phantomsection\label{\detokenize{index:TiePieLCR.TiePieLCR.start_stream}}\pysiglinewithargsret{\sphinxbfcode{\sphinxupquote{start\_stream}}}{\emph{\DUrole{n}{all\_settings}}}{}
\end{fulllineitems}

\index{state (TiePieLCR.TiePieLCR attribute)@\spxentry{state}\spxextra{TiePieLCR.TiePieLCR attribute}}

\begin{fulllineitems}
\phantomsection\label{\detokenize{index:TiePieLCR.TiePieLCR.state}}\pysigline{\sphinxbfcode{\sphinxupquote{state}}\sphinxbfcode{\sphinxupquote{\DUrole{w}{  }\DUrole{p}{=}\DUrole{w}{  }0}}}
\sphinxAtStartPar
State the system. The state is only set for the tiepieLCR instance that is used for getting the data (often called masterLCR)
\begin{itemize}
\item {} 
\sphinxAtStartPar
state 0: Not initialised

\item {} 
\sphinxAtStartPar
state 1: Initialising

\item {} 
\sphinxAtStartPar
state 2: Running

\item {} 
\sphinxAtStartPar
state 3: Stopping

\item {} 
\sphinxAtStartPar
state 4: Updating

\end{itemize}

\end{fulllineitems}

\index{stop\_gen() (TiePieLCR.TiePieLCR method)@\spxentry{stop\_gen()}\spxextra{TiePieLCR.TiePieLCR method}}

\begin{fulllineitems}
\phantomsection\label{\detokenize{index:TiePieLCR.TiePieLCR.stop_gen}}\pysiglinewithargsret{\sphinxbfcode{\sphinxupquote{stop\_gen}}}{\emph{\DUrole{n}{inst}}, \emph{\DUrole{n}{disable\_output}}}{}
\end{fulllineitems}

\index{stop\_measurement() (TiePieLCR.TiePieLCR method)@\spxentry{stop\_measurement()}\spxextra{TiePieLCR.TiePieLCR method}}

\begin{fulllineitems}
\phantomsection\label{\detokenize{index:TiePieLCR.TiePieLCR.stop_measurement}}\pysiglinewithargsret{\sphinxbfcode{\sphinxupquote{stop\_measurement}}}{}{}
\end{fulllineitems}

\index{stop\_stream() (TiePieLCR.TiePieLCR method)@\spxentry{stop\_stream()}\spxextra{TiePieLCR.TiePieLCR method}}

\begin{fulllineitems}
\phantomsection\label{\detokenize{index:TiePieLCR.TiePieLCR.stop_stream}}\pysiglinewithargsret{\sphinxbfcode{\sphinxupquote{stop\_stream}}}{}{}
\end{fulllineitems}

\index{update\_base\_vectors() (TiePieLCR.TiePieLCR method)@\spxentry{update\_base\_vectors()}\spxextra{TiePieLCR.TiePieLCR method}}

\begin{fulllineitems}
\phantomsection\label{\detokenize{index:TiePieLCR.TiePieLCR.update_base_vectors}}\pysiglinewithargsret{\sphinxbfcode{\sphinxupquote{update\_base\_vectors}}}{}{}
\sphinxAtStartPar
(re)Initialise some vectors that are needed for the demodulation algorithm, but depend on the settings.
\begin{quote}\begin{description}
\item[{Returns}] \leavevmode
\sphinxAtStartPar
None

\item[{Return type}] \leavevmode
\sphinxAtStartPar
None

\end{description}\end{quote}

\end{fulllineitems}

\index{update\_filter\_coefs() (TiePieLCR.TiePieLCR method)@\spxentry{update\_filter\_coefs()}\spxextra{TiePieLCR.TiePieLCR method}}

\begin{fulllineitems}
\phantomsection\label{\detokenize{index:TiePieLCR.TiePieLCR.update_filter_coefs}}\pysiglinewithargsret{\sphinxbfcode{\sphinxupquote{update\_filter\_coefs}}}{}{}
\sphinxAtStartPar
(re)Initialise the coefficients of the band and low pass filters as well as the processed window functions that are used for the inverse window
\begin{quote}\begin{description}
\item[{Returns}] \leavevmode
\sphinxAtStartPar
None

\item[{Return type}] \leavevmode
\sphinxAtStartPar
None

\end{description}\end{quote}

\end{fulllineitems}

\index{window\_cpu (TiePieLCR.TiePieLCR attribute)@\spxentry{window\_cpu}\spxextra{TiePieLCR.TiePieLCR attribute}}

\begin{fulllineitems}
\phantomsection\label{\detokenize{index:TiePieLCR.TiePieLCR.window_cpu}}\pysigline{\sphinxbfcode{\sphinxupquote{window\_cpu}}\sphinxbfcode{\sphinxupquote{\DUrole{w}{  }\DUrole{p}{=}\DUrole{w}{  }None}}}
\sphinxAtStartPar
The window that will be applied to a block of data before applying the fft and is located in the CPU memory. Master tiepieLCR only.

\end{fulllineitems}

\index{window\_gpu (TiePieLCR.TiePieLCR attribute)@\spxentry{window\_gpu}\spxextra{TiePieLCR.TiePieLCR attribute}}

\begin{fulllineitems}
\phantomsection\label{\detokenize{index:TiePieLCR.TiePieLCR.window_gpu}}\pysigline{\sphinxbfcode{\sphinxupquote{window\_gpu}}\sphinxbfcode{\sphinxupquote{\DUrole{w}{  }\DUrole{p}{=}\DUrole{w}{  }None}}}
\sphinxAtStartPar
The window that will be applied to a block of data before applying the fft and is located in the GPU memory. Master tiepieLCR only.

\end{fulllineitems}


\end{fulllineitems}



\chapter{acquisition class}
\label{\detokenize{index:module-acquisition}}\label{\detokenize{index:acquisition-class}}\index{module@\spxentry{module}!acquisition@\spxentry{acquisition}}\index{acquisition@\spxentry{acquisition}!module@\spxentry{module}}\phantomsection\label{\detokenize{index:module-2}}\index{module@\spxentry{module}!acquisition@\spxentry{acquisition}}\index{acquisition@\spxentry{acquisition}!module@\spxentry{module}}\index{acquisition (class in acquisition)@\spxentry{acquisition}\spxextra{class in acquisition}}

\begin{fulllineitems}
\phantomsection\label{\detokenize{index:acquisition.acquisition}}\pysiglinewithargsret{\sphinxbfcode{\sphinxupquote{class\DUrole{w}{  }}}\sphinxcode{\sphinxupquote{acquisition.}}\sphinxbfcode{\sphinxupquote{acquisition}}}{\emph{\DUrole{n}{LCR\_settings\_queue}}, \emph{\DUrole{n}{displayed\_error}}, \emph{\DUrole{n}{shared\_state}}, \emph{\DUrole{n}{plot\_data\_queue}}, \emph{\DUrole{n}{stored\_data\_queue}}, \emph{\DUrole{n}{stored\_data\_requested}}}{}
\sphinxAtStartPar
Bases: \sphinxcode{\sphinxupquote{object}}
\index{build\_tiepie\_list() (acquisition.acquisition method)@\spxentry{build\_tiepie\_list()}\spxextra{acquisition.acquisition method}}

\begin{fulllineitems}
\phantomsection\label{\detokenize{index:acquisition.acquisition.build_tiepie_list}}\pysiglinewithargsret{\sphinxbfcode{\sphinxupquote{build\_tiepie\_list}}}{}{}
\end{fulllineitems}

\index{data\_aquisition\_daemon() (acquisition.acquisition method)@\spxentry{data\_aquisition\_daemon()}\spxextra{acquisition.acquisition method}}

\begin{fulllineitems}
\phantomsection\label{\detokenize{index:acquisition.acquisition.data_aquisition_daemon}}\pysiglinewithargsret{\sphinxbfcode{\sphinxupquote{data\_aquisition\_daemon}}}{}{}
\end{fulllineitems}

\index{fft\_daemon() (acquisition.acquisition method)@\spxentry{fft\_daemon()}\spxextra{acquisition.acquisition method}}

\begin{fulllineitems}
\phantomsection\label{\detokenize{index:acquisition.acquisition.fft_daemon}}\pysiglinewithargsret{\sphinxbfcode{\sphinxupquote{fft\_daemon}}}{}{}
\end{fulllineitems}

\index{fft\_package\_list (acquisition.acquisition attribute)@\spxentry{fft\_package\_list}\spxextra{acquisition.acquisition attribute}}

\begin{fulllineitems}
\phantomsection\label{\detokenize{index:acquisition.acquisition.fft_package_list}}\pysigline{\sphinxbfcode{\sphinxupquote{fft\_package\_list}}\sphinxbfcode{\sphinxupquote{\DUrole{w}{  }\DUrole{p}{=}\DUrole{w}{  }{[}{]}}}}
\end{fulllineitems}

\index{package\_list (acquisition.acquisition attribute)@\spxentry{package\_list}\spxextra{acquisition.acquisition attribute}}

\begin{fulllineitems}
\phantomsection\label{\detokenize{index:acquisition.acquisition.package_list}}\pysigline{\sphinxbfcode{\sphinxupquote{package\_list}}\sphinxbfcode{\sphinxupquote{\DUrole{w}{  }\DUrole{p}{=}\DUrole{w}{  }{[}{]}}}}
\end{fulllineitems}

\index{processing\_daemon() (acquisition.acquisition method)@\spxentry{processing\_daemon()}\spxextra{acquisition.acquisition method}}

\begin{fulllineitems}
\phantomsection\label{\detokenize{index:acquisition.acquisition.processing_daemon}}\pysiglinewithargsret{\sphinxbfcode{\sphinxupquote{processing\_daemon}}}{}{}
\end{fulllineitems}


\end{fulllineitems}



\chapter{TiePieLCR settings class}
\label{\detokenize{index:module-TiePieLCR_settings}}\label{\detokenize{index:tiepielcr-settings-class}}\index{module@\spxentry{module}!TiePieLCR\_settings@\spxentry{TiePieLCR\_settings}}\index{TiePieLCR\_settings@\spxentry{TiePieLCR\_settings}!module@\spxentry{module}}\phantomsection\label{\detokenize{index:module-3}}\index{module@\spxentry{module}!TiePieLCR\_settings@\spxentry{TiePieLCR\_settings}}\index{TiePieLCR\_settings@\spxentry{TiePieLCR\_settings}!module@\spxentry{module}}\index{TiePieLCR\_settings (class in TiePieLCR\_settings)@\spxentry{TiePieLCR\_settings}\spxextra{class in TiePieLCR\_settings}}

\begin{fulllineitems}
\phantomsection\label{\detokenize{index:TiePieLCR_settings.TiePieLCR_settings}}\pysigline{\sphinxbfcode{\sphinxupquote{class\DUrole{w}{  }}}\sphinxcode{\sphinxupquote{TiePieLCR\_settings.}}\sphinxbfcode{\sphinxupquote{TiePieLCR\_settings}}}
\sphinxAtStartPar
Bases: \sphinxcode{\sphinxupquote{object}}
\index{Vmax (TiePieLCR\_settings.TiePieLCR\_settings attribute)@\spxentry{Vmax}\spxextra{TiePieLCR\_settings.TiePieLCR\_settings attribute}}

\begin{fulllineitems}
\phantomsection\label{\detokenize{index:TiePieLCR_settings.TiePieLCR_settings.Vmax}}\pysigline{\sphinxbfcode{\sphinxupquote{Vmax}}\sphinxbfcode{\sphinxupquote{\DUrole{w}{  }\DUrole{p}{=}\DUrole{w}{  }0.8}}}
\end{fulllineitems}

\index{bandwidth (TiePieLCR\_settings.TiePieLCR\_settings attribute)@\spxentry{bandwidth}\spxextra{TiePieLCR\_settings.TiePieLCR\_settings attribute}}

\begin{fulllineitems}
\phantomsection\label{\detokenize{index:TiePieLCR_settings.TiePieLCR_settings.bandwidth}}\pysigline{\sphinxbfcode{\sphinxupquote{bandwidth}}\sphinxbfcode{\sphinxupquote{\DUrole{w}{  }\DUrole{p}{=}\DUrole{w}{  }20}}}
\end{fulllineitems}

\index{base\_vector\_update\_required (TiePieLCR\_settings.TiePieLCR\_settings attribute)@\spxentry{base\_vector\_update\_required}\spxextra{TiePieLCR\_settings.TiePieLCR\_settings attribute}}

\begin{fulllineitems}
\phantomsection\label{\detokenize{index:TiePieLCR_settings.TiePieLCR_settings.base_vector_update_required}}\pysigline{\sphinxbfcode{\sphinxupquote{base\_vector\_update\_required}}\sphinxbfcode{\sphinxupquote{\DUrole{w}{  }\DUrole{p}{=}\DUrole{w}{  }False}}}
\end{fulllineitems}

\index{calculate\_downsapling\_rate() (TiePieLCR\_settings.TiePieLCR\_settings static method)@\spxentry{calculate\_downsapling\_rate()}\spxextra{TiePieLCR\_settings.TiePieLCR\_settings static method}}

\begin{fulllineitems}
\phantomsection\label{\detokenize{index:TiePieLCR_settings.TiePieLCR_settings.calculate_downsapling_rate}}\pysiglinewithargsret{\sphinxbfcode{\sphinxupquote{static\DUrole{w}{  }}}\sphinxbfcode{\sphinxupquote{calculate\_downsapling\_rate}}}{\emph{\DUrole{n}{n}}, \emph{\DUrole{n}{Wn}}}{}
\sphinxAtStartPar
Calculate the maximum possible downsampling rate
\begin{quote}\begin{description}
\item[{Returns}] \leavevmode
\sphinxAtStartPar
The downsampling rate

\item[{Return type}] \leavevmode
\sphinxAtStartPar
Complex double

\end{description}\end{quote}

\end{fulllineitems}

\index{calculate\_gen\_data\_size() (TiePieLCR\_settings.TiePieLCR\_settings method)@\spxentry{calculate\_gen\_data\_size()}\spxextra{TiePieLCR\_settings.TiePieLCR\_settings method}}

\begin{fulllineitems}
\phantomsection\label{\detokenize{index:TiePieLCR_settings.TiePieLCR_settings.calculate_gen_data_size}}\pysiglinewithargsret{\sphinxbfcode{\sphinxupquote{calculate\_gen\_data\_size}}}{}{}
\end{fulllineitems}

\index{crest\_factor\_cost\_function() (TiePieLCR\_settings.TiePieLCR\_settings method)@\spxentry{crest\_factor\_cost\_function()}\spxextra{TiePieLCR\_settings.TiePieLCR\_settings method}}

\begin{fulllineitems}
\phantomsection\label{\detokenize{index:TiePieLCR_settings.TiePieLCR_settings.crest_factor_cost_function}}\pysiglinewithargsret{\sphinxbfcode{\sphinxupquote{crest\_factor\_cost\_function}}}{\emph{\DUrole{n}{phases}}, \emph{\DUrole{n}{freqs}}, \emph{\DUrole{n}{weights}}, \emph{\DUrole{n}{t}}, \emph{\DUrole{n}{device}}}{}
\end{fulllineitems}

\index{dem\_freq\_update\_required (TiePieLCR\_settings.TiePieLCR\_settings attribute)@\spxentry{dem\_freq\_update\_required}\spxextra{TiePieLCR\_settings.TiePieLCR\_settings attribute}}

\begin{fulllineitems}
\phantomsection\label{\detokenize{index:TiePieLCR_settings.TiePieLCR_settings.dem_freq_update_required}}\pysigline{\sphinxbfcode{\sphinxupquote{dem\_freq\_update\_required}}\sphinxbfcode{\sphinxupquote{\DUrole{w}{  }\DUrole{p}{=}\DUrole{w}{  }False}}}
\end{fulllineitems}

\index{dem\_freqs (TiePieLCR\_settings.TiePieLCR\_settings attribute)@\spxentry{dem\_freqs}\spxextra{TiePieLCR\_settings.TiePieLCR\_settings attribute}}

\begin{fulllineitems}
\phantomsection\label{\detokenize{index:TiePieLCR_settings.TiePieLCR_settings.dem_freqs}}\pysigline{\sphinxbfcode{\sphinxupquote{dem\_freqs}}\sphinxbfcode{\sphinxupquote{\DUrole{w}{  }\DUrole{p}{=}\DUrole{w}{  }{[}5000{]}}}}
\end{fulllineitems}

\index{dem\_is\_gen (TiePieLCR\_settings.TiePieLCR\_settings attribute)@\spxentry{dem\_is\_gen}\spxextra{TiePieLCR\_settings.TiePieLCR\_settings attribute}}

\begin{fulllineitems}
\phantomsection\label{\detokenize{index:TiePieLCR_settings.TiePieLCR_settings.dem_is_gen}}\pysigline{\sphinxbfcode{\sphinxupquote{dem\_is\_gen}}\sphinxbfcode{\sphinxupquote{\DUrole{w}{  }\DUrole{p}{=}\DUrole{w}{  }True}}}
\end{fulllineitems}

\index{dem\_tiepies (TiePieLCR\_settings.TiePieLCR\_settings attribute)@\spxentry{dem\_tiepies}\spxextra{TiePieLCR\_settings.TiePieLCR\_settings attribute}}

\begin{fulllineitems}
\phantomsection\label{\detokenize{index:TiePieLCR_settings.TiePieLCR_settings.dem_tiepies}}\pysigline{\sphinxbfcode{\sphinxupquote{dem\_tiepies}}\sphinxbfcode{\sphinxupquote{\DUrole{w}{  }\DUrole{p}{=}\DUrole{w}{  }{[}1{]}}}}
\end{fulllineitems}

\index{demodulate\_plot\_points (TiePieLCR\_settings.TiePieLCR\_settings attribute)@\spxentry{demodulate\_plot\_points}\spxextra{TiePieLCR\_settings.TiePieLCR\_settings attribute}}

\begin{fulllineitems}
\phantomsection\label{\detokenize{index:TiePieLCR_settings.TiePieLCR_settings.demodulate_plot_points}}\pysigline{\sphinxbfcode{\sphinxupquote{demodulate\_plot\_points}}\sphinxbfcode{\sphinxupquote{\DUrole{w}{  }\DUrole{p}{=}\DUrole{w}{  }500}}}
\end{fulllineitems}

\index{enabled (TiePieLCR\_settings.TiePieLCR\_settings attribute)@\spxentry{enabled}\spxextra{TiePieLCR\_settings.TiePieLCR\_settings attribute}}

\begin{fulllineitems}
\phantomsection\label{\detokenize{index:TiePieLCR_settings.TiePieLCR_settings.enabled}}\pysigline{\sphinxbfcode{\sphinxupquote{enabled}}\sphinxbfcode{\sphinxupquote{\DUrole{w}{  }\DUrole{p}{=}\DUrole{w}{  }{[}True, True{]}}}}
\end{fulllineitems}

\index{f\_fun (TiePieLCR\_settings.TiePieLCR\_settings attribute)@\spxentry{f\_fun}\spxextra{TiePieLCR\_settings.TiePieLCR\_settings attribute}}

\begin{fulllineitems}
\phantomsection\label{\detokenize{index:TiePieLCR_settings.TiePieLCR_settings.f_fun}}\pysigline{\sphinxbfcode{\sphinxupquote{f\_fun}}\sphinxbfcode{\sphinxupquote{\DUrole{w}{  }\DUrole{p}{=}\DUrole{w}{  }0}}}
\end{fulllineitems}

\index{f\_max (TiePieLCR\_settings.TiePieLCR\_settings attribute)@\spxentry{f\_max}\spxextra{TiePieLCR\_settings.TiePieLCR\_settings attribute}}

\begin{fulllineitems}
\phantomsection\label{\detokenize{index:TiePieLCR_settings.TiePieLCR_settings.f_max}}\pysigline{\sphinxbfcode{\sphinxupquote{f\_max}}\sphinxbfcode{\sphinxupquote{\DUrole{w}{  }\DUrole{p}{=}\DUrole{w}{  }0}}}
\end{fulllineitems}

\index{f\_min (TiePieLCR\_settings.TiePieLCR\_settings attribute)@\spxentry{f\_min}\spxextra{TiePieLCR\_settings.TiePieLCR\_settings attribute}}

\begin{fulllineitems}
\phantomsection\label{\detokenize{index:TiePieLCR_settings.TiePieLCR_settings.f_min}}\pysigline{\sphinxbfcode{\sphinxupquote{f\_min}}\sphinxbfcode{\sphinxupquote{\DUrole{w}{  }\DUrole{p}{=}\DUrole{w}{  }0}}}
\end{fulllineitems}

\index{fft\_plot\_points (TiePieLCR\_settings.TiePieLCR\_settings attribute)@\spxentry{fft\_plot\_points}\spxextra{TiePieLCR\_settings.TiePieLCR\_settings attribute}}

\begin{fulllineitems}
\phantomsection\label{\detokenize{index:TiePieLCR_settings.TiePieLCR_settings.fft_plot_points}}\pysigline{\sphinxbfcode{\sphinxupquote{fft\_plot\_points}}\sphinxbfcode{\sphinxupquote{\DUrole{w}{  }\DUrole{p}{=}\DUrole{w}{  }500}}}
\end{fulllineitems}

\index{fft\_sensitivity (TiePieLCR\_settings.TiePieLCR\_settings attribute)@\spxentry{fft\_sensitivity}\spxextra{TiePieLCR\_settings.TiePieLCR\_settings attribute}}

\begin{fulllineitems}
\phantomsection\label{\detokenize{index:TiePieLCR_settings.TiePieLCR_settings.fft_sensitivity}}\pysigline{\sphinxbfcode{\sphinxupquote{fft\_sensitivity}}\sphinxbfcode{\sphinxupquote{\DUrole{w}{  }\DUrole{p}{=}\DUrole{w}{  }0}}}
\end{fulllineitems}

\index{fmax\_plot (TiePieLCR\_settings.TiePieLCR\_settings attribute)@\spxentry{fmax\_plot}\spxextra{TiePieLCR\_settings.TiePieLCR\_settings attribute}}

\begin{fulllineitems}
\phantomsection\label{\detokenize{index:TiePieLCR_settings.TiePieLCR_settings.fmax_plot}}\pysigline{\sphinxbfcode{\sphinxupquote{fmax\_plot}}\sphinxbfcode{\sphinxupquote{\DUrole{w}{  }\DUrole{p}{=}\DUrole{w}{  }1000000}}}
\end{fulllineitems}

\index{fmin\_plot (TiePieLCR\_settings.TiePieLCR\_settings attribute)@\spxentry{fmin\_plot}\spxextra{TiePieLCR\_settings.TiePieLCR\_settings attribute}}

\begin{fulllineitems}
\phantomsection\label{\detokenize{index:TiePieLCR_settings.TiePieLCR_settings.fmin_plot}}\pysigline{\sphinxbfcode{\sphinxupquote{fmin\_plot}}\sphinxbfcode{\sphinxupquote{\DUrole{w}{  }\DUrole{p}{=}\DUrole{w}{  }100}}}
\end{fulllineitems}

\index{fs\_list (TiePieLCR\_settings.TiePieLCR\_settings attribute)@\spxentry{fs\_list}\spxextra{TiePieLCR\_settings.TiePieLCR\_settings attribute}}

\begin{fulllineitems}
\phantomsection\label{\detokenize{index:TiePieLCR_settings.TiePieLCR_settings.fs_list}}\pysigline{\sphinxbfcode{\sphinxupquote{fs\_list}}\sphinxbfcode{\sphinxupquote{\DUrole{w}{  }\DUrole{p}{=}\DUrole{w}{  }{[}6250000, 3125000, 1562500, 781250{]}}}}
\end{fulllineitems}

\index{fs\_name\_list (TiePieLCR\_settings.TiePieLCR\_settings attribute)@\spxentry{fs\_name\_list}\spxextra{TiePieLCR\_settings.TiePieLCR\_settings attribute}}

\begin{fulllineitems}
\phantomsection\label{\detokenize{index:TiePieLCR_settings.TiePieLCR_settings.fs_name_list}}\pysigline{\sphinxbfcode{\sphinxupquote{fs\_name\_list}}\sphinxbfcode{\sphinxupquote{\DUrole{w}{  }\DUrole{p}{=}\DUrole{w}{  }{[}\textquotesingle{}6250000\textquotesingle{}, \textquotesingle{}3125000\textquotesingle{}, \textquotesingle{}1562500\textquotesingle{}, \textquotesingle{}781250\textquotesingle{}{]}}}}
\end{fulllineitems}

\index{fs\_setting (TiePieLCR\_settings.TiePieLCR\_settings attribute)@\spxentry{fs\_setting}\spxextra{TiePieLCR\_settings.TiePieLCR\_settings attribute}}

\begin{fulllineitems}
\phantomsection\label{\detokenize{index:TiePieLCR_settings.TiePieLCR_settings.fs_setting}}\pysigline{\sphinxbfcode{\sphinxupquote{fs\_setting}}\sphinxbfcode{\sphinxupquote{\DUrole{w}{  }\DUrole{p}{=}\DUrole{w}{  }1}}}
\end{fulllineitems}

\index{gain\_list (TiePieLCR\_settings.TiePieLCR\_settings attribute)@\spxentry{gain\_list}\spxextra{TiePieLCR\_settings.TiePieLCR\_settings attribute}}

\begin{fulllineitems}
\phantomsection\label{\detokenize{index:TiePieLCR_settings.TiePieLCR_settings.gain_list}}\pysigline{\sphinxbfcode{\sphinxupquote{gain\_list}}\sphinxbfcode{\sphinxupquote{\DUrole{w}{  }\DUrole{p}{=}\DUrole{w}{  }{[}1, 50{]}}}}
\end{fulllineitems}

\index{gain\_name\_list (TiePieLCR\_settings.TiePieLCR\_settings attribute)@\spxentry{gain\_name\_list}\spxextra{TiePieLCR\_settings.TiePieLCR\_settings attribute}}

\begin{fulllineitems}
\phantomsection\label{\detokenize{index:TiePieLCR_settings.TiePieLCR_settings.gain_name_list}}\pysigline{\sphinxbfcode{\sphinxupquote{gain\_name\_list}}\sphinxbfcode{\sphinxupquote{\DUrole{w}{  }\DUrole{p}{=}\DUrole{w}{  }{[}\textquotesingle{}1x\textquotesingle{}, \textquotesingle{}50x\textquotesingle{}{]}}}}
\end{fulllineitems}

\index{gain\_setting (TiePieLCR\_settings.TiePieLCR\_settings attribute)@\spxentry{gain\_setting}\spxextra{TiePieLCR\_settings.TiePieLCR\_settings attribute}}

\begin{fulllineitems}
\phantomsection\label{\detokenize{index:TiePieLCR_settings.TiePieLCR_settings.gain_setting}}\pysigline{\sphinxbfcode{\sphinxupquote{gain\_setting}}\sphinxbfcode{\sphinxupquote{\DUrole{w}{  }\DUrole{p}{=}\DUrole{w}{  }0}}}
\end{fulllineitems}

\index{gain\_update\_required (TiePieLCR\_settings.TiePieLCR\_settings attribute)@\spxentry{gain\_update\_required}\spxextra{TiePieLCR\_settings.TiePieLCR\_settings attribute}}

\begin{fulllineitems}
\phantomsection\label{\detokenize{index:TiePieLCR_settings.TiePieLCR_settings.gain_update_required}}\pysigline{\sphinxbfcode{\sphinxupquote{gain\_update\_required}}\sphinxbfcode{\sphinxupquote{\DUrole{w}{  }\DUrole{p}{=}\DUrole{w}{  }False}}}
\end{fulllineitems}

\index{gcd() (TiePieLCR\_settings.TiePieLCR\_settings static method)@\spxentry{gcd()}\spxextra{TiePieLCR\_settings.TiePieLCR\_settings static method}}

\begin{fulllineitems}
\phantomsection\label{\detokenize{index:TiePieLCR_settings.TiePieLCR_settings.gcd}}\pysiglinewithargsret{\sphinxbfcode{\sphinxupquote{static\DUrole{w}{  }}}\sphinxbfcode{\sphinxupquote{gcd}}}{\emph{\DUrole{n}{L}}}{}
\end{fulllineitems}

\index{gen\_amplitude (TiePieLCR\_settings.TiePieLCR\_settings attribute)@\spxentry{gen\_amplitude}\spxextra{TiePieLCR\_settings.TiePieLCR\_settings attribute}}

\begin{fulllineitems}
\phantomsection\label{\detokenize{index:TiePieLCR_settings.TiePieLCR_settings.gen_amplitude}}\pysigline{\sphinxbfcode{\sphinxupquote{gen\_amplitude}}\sphinxbfcode{\sphinxupquote{\DUrole{w}{  }\DUrole{p}{=}\DUrole{w}{  }0.7}}}
\end{fulllineitems}

\index{gen\_amplitude\_update\_required (TiePieLCR\_settings.TiePieLCR\_settings attribute)@\spxentry{gen\_amplitude\_update\_required}\spxextra{TiePieLCR\_settings.TiePieLCR\_settings attribute}}

\begin{fulllineitems}
\phantomsection\label{\detokenize{index:TiePieLCR_settings.TiePieLCR_settings.gen_amplitude_update_required}}\pysigline{\sphinxbfcode{\sphinxupquote{gen\_amplitude\_update\_required}}\sphinxbfcode{\sphinxupquote{\DUrole{w}{  }\DUrole{p}{=}\DUrole{w}{  }False}}}
\end{fulllineitems}

\index{gen\_freqs (TiePieLCR\_settings.TiePieLCR\_settings attribute)@\spxentry{gen\_freqs}\spxextra{TiePieLCR\_settings.TiePieLCR\_settings attribute}}

\begin{fulllineitems}
\phantomsection\label{\detokenize{index:TiePieLCR_settings.TiePieLCR_settings.gen_freqs}}\pysigline{\sphinxbfcode{\sphinxupquote{gen\_freqs}}\sphinxbfcode{\sphinxupquote{\DUrole{w}{  }\DUrole{p}{=}\DUrole{w}{  }{[}5000{]}}}}
\end{fulllineitems}

\index{gen\_offset (TiePieLCR\_settings.TiePieLCR\_settings attribute)@\spxentry{gen\_offset}\spxextra{TiePieLCR\_settings.TiePieLCR\_settings attribute}}

\begin{fulllineitems}
\phantomsection\label{\detokenize{index:TiePieLCR_settings.TiePieLCR_settings.gen_offset}}\pysigline{\sphinxbfcode{\sphinxupquote{gen\_offset}}\sphinxbfcode{\sphinxupquote{\DUrole{w}{  }\DUrole{p}{=}\DUrole{w}{  }0}}}
\end{fulllineitems}

\index{gen\_offset\_update\_required (TiePieLCR\_settings.TiePieLCR\_settings attribute)@\spxentry{gen\_offset\_update\_required}\spxextra{TiePieLCR\_settings.TiePieLCR\_settings attribute}}

\begin{fulllineitems}
\phantomsection\label{\detokenize{index:TiePieLCR_settings.TiePieLCR_settings.gen_offset_update_required}}\pysigline{\sphinxbfcode{\sphinxupquote{gen\_offset\_update\_required}}\sphinxbfcode{\sphinxupquote{\DUrole{w}{  }\DUrole{p}{=}\DUrole{w}{  }False}}}
\end{fulllineitems}

\index{gen\_phases (TiePieLCR\_settings.TiePieLCR\_settings attribute)@\spxentry{gen\_phases}\spxextra{TiePieLCR\_settings.TiePieLCR\_settings attribute}}

\begin{fulllineitems}
\phantomsection\label{\detokenize{index:TiePieLCR_settings.TiePieLCR_settings.gen_phases}}\pysigline{\sphinxbfcode{\sphinxupquote{gen\_phases}}\sphinxbfcode{\sphinxupquote{\DUrole{w}{  }\DUrole{p}{=}\DUrole{w}{  }{[}0{]}}}}
\end{fulllineitems}

\index{gen\_restart\_required (TiePieLCR\_settings.TiePieLCR\_settings attribute)@\spxentry{gen\_restart\_required}\spxextra{TiePieLCR\_settings.TiePieLCR\_settings attribute}}

\begin{fulllineitems}
\phantomsection\label{\detokenize{index:TiePieLCR_settings.TiePieLCR_settings.gen_restart_required}}\pysigline{\sphinxbfcode{\sphinxupquote{gen\_restart\_required}}\sphinxbfcode{\sphinxupquote{\DUrole{w}{  }\DUrole{p}{=}\DUrole{w}{  }False}}}
\end{fulllineitems}

\index{gen\_sample\_freq (TiePieLCR\_settings.TiePieLCR\_settings attribute)@\spxentry{gen\_sample\_freq}\spxextra{TiePieLCR\_settings.TiePieLCR\_settings attribute}}

\begin{fulllineitems}
\phantomsection\label{\detokenize{index:TiePieLCR_settings.TiePieLCR_settings.gen_sample_freq}}\pysigline{\sphinxbfcode{\sphinxupquote{gen\_sample\_freq}}\sphinxbfcode{\sphinxupquote{\DUrole{w}{  }\DUrole{p}{=}\DUrole{w}{  }700000.0}}}
\end{fulllineitems}

\index{gen\_sample\_freq\_max (TiePieLCR\_settings.TiePieLCR\_settings attribute)@\spxentry{gen\_sample\_freq\_max}\spxextra{TiePieLCR\_settings.TiePieLCR\_settings attribute}}

\begin{fulllineitems}
\phantomsection\label{\detokenize{index:TiePieLCR_settings.TiePieLCR_settings.gen_sample_freq_max}}\pysigline{\sphinxbfcode{\sphinxupquote{gen\_sample\_freq\_max}}\sphinxbfcode{\sphinxupquote{\DUrole{w}{  }\DUrole{p}{=}\DUrole{w}{  }240000000.0}}}
\end{fulllineitems}

\index{gen\_samples\_max (TiePieLCR\_settings.TiePieLCR\_settings attribute)@\spxentry{gen\_samples\_max}\spxextra{TiePieLCR\_settings.TiePieLCR\_settings attribute}}

\begin{fulllineitems}
\phantomsection\label{\detokenize{index:TiePieLCR_settings.TiePieLCR_settings.gen_samples_max}}\pysigline{\sphinxbfcode{\sphinxupquote{gen\_samples\_max}}\sphinxbfcode{\sphinxupquote{\DUrole{w}{  }\DUrole{p}{=}\DUrole{w}{  }1000000.0}}}
\end{fulllineitems}

\index{gen\_samples\_min (TiePieLCR\_settings.TiePieLCR\_settings attribute)@\spxentry{gen\_samples\_min}\spxextra{TiePieLCR\_settings.TiePieLCR\_settings attribute}}

\begin{fulllineitems}
\phantomsection\label{\detokenize{index:TiePieLCR_settings.TiePieLCR_settings.gen_samples_min}}\pysigline{\sphinxbfcode{\sphinxupquote{gen\_samples\_min}}\sphinxbfcode{\sphinxupquote{\DUrole{w}{  }\DUrole{p}{=}\DUrole{w}{  }500000.0}}}
\end{fulllineitems}

\index{gen\_weights (TiePieLCR\_settings.TiePieLCR\_settings attribute)@\spxentry{gen\_weights}\spxextra{TiePieLCR\_settings.TiePieLCR\_settings attribute}}

\begin{fulllineitems}
\phantomsection\label{\detokenize{index:TiePieLCR_settings.TiePieLCR_settings.gen_weights}}\pysigline{\sphinxbfcode{\sphinxupquote{gen\_weights}}\sphinxbfcode{\sphinxupquote{\DUrole{w}{  }\DUrole{p}{=}\DUrole{w}{  }{[}1.0{]}}}}
\end{fulllineitems}

\index{get\_block\_size() (TiePieLCR\_settings.TiePieLCR\_settings method)@\spxentry{get\_block\_size()}\spxextra{TiePieLCR\_settings.TiePieLCR\_settings method}}

\begin{fulllineitems}
\phantomsection\label{\detokenize{index:TiePieLCR_settings.TiePieLCR_settings.get_block_size}}\pysiglinewithargsret{\sphinxbfcode{\sphinxupquote{get\_block\_size}}}{}{}
\sphinxAtStartPar
Get number of samples in each block of data that will be retrieved from the scope
\begin{quote}\begin{description}
\item[{Returns}] \leavevmode
\sphinxAtStartPar
The number of samples in one block

\item[{Return type}] \leavevmode
\sphinxAtStartPar
Int

\end{description}\end{quote}

\end{fulllineitems}

\index{get\_demodulate\_plot\_points() (TiePieLCR\_settings.TiePieLCR\_settings method)@\spxentry{get\_demodulate\_plot\_points()}\spxextra{TiePieLCR\_settings.TiePieLCR\_settings method}}

\begin{fulllineitems}
\phantomsection\label{\detokenize{index:TiePieLCR_settings.TiePieLCR_settings.get_demodulate_plot_points}}\pysiglinewithargsret{\sphinxbfcode{\sphinxupquote{get\_demodulate\_plot\_points}}}{}{}
\sphinxAtStartPar
Get the number of points plotted in the bottom two plots. Setting this to a to high value will make things slow, settings this to a too high value will cause aliasing.
\begin{quote}\begin{description}
\item[{Returns}] \leavevmode
\sphinxAtStartPar
The number of points

\item[{Return type}] \leavevmode
\sphinxAtStartPar
Float

\end{description}\end{quote}

\end{fulllineitems}

\index{get\_demodulation\_bandwidth() (TiePieLCR\_settings.TiePieLCR\_settings method)@\spxentry{get\_demodulation\_bandwidth()}\spxextra{TiePieLCR\_settings.TiePieLCR\_settings method}}

\begin{fulllineitems}
\phantomsection\label{\detokenize{index:TiePieLCR_settings.TiePieLCR_settings.get_demodulation_bandwidth}}\pysiglinewithargsret{\sphinxbfcode{\sphinxupquote{get\_demodulation\_bandwidth}}}{}{}
\end{fulllineitems}

\index{get\_demodulation\_freqs() (TiePieLCR\_settings.TiePieLCR\_settings method)@\spxentry{get\_demodulation\_freqs()}\spxextra{TiePieLCR\_settings.TiePieLCR\_settings method}}

\begin{fulllineitems}
\phantomsection\label{\detokenize{index:TiePieLCR_settings.TiePieLCR_settings.get_demodulation_freqs}}\pysiglinewithargsret{\sphinxbfcode{\sphinxupquote{get\_demodulation\_freqs}}}{}{}
\sphinxAtStartPar
Get a list with the frequencies at which the measured singal is demodulated.
\begin{quote}\begin{description}
\item[{Returns}] \leavevmode
\sphinxAtStartPar
The frequencies

\item[{Return type}] \leavevmode
\sphinxAtStartPar
list of floats

\end{description}\end{quote}

\end{fulllineitems}

\index{get\_demodulation\_tiepies() (TiePieLCR\_settings.TiePieLCR\_settings method)@\spxentry{get\_demodulation\_tiepies()}\spxextra{TiePieLCR\_settings.TiePieLCR\_settings method}}

\begin{fulllineitems}
\phantomsection\label{\detokenize{index:TiePieLCR_settings.TiePieLCR_settings.get_demodulation_tiepies}}\pysiglinewithargsret{\sphinxbfcode{\sphinxupquote{get\_demodulation\_tiepies}}}{}{}
\sphinxAtStartPar
Get a list with tiepie the reference signal of a specific frequency can be found.
\begin{quote}\begin{description}
\item[{Returns}] \leavevmode
\sphinxAtStartPar
The tiepie

\item[{Return type}] \leavevmode
\sphinxAtStartPar
list of ints

\end{description}\end{quote}

\end{fulllineitems}

\index{get\_demodulation\_time\_vector() (TiePieLCR\_settings.TiePieLCR\_settings method)@\spxentry{get\_demodulation\_time\_vector()}\spxextra{TiePieLCR\_settings.TiePieLCR\_settings method}}

\begin{fulllineitems}
\phantomsection\label{\detokenize{index:TiePieLCR_settings.TiePieLCR_settings.get_demodulation_time_vector}}\pysiglinewithargsret{\sphinxbfcode{\sphinxupquote{get\_demodulation\_time\_vector}}}{\emph{\DUrole{n}{blocks}}}{}
\sphinxAtStartPar
Calculate the time vector for a certain number of block of demodulation data.
\begin{quote}\begin{description}
\item[{Parameters}] \leavevmode
\sphinxAtStartPar
\sphinxstyleliteralstrong{\sphinxupquote{blocks}} (\sphinxstyleliteralemphasis{\sphinxupquote{Int}}) \textendash{} The number of blocks of scope data for which a time vector should be calcualted

\item[{Returns}] \leavevmode
\sphinxAtStartPar
The calculated time vector

\item[{Return type}] \leavevmode
\sphinxAtStartPar
Numpy vector

\end{description}\end{quote}

\end{fulllineitems}

\index{get\_df() (TiePieLCR\_settings.TiePieLCR\_settings method)@\spxentry{get\_df()}\spxextra{TiePieLCR\_settings.TiePieLCR\_settings method}}

\begin{fulllineitems}
\phantomsection\label{\detokenize{index:TiePieLCR_settings.TiePieLCR_settings.get_df}}\pysiglinewithargsret{\sphinxbfcode{\sphinxupquote{get\_df}}}{}{}
\sphinxAtStartPar
Get number of frequencies in one bin of the fft
\begin{quote}\begin{description}
\item[{Returns}] \leavevmode
\sphinxAtStartPar
The number of frequencies in one bin

\item[{Return type}] \leavevmode
\sphinxAtStartPar
Float

\end{description}\end{quote}

\end{fulllineitems}

\index{get\_f\_fun() (TiePieLCR\_settings.TiePieLCR\_settings method)@\spxentry{get\_f\_fun()}\spxextra{TiePieLCR\_settings.TiePieLCR\_settings method}}

\begin{fulllineitems}
\phantomsection\label{\detokenize{index:TiePieLCR_settings.TiePieLCR_settings.get_f_fun}}\pysiglinewithargsret{\sphinxbfcode{\sphinxupquote{get\_f\_fun}}}{}{}
\sphinxAtStartPar
Calculate frequency with which the multisine will repeat itself.
\begin{quote}\begin{description}
\item[{Returns}] \leavevmode
\sphinxAtStartPar
The fundamental frequency

\item[{Return type}] \leavevmode
\sphinxAtStartPar
float

\end{description}\end{quote}

\end{fulllineitems}

\index{get\_f\_max() (TiePieLCR\_settings.TiePieLCR\_settings method)@\spxentry{get\_f\_max()}\spxextra{TiePieLCR\_settings.TiePieLCR\_settings method}}

\begin{fulllineitems}
\phantomsection\label{\detokenize{index:TiePieLCR_settings.TiePieLCR_settings.get_f_max}}\pysiglinewithargsret{\sphinxbfcode{\sphinxupquote{get\_f\_max}}}{}{}
\sphinxAtStartPar
The maximum frequency in the multisine
\begin{quote}\begin{description}
\item[{Returns}] \leavevmode
\sphinxAtStartPar
The maximum frequency

\item[{Return type}] \leavevmode
\sphinxAtStartPar
float

\end{description}\end{quote}

\end{fulllineitems}

\index{get\_f\_min() (TiePieLCR\_settings.TiePieLCR\_settings method)@\spxentry{get\_f\_min()}\spxextra{TiePieLCR\_settings.TiePieLCR\_settings method}}

\begin{fulllineitems}
\phantomsection\label{\detokenize{index:TiePieLCR_settings.TiePieLCR_settings.get_f_min}}\pysiglinewithargsret{\sphinxbfcode{\sphinxupquote{get\_f\_min}}}{}{}
\sphinxAtStartPar
The minimum frequency in the multisine
\begin{quote}\begin{description}
\item[{Returns}] \leavevmode
\sphinxAtStartPar
The minimum frequency

\item[{Return type}] \leavevmode
\sphinxAtStartPar
float

\end{description}\end{quote}

\end{fulllineitems}

\index{get\_fft\_sensitivity() (TiePieLCR\_settings.TiePieLCR\_settings method)@\spxentry{get\_fft\_sensitivity()}\spxextra{TiePieLCR\_settings.TiePieLCR\_settings method}}

\begin{fulllineitems}
\phantomsection\label{\detokenize{index:TiePieLCR_settings.TiePieLCR_settings.get_fft_sensitivity}}\pysiglinewithargsret{\sphinxbfcode{\sphinxupquote{get\_fft\_sensitivity}}}{}{}
\sphinxAtStartPar
Calculate the minimum width of a peak in the fft
\begin{quote}\begin{description}
\item[{Returns}] \leavevmode
\sphinxAtStartPar
The minimum widht in Hertz

\item[{Return type}] \leavevmode
\sphinxAtStartPar
Float

\end{description}\end{quote}

\end{fulllineitems}

\index{get\_final\_offset\_block\_size() (TiePieLCR\_settings.TiePieLCR\_settings method)@\spxentry{get\_final\_offset\_block\_size()}\spxextra{TiePieLCR\_settings.TiePieLCR\_settings method}}

\begin{fulllineitems}
\phantomsection\label{\detokenize{index:TiePieLCR_settings.TiePieLCR_settings.get_final_offset_block_size}}\pysiglinewithargsret{\sphinxbfcode{\sphinxupquote{get\_final\_offset\_block\_size}}}{}{}
\sphinxAtStartPar
Get the number of offset samples that will result from each block after downsampling.
\begin{quote}\begin{description}
\item[{Returns}] \leavevmode
\sphinxAtStartPar
The number of samples

\item[{Return type}] \leavevmode
\sphinxAtStartPar
Int

\end{description}\end{quote}

\end{fulllineitems}

\index{get\_final\_output\_block\_size() (TiePieLCR\_settings.TiePieLCR\_settings method)@\spxentry{get\_final\_output\_block\_size()}\spxextra{TiePieLCR\_settings.TiePieLCR\_settings method}}

\begin{fulllineitems}
\phantomsection\label{\detokenize{index:TiePieLCR_settings.TiePieLCR_settings.get_final_output_block_size}}\pysiglinewithargsret{\sphinxbfcode{\sphinxupquote{get\_final\_output\_block\_size}}}{}{}
\sphinxAtStartPar
Get the number of impedance samples that will result from each block after downsampling.
\begin{quote}\begin{description}
\item[{Returns}] \leavevmode
\sphinxAtStartPar
The number of samples

\item[{Return type}] \leavevmode
\sphinxAtStartPar
Int

\end{description}\end{quote}

\end{fulllineitems}

\index{get\_gain\_name\_list() (TiePieLCR\_settings.TiePieLCR\_settings method)@\spxentry{get\_gain\_name\_list()}\spxextra{TiePieLCR\_settings.TiePieLCR\_settings method}}

\begin{fulllineitems}
\phantomsection\label{\detokenize{index:TiePieLCR_settings.TiePieLCR_settings.get_gain_name_list}}\pysiglinewithargsret{\sphinxbfcode{\sphinxupquote{get\_gain\_name\_list}}}{}{}
\sphinxAtStartPar
Get a list of possible gains for the instrumentation amplifier.
\begin{quote}\begin{description}
\item[{Returns}] \leavevmode
\sphinxAtStartPar
List of gains

\item[{Return type}] \leavevmode
\sphinxAtStartPar
List of floats

\end{description}\end{quote}

\end{fulllineitems}

\index{get\_gain\_setting() (TiePieLCR\_settings.TiePieLCR\_settings method)@\spxentry{get\_gain\_setting()}\spxextra{TiePieLCR\_settings.TiePieLCR\_settings method}}

\begin{fulllineitems}
\phantomsection\label{\detokenize{index:TiePieLCR_settings.TiePieLCR_settings.get_gain_setting}}\pysiglinewithargsret{\sphinxbfcode{\sphinxupquote{get\_gain\_setting}}}{}{}
\sphinxAtStartPar
Get the index of the currently selected gain setting. The list of possible gain can be obtained using {\hyperref[\detokenize{index:TiePieLCR_settings.TiePieLCR_settings.get_gain_name_list}]{\sphinxcrossref{\sphinxcode{\sphinxupquote{TiePieLCR\_settings.TiePieLCR\_settings.get\_gain\_name\_list()}}}}}
\begin{quote}\begin{description}
\item[{Returns}] \leavevmode
\sphinxAtStartPar
The index

\item[{Return type}] \leavevmode
\sphinxAtStartPar
Int

\end{description}\end{quote}

\end{fulllineitems}

\index{get\_gain\_value() (TiePieLCR\_settings.TiePieLCR\_settings method)@\spxentry{get\_gain\_value()}\spxextra{TiePieLCR\_settings.TiePieLCR\_settings method}}

\begin{fulllineitems}
\phantomsection\label{\detokenize{index:TiePieLCR_settings.TiePieLCR_settings.get_gain_value}}\pysiglinewithargsret{\sphinxbfcode{\sphinxupquote{get\_gain\_value}}}{}{}
\sphinxAtStartPar
Get the current gain of the instrumentation amplifier
\begin{quote}\begin{description}
\item[{Returns}] \leavevmode
\sphinxAtStartPar
The gain

\item[{Return type}] \leavevmode
\sphinxAtStartPar
float

\end{description}\end{quote}

\end{fulllineitems}

\index{get\_gen\_amplitude() (TiePieLCR\_settings.TiePieLCR\_settings method)@\spxentry{get\_gen\_amplitude()}\spxextra{TiePieLCR\_settings.TiePieLCR\_settings method}}

\begin{fulllineitems}
\phantomsection\label{\detokenize{index:TiePieLCR_settings.TiePieLCR_settings.get_gen_amplitude}}\pysiglinewithargsret{\sphinxbfcode{\sphinxupquote{get\_gen\_amplitude}}}{}{}
\sphinxAtStartPar
Get the maximum amplitude of the multi\sphinxhyphen{}sine used for the excitation. The excitation signal will determine the voltage on HcurV and when multiplied with 370uA/V also the current through HcurI. The inverse of the excitation signal will determine the voltage on nHcurV and when multiplied with 370uA/V also the current through nHcurI. This is equivalent to the excitation amplitude in the interface.
\begin{quote}\begin{description}
\item[{Returns}] \leavevmode
\sphinxAtStartPar
The amplitude

\item[{Return type}] \leavevmode
\sphinxAtStartPar
float

\end{description}\end{quote}

\end{fulllineitems}

\index{get\_gen\_offset() (TiePieLCR\_settings.TiePieLCR\_settings method)@\spxentry{get\_gen\_offset()}\spxextra{TiePieLCR\_settings.TiePieLCR\_settings method}}

\begin{fulllineitems}
\phantomsection\label{\detokenize{index:TiePieLCR_settings.TiePieLCR_settings.get_gen_offset}}\pysiglinewithargsret{\sphinxbfcode{\sphinxupquote{get\_gen\_offset}}}{}{}
\sphinxAtStartPar
Get the offset of the excitation. The excitation signal will determine the voltage on HcurV and when multiplied with 370uA/V also the current through HcurI. The inverse of the excitation signal will determine the voltage on nHcurV and when multiplied with 370uA/V also the current through nHcurI. This is equivalent to the excitation amplitude in the interface.
\begin{quote}\begin{description}
\item[{Returns}] \leavevmode
\sphinxAtStartPar
The offset

\item[{Return type}] \leavevmode
\sphinxAtStartPar
float

\end{description}\end{quote}

\end{fulllineitems}

\index{get\_gen\_sample\_freq() (TiePieLCR\_settings.TiePieLCR\_settings method)@\spxentry{get\_gen\_sample\_freq()}\spxextra{TiePieLCR\_settings.TiePieLCR\_settings method}}

\begin{fulllineitems}
\phantomsection\label{\detokenize{index:TiePieLCR_settings.TiePieLCR_settings.get_gen_sample_freq}}\pysiglinewithargsret{\sphinxbfcode{\sphinxupquote{get\_gen\_sample\_freq}}}{}{}
\sphinxAtStartPar
Calculate the sample frequency that should be used for the multisine
\begin{quote}\begin{description}
\item[{Returns}] \leavevmode
\sphinxAtStartPar
The sample frequency

\item[{Return type}] \leavevmode
\sphinxAtStartPar
Float

\end{description}\end{quote}

\end{fulllineitems}

\index{get\_gen\_samples() (TiePieLCR\_settings.TiePieLCR\_settings method)@\spxentry{get\_gen\_samples()}\spxextra{TiePieLCR\_settings.TiePieLCR\_settings method}}

\begin{fulllineitems}
\phantomsection\label{\detokenize{index:TiePieLCR_settings.TiePieLCR_settings.get_gen_samples}}\pysiglinewithargsret{\sphinxbfcode{\sphinxupquote{get\_gen\_samples}}}{}{}
\sphinxAtStartPar
Calculate the number of samples that should be used for the multisine
\begin{quote}\begin{description}
\item[{Returns}] \leavevmode
\sphinxAtStartPar
The number of samples

\item[{Return type}] \leavevmode
\sphinxAtStartPar
Int

\end{description}\end{quote}

\end{fulllineitems}

\index{get\_impedance\_format() (TiePieLCR\_settings.TiePieLCR\_settings method)@\spxentry{get\_impedance\_format()}\spxextra{TiePieLCR\_settings.TiePieLCR\_settings method}}

\begin{fulllineitems}
\phantomsection\label{\detokenize{index:TiePieLCR_settings.TiePieLCR_settings.get_impedance_format}}\pysiglinewithargsret{\sphinxbfcode{\sphinxupquote{get\_impedance\_format}}}{}{}
\sphinxAtStartPar
Get the complex impedance measured by the LCR can be represented in different ways:
\begin{itemize}
\item {} 
\sphinxAtStartPar
XY: As a complex and an imaginary part

\item {} 
\sphinxAtStartPar
RpCp: As a capacitor and a resistor in parallel

\item {} 
\sphinxAtStartPar
RsCs: As a capacitor and a reisistor in series

\item {} 
\sphinxAtStartPar
ZPhi: As an absolute value and a phase

\end{itemize}
\begin{quote}\begin{description}
\item[{Returns}] \leavevmode
\sphinxAtStartPar
The used format as a string i.e. ‘XY’ or ‘RpCp’

\item[{Return type}] \leavevmode
\sphinxAtStartPar
String

\end{description}\end{quote}

\end{fulllineitems}

\index{get\_impedance\_format\_label1() (TiePieLCR\_settings.TiePieLCR\_settings method)@\spxentry{get\_impedance\_format\_label1()}\spxextra{TiePieLCR\_settings.TiePieLCR\_settings method}}

\begin{fulllineitems}
\phantomsection\label{\detokenize{index:TiePieLCR_settings.TiePieLCR_settings.get_impedance_format_label1}}\pysiglinewithargsret{\sphinxbfcode{\sphinxupquote{get\_impedance\_format\_label1}}}{}{}
\sphinxAtStartPar
Get the label of the first value of the impedance format
\begin{quote}\begin{description}
\item[{Returns}] \leavevmode
\sphinxAtStartPar
The used format as a string i.e. ‘F’

\item[{Return type}] \leavevmode
\sphinxAtStartPar
String

\end{description}\end{quote}

\end{fulllineitems}

\index{get\_impedance\_format\_label2() (TiePieLCR\_settings.TiePieLCR\_settings method)@\spxentry{get\_impedance\_format\_label2()}\spxextra{TiePieLCR\_settings.TiePieLCR\_settings method}}

\begin{fulllineitems}
\phantomsection\label{\detokenize{index:TiePieLCR_settings.TiePieLCR_settings.get_impedance_format_label2}}\pysiglinewithargsret{\sphinxbfcode{\sphinxupquote{get\_impedance\_format\_label2}}}{}{}
\sphinxAtStartPar
Get the label of the second value of the impedance format
\begin{quote}\begin{description}
\item[{Returns}] \leavevmode
\sphinxAtStartPar
The used format as a string i.e. ‘F’

\item[{Return type}] \leavevmode
\sphinxAtStartPar
String

\end{description}\end{quote}

\end{fulllineitems}

\index{get\_impedance\_format\_unit1() (TiePieLCR\_settings.TiePieLCR\_settings method)@\spxentry{get\_impedance\_format\_unit1()}\spxextra{TiePieLCR\_settings.TiePieLCR\_settings method}}

\begin{fulllineitems}
\phantomsection\label{\detokenize{index:TiePieLCR_settings.TiePieLCR_settings.get_impedance_format_unit1}}\pysiglinewithargsret{\sphinxbfcode{\sphinxupquote{get\_impedance\_format\_unit1}}}{}{}
\sphinxAtStartPar
Get the unit of the first value of the impedance format
\begin{quote}\begin{description}
\item[{Returns}] \leavevmode
\sphinxAtStartPar
The used format as a string i.e. ‘F’

\item[{Return type}] \leavevmode
\sphinxAtStartPar
String

\end{description}\end{quote}

\end{fulllineitems}

\index{get\_impedance\_format\_unit2() (TiePieLCR\_settings.TiePieLCR\_settings method)@\spxentry{get\_impedance\_format\_unit2()}\spxextra{TiePieLCR\_settings.TiePieLCR\_settings method}}

\begin{fulllineitems}
\phantomsection\label{\detokenize{index:TiePieLCR_settings.TiePieLCR_settings.get_impedance_format_unit2}}\pysiglinewithargsret{\sphinxbfcode{\sphinxupquote{get\_impedance\_format\_unit2}}}{}{}
\sphinxAtStartPar
Get the unit of the second value of the impedance format
\begin{quote}\begin{description}
\item[{Returns}] \leavevmode
\sphinxAtStartPar
The used format as a string i.e.’F’

\item[{Return type}] \leavevmode
\sphinxAtStartPar
String

\end{description}\end{quote}

\end{fulllineitems}

\index{get\_integration\_time() (TiePieLCR\_settings.TiePieLCR\_settings method)@\spxentry{get\_integration\_time()}\spxextra{TiePieLCR\_settings.TiePieLCR\_settings method}}

\begin{fulllineitems}
\phantomsection\label{\detokenize{index:TiePieLCR_settings.TiePieLCR_settings.get_integration_time}}\pysiglinewithargsret{\sphinxbfcode{\sphinxupquote{get\_integration\_time}}}{}{}
\sphinxAtStartPar
Get the time over which the demodulation signals will be avaraged to compute the offset displayed in the interface and the offset obtain using \sphinxcode{\sphinxupquote{TiePieLCR\_api.TiePieLCR\_api.get\_impedance()}}.
\begin{quote}\begin{description}
\item[{Returns}] \leavevmode
\sphinxAtStartPar
The amount of time

\item[{Return type}] \leavevmode
\sphinxAtStartPar
Float

\end{description}\end{quote}

\end{fulllineitems}

\index{get\_maximum\_plot\_frequency() (TiePieLCR\_settings.TiePieLCR\_settings method)@\spxentry{get\_maximum\_plot\_frequency()}\spxextra{TiePieLCR\_settings.TiePieLCR\_settings method}}

\begin{fulllineitems}
\phantomsection\label{\detokenize{index:TiePieLCR_settings.TiePieLCR_settings.get_maximum_plot_frequency}}\pysiglinewithargsret{\sphinxbfcode{\sphinxupquote{get\_maximum\_plot\_frequency}}}{}{}
\sphinxAtStartPar
Get the maximum frequency that is shown in the frequency plots on the top right.
\begin{quote}\begin{description}
\item[{Returns}] \leavevmode
\sphinxAtStartPar
The frequency

\item[{Return type}] \leavevmode
\sphinxAtStartPar
float

\end{description}\end{quote}

\end{fulllineitems}

\index{get\_measurement\_z() (TiePieLCR\_settings.TiePieLCR\_settings method)@\spxentry{get\_measurement\_z()}\spxextra{TiePieLCR\_settings.TiePieLCR\_settings method}}

\begin{fulllineitems}
\phantomsection\label{\detokenize{index:TiePieLCR_settings.TiePieLCR_settings.get_measurement_z}}\pysiglinewithargsret{\sphinxbfcode{\sphinxupquote{get\_measurement\_z}}}{\emph{\DUrole{n}{freqs}}}{}
\sphinxAtStartPar
Calculate the impedance used in the feedback of the currently selected transimpedance amplifier for a specific frequency, taking into account stability caps, bias resistors and anti\sphinxhyphen{}aliassing filters.
\begin{quote}\begin{description}
\item[{Parameters}] \leavevmode
\sphinxAtStartPar
\sphinxstyleliteralstrong{\sphinxupquote{freqs}} \textendash{} The frequency at which the impedance should be calculated

\item[{Typ freqs}] \leavevmode
\sphinxAtStartPar
float

\item[{Returns}] \leavevmode
\sphinxAtStartPar
The impedance

\item[{Return type}] \leavevmode
\sphinxAtStartPar
Complex double

\end{description}\end{quote}

\end{fulllineitems}

\index{get\_minimum\_plot\_frequency() (TiePieLCR\_settings.TiePieLCR\_settings method)@\spxentry{get\_minimum\_plot\_frequency()}\spxextra{TiePieLCR\_settings.TiePieLCR\_settings method}}

\begin{fulllineitems}
\phantomsection\label{\detokenize{index:TiePieLCR_settings.TiePieLCR_settings.get_minimum_plot_frequency}}\pysiglinewithargsret{\sphinxbfcode{\sphinxupquote{get\_minimum\_plot\_frequency}}}{}{}
\sphinxAtStartPar
Get the minimum frequency that is shown in the frequency plots on the top right.
\begin{quote}\begin{description}
\item[{Returns}] \leavevmode
\sphinxAtStartPar
The frequency

\item[{Return type}] \leavevmode
\sphinxAtStartPar
float

\end{description}\end{quote}

\end{fulllineitems}

\index{get\_multisine\_crest\_factor() (TiePieLCR\_settings.TiePieLCR\_settings method)@\spxentry{get\_multisine\_crest\_factor()}\spxextra{TiePieLCR\_settings.TiePieLCR\_settings method}}

\begin{fulllineitems}
\phantomsection\label{\detokenize{index:TiePieLCR_settings.TiePieLCR_settings.get_multisine_crest_factor}}\pysiglinewithargsret{\sphinxbfcode{\sphinxupquote{get\_multisine\_crest\_factor}}}{}{}
\end{fulllineitems}

\index{get\_multisine\_freqs() (TiePieLCR\_settings.TiePieLCR\_settings method)@\spxentry{get\_multisine\_freqs()}\spxextra{TiePieLCR\_settings.TiePieLCR\_settings method}}

\begin{fulllineitems}
\phantomsection\label{\detokenize{index:TiePieLCR_settings.TiePieLCR_settings.get_multisine_freqs}}\pysiglinewithargsret{\sphinxbfcode{\sphinxupquote{get\_multisine\_freqs}}}{}{}
\sphinxAtStartPar
Get a list with the frequencies in the multisine that are used as an excitation signal.
\begin{quote}\begin{description}
\item[{Returns}] \leavevmode
\sphinxAtStartPar
The frequencies

\item[{Return type}] \leavevmode
\sphinxAtStartPar
list of floats

\end{description}\end{quote}

\end{fulllineitems}

\index{get\_multisine\_phases() (TiePieLCR\_settings.TiePieLCR\_settings method)@\spxentry{get\_multisine\_phases()}\spxextra{TiePieLCR\_settings.TiePieLCR\_settings method}}

\begin{fulllineitems}
\phantomsection\label{\detokenize{index:TiePieLCR_settings.TiePieLCR_settings.get_multisine_phases}}\pysiglinewithargsret{\sphinxbfcode{\sphinxupquote{get\_multisine\_phases}}}{}{}
\sphinxAtStartPar
Get a list with phases of each of the frequencies in the multisine. The weights are defined relative to the output of \sphinxcode{\sphinxupquote{TiePieLCR\_api.TiePieLCR\_api.get\_gen\_amplitude()}} .
\begin{quote}\begin{description}
\item[{Returns}] \leavevmode
\sphinxAtStartPar
The phases

\item[{Return type}] \leavevmode
\sphinxAtStartPar
list of floats

\end{description}\end{quote}

\end{fulllineitems}

\index{get\_multisine\_vector() (TiePieLCR\_settings.TiePieLCR\_settings method)@\spxentry{get\_multisine\_vector()}\spxextra{TiePieLCR\_settings.TiePieLCR\_settings method}}

\begin{fulllineitems}
\phantomsection\label{\detokenize{index:TiePieLCR_settings.TiePieLCR_settings.get_multisine_vector}}\pysiglinewithargsret{\sphinxbfcode{\sphinxupquote{get\_multisine\_vector}}}{}{}
\sphinxAtStartPar
Get the multisine signal that the AWG in the TiePieLCR will be set to.
\begin{quote}\begin{description}
\item[{Returns}] \leavevmode
\sphinxAtStartPar
The calculated multisine signal

\item[{Return type}] \leavevmode
\sphinxAtStartPar
Numpy vector

\end{description}\end{quote}

\end{fulllineitems}

\index{get\_multisine\_weights() (TiePieLCR\_settings.TiePieLCR\_settings method)@\spxentry{get\_multisine\_weights()}\spxextra{TiePieLCR\_settings.TiePieLCR\_settings method}}

\begin{fulllineitems}
\phantomsection\label{\detokenize{index:TiePieLCR_settings.TiePieLCR_settings.get_multisine_weights}}\pysiglinewithargsret{\sphinxbfcode{\sphinxupquote{get\_multisine\_weights}}}{}{}
\sphinxAtStartPar
Get a list with weights of each of the frequencies in the multisine. The weights are defined relative to the output of \sphinxcode{\sphinxupquote{TiePieLCR\_api.TiePieLCR\_api.get\_gen\_amplitude()}} .
\begin{quote}\begin{description}
\item[{Returns}] \leavevmode
\sphinxAtStartPar
The weights

\item[{Return type}] \leavevmode
\sphinxAtStartPar
list of floats

\end{description}\end{quote}

\end{fulllineitems}

\index{get\_number\_of\_demodulate\_freqs() (TiePieLCR\_settings.TiePieLCR\_settings method)@\spxentry{get\_number\_of\_demodulate\_freqs()}\spxextra{TiePieLCR\_settings.TiePieLCR\_settings method}}

\begin{fulllineitems}
\phantomsection\label{\detokenize{index:TiePieLCR_settings.TiePieLCR_settings.get_number_of_demodulate_freqs}}\pysiglinewithargsret{\sphinxbfcode{\sphinxupquote{get\_number\_of\_demodulate\_freqs}}}{}{}
\sphinxAtStartPar
Get the number of frequencies inside the multisine
\begin{quote}\begin{description}
\item[{Returns}] \leavevmode
\sphinxAtStartPar
The used format as a string i.e. ‘F’

\item[{Return type}] \leavevmode
\sphinxAtStartPar
Int

\end{description}\end{quote}

\end{fulllineitems}

\index{get\_number\_of\_demodulation\_freqs() (TiePieLCR\_settings.TiePieLCR\_settings method)@\spxentry{get\_number\_of\_demodulation\_freqs()}\spxextra{TiePieLCR\_settings.TiePieLCR\_settings method}}

\begin{fulllineitems}
\phantomsection\label{\detokenize{index:TiePieLCR_settings.TiePieLCR_settings.get_number_of_demodulation_freqs}}\pysiglinewithargsret{\sphinxbfcode{\sphinxupquote{get\_number\_of\_demodulation\_freqs}}}{}{}
\sphinxAtStartPar
The of frequencies in the multisine
\begin{quote}\begin{description}
\item[{Returns}] \leavevmode
\sphinxAtStartPar
The number of frequencies

\item[{Return type}] \leavevmode
\sphinxAtStartPar
Int

\end{description}\end{quote}

\end{fulllineitems}

\index{get\_number\_of\_multisine\_freqs() (TiePieLCR\_settings.TiePieLCR\_settings method)@\spxentry{get\_number\_of\_multisine\_freqs()}\spxextra{TiePieLCR\_settings.TiePieLCR\_settings method}}

\begin{fulllineitems}
\phantomsection\label{\detokenize{index:TiePieLCR_settings.TiePieLCR_settings.get_number_of_multisine_freqs}}\pysiglinewithargsret{\sphinxbfcode{\sphinxupquote{get\_number\_of\_multisine\_freqs}}}{}{}
\sphinxAtStartPar
The of frequencies in the multisine
\begin{quote}\begin{description}
\item[{Returns}] \leavevmode
\sphinxAtStartPar
The number of frequencies

\item[{Return type}] \leavevmode
\sphinxAtStartPar
Int

\end{description}\end{quote}

\end{fulllineitems}

\index{get\_offset\_bandwidth() (TiePieLCR\_settings.TiePieLCR\_settings method)@\spxentry{get\_offset\_bandwidth()}\spxextra{TiePieLCR\_settings.TiePieLCR\_settings method}}

\begin{fulllineitems}
\phantomsection\label{\detokenize{index:TiePieLCR_settings.TiePieLCR_settings.get_offset_bandwidth}}\pysiglinewithargsret{\sphinxbfcode{\sphinxupquote{get\_offset\_bandwidth}}}{}{}
\sphinxAtStartPar
Get the bandwidth of the offset signals. The signal computed using this bandwidth can be found in the stored mat file\textasciigrave{}.
\begin{quote}\begin{description}
\item[{Returns}] \leavevmode
\sphinxAtStartPar
The bandwidth

\item[{Return type}] \leavevmode
\sphinxAtStartPar
Float

\end{description}\end{quote}

\end{fulllineitems}

\index{get\_offset\_block\_size() (TiePieLCR\_settings.TiePieLCR\_settings method)@\spxentry{get\_offset\_block\_size()}\spxextra{TiePieLCR\_settings.TiePieLCR\_settings method}}

\begin{fulllineitems}
\phantomsection\label{\detokenize{index:TiePieLCR_settings.TiePieLCR_settings.get_offset_block_size}}\pysiglinewithargsret{\sphinxbfcode{\sphinxupquote{get\_offset\_block\_size}}}{}{}
\sphinxAtStartPar
Get number of offset samples that will be calculated during each block using the ffts, before the impedance signal is downsampled.
\begin{quote}\begin{description}
\item[{Returns}] \leavevmode
\sphinxAtStartPar
The number of offset samples per sub\sphinxhyphen{}block

\item[{Return type}] \leavevmode
\sphinxAtStartPar
Int

\end{description}\end{quote}

\end{fulllineitems}

\index{get\_offset\_downsampling\_rate() (TiePieLCR\_settings.TiePieLCR\_settings method)@\spxentry{get\_offset\_downsampling\_rate()}\spxextra{TiePieLCR\_settings.TiePieLCR\_settings method}}

\begin{fulllineitems}
\phantomsection\label{\detokenize{index:TiePieLCR_settings.TiePieLCR_settings.get_offset_downsampling_rate}}\pysiglinewithargsret{\sphinxbfcode{\sphinxupquote{get\_offset\_downsampling\_rate}}}{}{}
\sphinxAtStartPar
Get the factor by which the offset signal calculated using the ffts will be downsampled.
\begin{quote}\begin{description}
\item[{Returns}] \leavevmode
\sphinxAtStartPar
The downsampling ratio

\item[{Return type}] \leavevmode
\sphinxAtStartPar
Int

\end{description}\end{quote}

\end{fulllineitems}

\index{get\_offset\_integration\_time() (TiePieLCR\_settings.TiePieLCR\_settings method)@\spxentry{get\_offset\_integration\_time()}\spxextra{TiePieLCR\_settings.TiePieLCR\_settings method}}

\begin{fulllineitems}
\phantomsection\label{\detokenize{index:TiePieLCR_settings.TiePieLCR_settings.get_offset_integration_time}}\pysiglinewithargsret{\sphinxbfcode{\sphinxupquote{get\_offset\_integration\_time}}}{}{}
\sphinxAtStartPar
Get the time over which the offset signals will be avaraged to compute the offset displayed in the interface and the offset obtain using \sphinxcode{\sphinxupquote{TiePieLCR\_api.TiePieLCR\_api.get\_impedance()}}.
\begin{quote}\begin{description}
\item[{Returns}] \leavevmode
\sphinxAtStartPar
The amount of time

\item[{Return type}] \leavevmode
\sphinxAtStartPar
Int

\end{description}\end{quote}

\end{fulllineitems}

\index{get\_offset\_sample\_frequency() (TiePieLCR\_settings.TiePieLCR\_settings method)@\spxentry{get\_offset\_sample\_frequency()}\spxextra{TiePieLCR\_settings.TiePieLCR\_settings method}}

\begin{fulllineitems}
\phantomsection\label{\detokenize{index:TiePieLCR_settings.TiePieLCR_settings.get_offset_sample_frequency}}\pysiglinewithargsret{\sphinxbfcode{\sphinxupquote{get\_offset\_sample\_frequency}}}{}{}
\sphinxAtStartPar
Get the sample frequency of the offset samples that will be calculated using the ffts, before the offset signal is downsampled.
\begin{quote}\begin{description}
\item[{Returns}] \leavevmode
\sphinxAtStartPar
Number of offset samples per second

\item[{Return type}] \leavevmode
\sphinxAtStartPar
Float

\end{description}\end{quote}

\end{fulllineitems}

\index{get\_offset\_sub\_block\_size() (TiePieLCR\_settings.TiePieLCR\_settings method)@\spxentry{get\_offset\_sub\_block\_size()}\spxextra{TiePieLCR\_settings.TiePieLCR\_settings method}}

\begin{fulllineitems}
\phantomsection\label{\detokenize{index:TiePieLCR_settings.TiePieLCR_settings.get_offset_sub_block_size}}\pysiglinewithargsret{\sphinxbfcode{\sphinxupquote{get\_offset\_sub\_block\_size}}}{}{}
\sphinxAtStartPar
Get number of offset samples that will be calculated during each sub\sphinxhyphen{}block using the ffts, before the impedance signal is downsampled.
\begin{quote}\begin{description}
\item[{Returns}] \leavevmode
\sphinxAtStartPar
The number of offset samples per sub\sphinxhyphen{}block

\item[{Return type}] \leavevmode
\sphinxAtStartPar
Int

\end{description}\end{quote}

\end{fulllineitems}

\index{get\_offset\_time\_vector() (TiePieLCR\_settings.TiePieLCR\_settings method)@\spxentry{get\_offset\_time\_vector()}\spxextra{TiePieLCR\_settings.TiePieLCR\_settings method}}

\begin{fulllineitems}
\phantomsection\label{\detokenize{index:TiePieLCR_settings.TiePieLCR_settings.get_offset_time_vector}}\pysiglinewithargsret{\sphinxbfcode{\sphinxupquote{get\_offset\_time\_vector}}}{\emph{\DUrole{n}{blocks}}}{}
\sphinxAtStartPar
Calculate the time vector for a certain number of block of offset data.
\begin{quote}\begin{description}
\item[{Parameters}] \leavevmode
\sphinxAtStartPar
\sphinxstyleliteralstrong{\sphinxupquote{blocks}} (\sphinxstyleliteralemphasis{\sphinxupquote{Int}}) \textendash{} The number of blocks of scope data for which a time vector should be calcualted

\item[{Returns}] \leavevmode
\sphinxAtStartPar
The calculated time vector

\item[{Return type}] \leavevmode
\sphinxAtStartPar
Numpy vector

\end{description}\end{quote}

\end{fulllineitems}

\index{get\_output\_block\_size() (TiePieLCR\_settings.TiePieLCR\_settings method)@\spxentry{get\_output\_block\_size()}\spxextra{TiePieLCR\_settings.TiePieLCR\_settings method}}

\begin{fulllineitems}
\phantomsection\label{\detokenize{index:TiePieLCR_settings.TiePieLCR_settings.get_output_block_size}}\pysiglinewithargsret{\sphinxbfcode{\sphinxupquote{get\_output\_block\_size}}}{}{}
\sphinxAtStartPar
Get number of impedance samples that will be calculated during each block using the ffts, before the impedance signal is downsampled.

\sphinxAtStartPar
This currently is calculated incorrectly and will probably cause the gui to crash if the demodulation frequency is set too high
\begin{quote}\begin{description}
\item[{Returns}] \leavevmode
\sphinxAtStartPar
The number of impedance samples per block

\item[{Return type}] \leavevmode
\sphinxAtStartPar
Int

\end{description}\end{quote}

\end{fulllineitems}

\index{get\_output\_downsampling\_rate() (TiePieLCR\_settings.TiePieLCR\_settings method)@\spxentry{get\_output\_downsampling\_rate()}\spxextra{TiePieLCR\_settings.TiePieLCR\_settings method}}

\begin{fulllineitems}
\phantomsection\label{\detokenize{index:TiePieLCR_settings.TiePieLCR_settings.get_output_downsampling_rate}}\pysiglinewithargsret{\sphinxbfcode{\sphinxupquote{get\_output\_downsampling\_rate}}}{}{}
\sphinxAtStartPar
Get the factor by which the impedance signal calculated using the ffts will be downsampled.
\begin{quote}\begin{description}
\item[{Returns}] \leavevmode
\sphinxAtStartPar
The downsampling ratio

\item[{Return type}] \leavevmode
\sphinxAtStartPar
Int

\end{description}\end{quote}

\end{fulllineitems}

\index{get\_output\_oversample\_ratio() (TiePieLCR\_settings.TiePieLCR\_settings method)@\spxentry{get\_output\_oversample\_ratio()}\spxextra{TiePieLCR\_settings.TiePieLCR\_settings method}}

\begin{fulllineitems}
\phantomsection\label{\detokenize{index:TiePieLCR_settings.TiePieLCR_settings.get_output_oversample_ratio}}\pysiglinewithargsret{\sphinxbfcode{\sphinxupquote{get\_output\_oversample\_ratio}}}{}{}
\sphinxAtStartPar
Get the number of times the output signal will be oversamples. The sample frequency fill be oversample ratio times 2 times the bandwidth.
\begin{quote}\begin{description}
\item[{Returns}] \leavevmode
\sphinxAtStartPar
The oversampling ratio of the output

\item[{Return type}] \leavevmode
\sphinxAtStartPar
Int

\end{description}\end{quote}

\end{fulllineitems}

\index{get\_output\_sample\_freq() (TiePieLCR\_settings.TiePieLCR\_settings method)@\spxentry{get\_output\_sample\_freq()}\spxextra{TiePieLCR\_settings.TiePieLCR\_settings method}}

\begin{fulllineitems}
\phantomsection\label{\detokenize{index:TiePieLCR_settings.TiePieLCR_settings.get_output_sample_freq}}\pysiglinewithargsret{\sphinxbfcode{\sphinxupquote{get\_output\_sample\_freq}}}{}{}
\sphinxAtStartPar
Get the sample frequency of the impedance samples that will be calculated using the ffts, before the impedance signal is downsampled.
\begin{quote}\begin{description}
\item[{Returns}] \leavevmode
\sphinxAtStartPar
Number of impedance samples per second

\item[{Return type}] \leavevmode
\sphinxAtStartPar
Float

\end{description}\end{quote}

\end{fulllineitems}

\index{get\_output\_sub\_block\_size() (TiePieLCR\_settings.TiePieLCR\_settings method)@\spxentry{get\_output\_sub\_block\_size()}\spxextra{TiePieLCR\_settings.TiePieLCR\_settings method}}

\begin{fulllineitems}
\phantomsection\label{\detokenize{index:TiePieLCR_settings.TiePieLCR_settings.get_output_sub_block_size}}\pysiglinewithargsret{\sphinxbfcode{\sphinxupquote{get\_output\_sub\_block\_size}}}{}{}
\sphinxAtStartPar
Get number of samples of impedance samples that will be calculated during each sub\sphinxhyphen{}block using the ffts, before the impedance signal is downsampled.

\sphinxAtStartPar
This currently is calculated incorrectly and will probably cause the gui to crash if the demodulation frequency is set too high
\begin{quote}\begin{description}
\item[{Returns}] \leavevmode
\sphinxAtStartPar
The number of impedance samples per sub\sphinxhyphen{}block

\item[{Return type}] \leavevmode
\sphinxAtStartPar
Int

\end{description}\end{quote}

\end{fulllineitems}

\index{get\_plot\_blocks() (TiePieLCR\_settings.TiePieLCR\_settings method)@\spxentry{get\_plot\_blocks()}\spxextra{TiePieLCR\_settings.TiePieLCR\_settings method}}

\begin{fulllineitems}
\phantomsection\label{\detokenize{index:TiePieLCR_settings.TiePieLCR_settings.get_plot_blocks}}\pysiglinewithargsret{\sphinxbfcode{\sphinxupquote{get\_plot\_blocks}}}{}{}
\sphinxAtStartPar
Get how many blocks of data should be used to plot impedance data for a period equal to the plot time.
\begin{quote}\begin{description}
\item[{Returns}] \leavevmode
\sphinxAtStartPar
Number of blocks

\item[{Return type}] \leavevmode
\sphinxAtStartPar
Int

\end{description}\end{quote}

\end{fulllineitems}

\index{get\_plot\_periods() (TiePieLCR\_settings.TiePieLCR\_settings method)@\spxentry{get\_plot\_periods()}\spxextra{TiePieLCR\_settings.TiePieLCR\_settings method}}

\begin{fulllineitems}
\phantomsection\label{\detokenize{index:TiePieLCR_settings.TiePieLCR_settings.get_plot_periods}}\pysiglinewithargsret{\sphinxbfcode{\sphinxupquote{get\_plot\_periods}}}{}{}
\sphinxAtStartPar
Get the number of repetitions that are shown in the time plot in the left top of the interface.
\begin{quote}\begin{description}
\item[{Returns}] \leavevmode
\sphinxAtStartPar
The number of periods

\item[{Return type}] \leavevmode
\sphinxAtStartPar
float

\end{description}\end{quote}

\end{fulllineitems}

\index{get\_plot\_time() (TiePieLCR\_settings.TiePieLCR\_settings method)@\spxentry{get\_plot\_time()}\spxextra{TiePieLCR\_settings.TiePieLCR\_settings method}}

\begin{fulllineitems}
\phantomsection\label{\detokenize{index:TiePieLCR_settings.TiePieLCR_settings.get_plot_time}}\pysiglinewithargsret{\sphinxbfcode{\sphinxupquote{get\_plot\_time}}}{}{}
\sphinxAtStartPar
Get the amount of time shown on the x\sphinxhyphen{}axis of the bottom two graphs.
\begin{quote}\begin{description}
\item[{Returns}] \leavevmode
\sphinxAtStartPar
The amount of time

\item[{Return type}] \leavevmode
\sphinxAtStartPar
Float

\end{description}\end{quote}

\end{fulllineitems}

\index{get\_reference\_gain() (TiePieLCR\_settings.TiePieLCR\_settings method)@\spxentry{get\_reference\_gain()}\spxextra{TiePieLCR\_settings.TiePieLCR\_settings method}}

\begin{fulllineitems}
\phantomsection\label{\detokenize{index:TiePieLCR_settings.TiePieLCR_settings.get_reference_gain}}\pysiglinewithargsret{\sphinxbfcode{\sphinxupquote{get\_reference\_gain}}}{}{}
\sphinxAtStartPar
Get the gain of the transimpedance amplifier
\begin{quote}\begin{description}
\item[{Returns}] \leavevmode
\sphinxAtStartPar
The gain of the transimpedance amplifier in A/V or C/V

\item[{Return type}] \leavevmode
\sphinxAtStartPar
float

\end{description}\end{quote}

\end{fulllineitems}

\index{get\_reference\_name\_list() (TiePieLCR\_settings.TiePieLCR\_settings method)@\spxentry{get\_reference\_name\_list()}\spxextra{TiePieLCR\_settings.TiePieLCR\_settings method}}

\begin{fulllineitems}
\phantomsection\label{\detokenize{index:TiePieLCR_settings.TiePieLCR_settings.get_reference_name_list}}\pysiglinewithargsret{\sphinxbfcode{\sphinxupquote{get\_reference\_name\_list}}}{}{}
\sphinxAtStartPar
Get a list of available transimpedance amplifiers
\begin{quote}\begin{description}
\item[{Returns}] \leavevmode
\sphinxAtStartPar
List of transimpedance amplifiers

\item[{Return type}] \leavevmode
\sphinxAtStartPar
List of floats

\end{description}\end{quote}

\end{fulllineitems}

\index{get\_reference\_offset\_unit() (TiePieLCR\_settings.TiePieLCR\_settings method)@\spxentry{get\_reference\_offset\_unit()}\spxextra{TiePieLCR\_settings.TiePieLCR\_settings method}}

\begin{fulllineitems}
\phantomsection\label{\detokenize{index:TiePieLCR_settings.TiePieLCR_settings.get_reference_offset_unit}}\pysiglinewithargsret{\sphinxbfcode{\sphinxupquote{get\_reference\_offset\_unit}}}{}{}
\sphinxAtStartPar
Get the unit of the reference offset.
\begin{quote}\begin{description}
\item[{Returns}] \leavevmode
\sphinxAtStartPar
A string with the abbreviation of the unit. i.e. ‘V’ or ‘A’

\item[{Return type}] \leavevmode
\sphinxAtStartPar
String

\end{description}\end{quote}

\end{fulllineitems}

\index{get\_reference\_scope\_coupling() (TiePieLCR\_settings.TiePieLCR\_settings method)@\spxentry{get\_reference\_scope\_coupling()}\spxextra{TiePieLCR\_settings.TiePieLCR\_settings method}}

\begin{fulllineitems}
\phantomsection\label{\detokenize{index:TiePieLCR_settings.TiePieLCR_settings.get_reference_scope_coupling}}\pysiglinewithargsret{\sphinxbfcode{\sphinxupquote{get\_reference\_scope\_coupling}}}{}{}
\sphinxAtStartPar
Get the index of the coupling of the channel of the scope that is used to measure the reference
\begin{quote}\begin{description}
\item[{Returns}] \leavevmode
\sphinxAtStartPar
Index in scope\_coupling\_name\_list

\item[{Return type}] \leavevmode
\sphinxAtStartPar
Int

\end{description}\end{quote}

\end{fulllineitems}

\index{get\_reference\_scope\_range\_index() (TiePieLCR\_settings.TiePieLCR\_settings method)@\spxentry{get\_reference\_scope\_range\_index()}\spxextra{TiePieLCR\_settings.TiePieLCR\_settings method}}

\begin{fulllineitems}
\phantomsection\label{\detokenize{index:TiePieLCR_settings.TiePieLCR_settings.get_reference_scope_range_index}}\pysiglinewithargsret{\sphinxbfcode{\sphinxupquote{get\_reference\_scope\_range\_index}}}{}{}
\sphinxAtStartPar
Get which element of the scope range list is selected for the reference channel
\begin{quote}\begin{description}
\item[{Returns}] \leavevmode
\sphinxAtStartPar
The current index

\item[{Return type}] \leavevmode
\sphinxAtStartPar
Int

\end{description}\end{quote}

\end{fulllineitems}

\index{get\_reference\_scope\_range\_value() (TiePieLCR\_settings.TiePieLCR\_settings method)@\spxentry{get\_reference\_scope\_range\_value()}\spxextra{TiePieLCR\_settings.TiePieLCR\_settings method}}

\begin{fulllineitems}
\phantomsection\label{\detokenize{index:TiePieLCR_settings.TiePieLCR_settings.get_reference_scope_range_value}}\pysiglinewithargsret{\sphinxbfcode{\sphinxupquote{get\_reference\_scope\_range\_value}}}{}{}
\sphinxAtStartPar
Get the voltage range of the channel of the scope that is used to measure the reference
\begin{quote}\begin{description}
\item[{Returns}] \leavevmode
\sphinxAtStartPar
The range as a float. i.e. 0.2 or 2

\item[{Return type}] \leavevmode
\sphinxAtStartPar
Float

\end{description}\end{quote}

\end{fulllineitems}

\index{get\_reference\_setting() (TiePieLCR\_settings.TiePieLCR\_settings method)@\spxentry{get\_reference\_setting()}\spxextra{TiePieLCR\_settings.TiePieLCR\_settings method}}

\begin{fulllineitems}
\phantomsection\label{\detokenize{index:TiePieLCR_settings.TiePieLCR_settings.get_reference_setting}}\pysiglinewithargsret{\sphinxbfcode{\sphinxupquote{get\_reference\_setting}}}{}{}
\sphinxAtStartPar
Get the index of the currently selected reference setting. The list of possible references can be obtained using {\hyperref[\detokenize{index:TiePieLCR_settings.TiePieLCR_settings.get_gain_name_list}]{\sphinxcrossref{\sphinxcode{\sphinxupquote{TiePieLCR\_settings.TiePieLCR\_settings.get\_gain\_name\_list()}}}}}
\begin{quote}\begin{description}
\item[{Returns}] \leavevmode
\sphinxAtStartPar
The index

\item[{Return type}] \leavevmode
\sphinxAtStartPar
Int

\end{description}\end{quote}

\end{fulllineitems}

\index{get\_reference\_unit() (TiePieLCR\_settings.TiePieLCR\_settings method)@\spxentry{get\_reference\_unit()}\spxextra{TiePieLCR\_settings.TiePieLCR\_settings method}}

\begin{fulllineitems}
\phantomsection\label{\detokenize{index:TiePieLCR_settings.TiePieLCR_settings.get_reference_unit}}\pysiglinewithargsret{\sphinxbfcode{\sphinxupquote{get\_reference\_unit}}}{}{}
\sphinxAtStartPar
Get the unit of the reference signal.
\begin{quote}\begin{description}
\item[{Returns}] \leavevmode
\sphinxAtStartPar
A string with the abbreviation of the unit. i.e. ‘V’, ‘A’ or ‘C’

\item[{Return type}] \leavevmode
\sphinxAtStartPar
String

\end{description}\end{quote}

\end{fulllineitems}

\index{get\_sample\_frequency() (TiePieLCR\_settings.TiePieLCR\_settings method)@\spxentry{get\_sample\_frequency()}\spxextra{TiePieLCR\_settings.TiePieLCR\_settings method}}

\begin{fulllineitems}
\phantomsection\label{\detokenize{index:TiePieLCR_settings.TiePieLCR_settings.get_sample_frequency}}\pysiglinewithargsret{\sphinxbfcode{\sphinxupquote{get\_sample\_frequency}}}{}{}
\sphinxAtStartPar
Get the sample frequency at which the scope is running. A higher value results in less noise, however it might be reduced if the LCR stop and gives an error complaining that your PC is not fast enough.
\begin{quote}\begin{description}
\item[{Returns}] \leavevmode
\sphinxAtStartPar
The sample frequency

\item[{Return type}] \leavevmode
\sphinxAtStartPar
Float

\end{description}\end{quote}

\end{fulllineitems}

\index{get\_scope\_coupling\_name\_list() (TiePieLCR\_settings.TiePieLCR\_settings method)@\spxentry{get\_scope\_coupling\_name\_list()}\spxextra{TiePieLCR\_settings.TiePieLCR\_settings method}}

\begin{fulllineitems}
\phantomsection\label{\detokenize{index:TiePieLCR_settings.TiePieLCR_settings.get_scope_coupling_name_list}}\pysiglinewithargsret{\sphinxbfcode{\sphinxupquote{get\_scope\_coupling\_name\_list}}}{}{}
\sphinxAtStartPar
Get a list with the different coupling options
\begin{quote}\begin{description}
\item[{Returns}] \leavevmode
\sphinxAtStartPar
A list containing the coupling options

\item[{Return type}] \leavevmode
\sphinxAtStartPar
Int

\end{description}\end{quote}

\end{fulllineitems}

\index{get\_scope\_range\_list() (TiePieLCR\_settings.TiePieLCR\_settings method)@\spxentry{get\_scope\_range\_list()}\spxextra{TiePieLCR\_settings.TiePieLCR\_settings method}}

\begin{fulllineitems}
\phantomsection\label{\detokenize{index:TiePieLCR_settings.TiePieLCR_settings.get_scope_range_list}}\pysiglinewithargsret{\sphinxbfcode{\sphinxupquote{get\_scope\_range\_list}}}{}{}
\sphinxAtStartPar
Get a list of the different gains of the scope that can be used.
\begin{quote}\begin{description}
\item[{Returns}] \leavevmode
\sphinxAtStartPar
The current index

\item[{Return type}] \leavevmode
\sphinxAtStartPar
Int

\end{description}\end{quote}

\end{fulllineitems}

\index{get\_settings\_dict() (TiePieLCR\_settings.TiePieLCR\_settings method)@\spxentry{get\_settings\_dict()}\spxextra{TiePieLCR\_settings.TiePieLCR\_settings method}}

\begin{fulllineitems}
\phantomsection\label{\detokenize{index:TiePieLCR_settings.TiePieLCR_settings.get_settings_dict}}\pysiglinewithargsret{\sphinxbfcode{\sphinxupquote{get\_settings\_dict}}}{\emph{\DUrole{n}{filename}}}{}
\sphinxAtStartPar
Get the settings in the object formatted as a dictionary object.
\begin{quote}\begin{description}
\item[{Returns}] \leavevmode
\sphinxAtStartPar
A dict with the settings in this object

\item[{Return type}] \leavevmode
\sphinxAtStartPar
Dictionary

\end{description}\end{quote}

\end{fulllineitems}

\index{get\_signal\_scope\_coupling() (TiePieLCR\_settings.TiePieLCR\_settings method)@\spxentry{get\_signal\_scope\_coupling()}\spxextra{TiePieLCR\_settings.TiePieLCR\_settings method}}

\begin{fulllineitems}
\phantomsection\label{\detokenize{index:TiePieLCR_settings.TiePieLCR_settings.get_signal_scope_coupling}}\pysiglinewithargsret{\sphinxbfcode{\sphinxupquote{get\_signal\_scope\_coupling}}}{}{}
\sphinxAtStartPar
Get the index of the coupling of the channel of the scope that is used to measure the signal
\begin{quote}\begin{description}
\item[{Returns}] \leavevmode
\sphinxAtStartPar
0 for AC, 1 for DC

\item[{Return type}] \leavevmode
\sphinxAtStartPar
Int

\end{description}\end{quote}

\end{fulllineitems}

\index{get\_signal\_scope\_range\_index() (TiePieLCR\_settings.TiePieLCR\_settings method)@\spxentry{get\_signal\_scope\_range\_index()}\spxextra{TiePieLCR\_settings.TiePieLCR\_settings method}}

\begin{fulllineitems}
\phantomsection\label{\detokenize{index:TiePieLCR_settings.TiePieLCR_settings.get_signal_scope_range_index}}\pysiglinewithargsret{\sphinxbfcode{\sphinxupquote{get\_signal\_scope\_range\_index}}}{}{}
\sphinxAtStartPar
Get which element of the scope range list is selected for the signal channel
\begin{quote}\begin{description}
\item[{Returns}] \leavevmode
\sphinxAtStartPar
The current index

\item[{Return type}] \leavevmode
\sphinxAtStartPar
Int

\end{description}\end{quote}

\end{fulllineitems}

\index{get\_signal\_scope\_range\_value() (TiePieLCR\_settings.TiePieLCR\_settings method)@\spxentry{get\_signal\_scope\_range\_value()}\spxextra{TiePieLCR\_settings.TiePieLCR\_settings method}}

\begin{fulllineitems}
\phantomsection\label{\detokenize{index:TiePieLCR_settings.TiePieLCR_settings.get_signal_scope_range_value}}\pysiglinewithargsret{\sphinxbfcode{\sphinxupquote{get\_signal\_scope\_range\_value}}}{}{}
\sphinxAtStartPar
Get the voltage range of the channel of the scope that is used to measure the signal
\begin{quote}\begin{description}
\item[{Returns}] \leavevmode
\sphinxAtStartPar
The range as a float. i.e. 0.2 or 2

\item[{Return type}] \leavevmode
\sphinxAtStartPar
Float

\end{description}\end{quote}

\end{fulllineitems}

\index{get\_sub\_block\_freq() (TiePieLCR\_settings.TiePieLCR\_settings method)@\spxentry{get\_sub\_block\_freq()}\spxextra{TiePieLCR\_settings.TiePieLCR\_settings method}}

\begin{fulllineitems}
\phantomsection\label{\detokenize{index:TiePieLCR_settings.TiePieLCR_settings.get_sub_block_freq}}\pysiglinewithargsret{\sphinxbfcode{\sphinxupquote{get\_sub\_block\_freq}}}{}{}
\sphinxAtStartPar
The number of sub blocks per second that need to be processed. This equavalent to the number of FFT’s that are being taken per second.
\begin{quote}\begin{description}
\item[{Returns}] \leavevmode
\sphinxAtStartPar
The number of sub\sphinxhyphen{}blocks per second

\item[{Return type}] \leavevmode
\sphinxAtStartPar
Int

\end{description}\end{quote}

\end{fulllineitems}

\index{get\_sub\_block\_size() (TiePieLCR\_settings.TiePieLCR\_settings method)@\spxentry{get\_sub\_block\_size()}\spxextra{TiePieLCR\_settings.TiePieLCR\_settings method}}

\begin{fulllineitems}
\phantomsection\label{\detokenize{index:TiePieLCR_settings.TiePieLCR_settings.get_sub_block_size}}\pysiglinewithargsret{\sphinxbfcode{\sphinxupquote{get\_sub\_block\_size}}}{}{}
\sphinxAtStartPar
Get the size of the blocks on which the fft is performed. Since the time it takes to do an fft increases with a power of 1.4 with the amount of points, a higher number of sub\sphinxhyphen{}blocks will reduce the computation load. A higher number sub\sphinxhyphen{}blocks also moves the frequency at which noise at the edge of each block will appear. A higher number of sub\sphinxhyphen{}blocks makes very lower frequency measurements more computationally intensive though.
\begin{quote}\begin{description}
\item[{Returns}] \leavevmode
\sphinxAtStartPar
The number of samples per sub\sphinxhyphen{}block

\item[{Return type}] \leavevmode
\sphinxAtStartPar
Int

\end{description}\end{quote}

\end{fulllineitems}

\index{get\_sub\_blocks() (TiePieLCR\_settings.TiePieLCR\_settings method)@\spxentry{get\_sub\_blocks()}\spxextra{TiePieLCR\_settings.TiePieLCR\_settings method}}

\begin{fulllineitems}
\phantomsection\label{\detokenize{index:TiePieLCR_settings.TiePieLCR_settings.get_sub_blocks}}\pysiglinewithargsret{\sphinxbfcode{\sphinxupquote{get\_sub\_blocks}}}{}{}
\sphinxAtStartPar
Get in how many blocks the data retrieved from the scope is split up before doing the fft. Since the time it takes to do an fft increases with a power of 1.4 with the amount of points, a higher number of sub\sphinxhyphen{}blocks will reduce the computation load. A higher number sub\sphinxhyphen{}blocks also moves the frequency at which noise at the edge of each block will appear. A higher number of sub\sphinxhyphen{}blocks makes very lower frequency measurements more computationally intensive though.
\begin{quote}\begin{description}
\item[{Returns}] \leavevmode
\sphinxAtStartPar
The number of sub\sphinxhyphen{}blocks

\item[{Return type}] \leavevmode
\sphinxAtStartPar
Int

\end{description}\end{quote}

\end{fulllineitems}

\index{get\_sub\_blocks\_list() (TiePieLCR\_settings.TiePieLCR\_settings method)@\spxentry{get\_sub\_blocks\_list()}\spxextra{TiePieLCR\_settings.TiePieLCR\_settings method}}

\begin{fulllineitems}
\phantomsection\label{\detokenize{index:TiePieLCR_settings.TiePieLCR_settings.get_sub_blocks_list}}\pysiglinewithargsret{\sphinxbfcode{\sphinxupquote{get\_sub\_blocks\_list}}}{}{}
\sphinxAtStartPar
get a list of options to which the sub\sphinxhyphen{}blocks can be set.
\begin{quote}\begin{description}
\item[{Returns}] \leavevmode
\sphinxAtStartPar
A list with the possible sub\sphinxhyphen{}block settings

\item[{Return type}] \leavevmode
\sphinxAtStartPar
List of Ints

\end{description}\end{quote}

\end{fulllineitems}

\index{get\_time\_plot\_points() (TiePieLCR\_settings.TiePieLCR\_settings method)@\spxentry{get\_time\_plot\_points()}\spxextra{TiePieLCR\_settings.TiePieLCR\_settings method}}

\begin{fulllineitems}
\phantomsection\label{\detokenize{index:TiePieLCR_settings.TiePieLCR_settings.get_time_plot_points}}\pysiglinewithargsret{\sphinxbfcode{\sphinxupquote{get\_time\_plot\_points}}}{}{}
\sphinxAtStartPar
Get the number of points that are plotted in the time plots. Setting this to a to high value will make things slow, settings this to a too high value will cause aliasing.
\begin{quote}\begin{description}
\item[{Returns}] \leavevmode
\sphinxAtStartPar
The number of points

\item[{Return type}] \leavevmode
\sphinxAtStartPar
Int

\end{description}\end{quote}

\end{fulllineitems}

\index{get\_update\_frequency() (TiePieLCR\_settings.TiePieLCR\_settings method)@\spxentry{get\_update\_frequency()}\spxextra{TiePieLCR\_settings.TiePieLCR\_settings method}}

\begin{fulllineitems}
\phantomsection\label{\detokenize{index:TiePieLCR_settings.TiePieLCR_settings.get_update_frequency}}\pysiglinewithargsret{\sphinxbfcode{\sphinxupquote{get\_update\_frequency}}}{}{}
\sphinxAtStartPar
Get how often data is retrieved from the scope, the gui is updated and how often the value obtained using \sphinxcode{\sphinxupquote{TiePieLCR\_api.TiePieLCR\_api.get\_impedance()}} is updated. Only reduce this if your computer is really really slow..
\begin{quote}\begin{description}
\item[{Returns}] \leavevmode
\sphinxAtStartPar
The update frequency

\item[{Return type}] \leavevmode
\sphinxAtStartPar
Float

\end{description}\end{quote}

\end{fulllineitems}

\index{impedance\_format (TiePieLCR\_settings.TiePieLCR\_settings attribute)@\spxentry{impedance\_format}\spxextra{TiePieLCR\_settings.TiePieLCR\_settings attribute}}

\begin{fulllineitems}
\phantomsection\label{\detokenize{index:TiePieLCR_settings.TiePieLCR_settings.impedance_format}}\pysigline{\sphinxbfcode{\sphinxupquote{impedance\_format}}\sphinxbfcode{\sphinxupquote{\DUrole{w}{  }\DUrole{p}{=}\DUrole{w}{  }0}}}
\end{fulllineitems}

\index{impedance\_format\_label1 (TiePieLCR\_settings.TiePieLCR\_settings attribute)@\spxentry{impedance\_format\_label1}\spxextra{TiePieLCR\_settings.TiePieLCR\_settings attribute}}

\begin{fulllineitems}
\phantomsection\label{\detokenize{index:TiePieLCR_settings.TiePieLCR_settings.impedance_format_label1}}\pysigline{\sphinxbfcode{\sphinxupquote{impedance\_format\_label1}}\sphinxbfcode{\sphinxupquote{\DUrole{w}{  }\DUrole{p}{=}\DUrole{w}{  }{[}\textquotesingle{}X\textquotesingle{}, \textquotesingle{}Rp\textquotesingle{}, \textquotesingle{}Rs\textquotesingle{}, \textquotesingle{}Z\textquotesingle{}{]}}}}
\end{fulllineitems}

\index{impedance\_format\_label2 (TiePieLCR\_settings.TiePieLCR\_settings attribute)@\spxentry{impedance\_format\_label2}\spxextra{TiePieLCR\_settings.TiePieLCR\_settings attribute}}

\begin{fulllineitems}
\phantomsection\label{\detokenize{index:TiePieLCR_settings.TiePieLCR_settings.impedance_format_label2}}\pysigline{\sphinxbfcode{\sphinxupquote{impedance\_format\_label2}}\sphinxbfcode{\sphinxupquote{\DUrole{w}{  }\DUrole{p}{=}\DUrole{w}{  }{[}\textquotesingle{}Y\textquotesingle{}, \textquotesingle{}Cp\textquotesingle{}, \textquotesingle{}Cs\textquotesingle{}, \textquotesingle{}Phi\textquotesingle{}{]}}}}
\end{fulllineitems}

\index{impedance\_format\_unit1 (TiePieLCR\_settings.TiePieLCR\_settings attribute)@\spxentry{impedance\_format\_unit1}\spxextra{TiePieLCR\_settings.TiePieLCR\_settings attribute}}

\begin{fulllineitems}
\phantomsection\label{\detokenize{index:TiePieLCR_settings.TiePieLCR_settings.impedance_format_unit1}}\pysigline{\sphinxbfcode{\sphinxupquote{impedance\_format\_unit1}}\sphinxbfcode{\sphinxupquote{\DUrole{w}{  }\DUrole{p}{=}\DUrole{w}{  }{[}\textquotesingle{}Ω\textquotesingle{}, \textquotesingle{}Ω\textquotesingle{}, \textquotesingle{}Ω\textquotesingle{}, \textquotesingle{}Ω\textquotesingle{}{]}}}}
\end{fulllineitems}

\index{impedance\_format\_unit2 (TiePieLCR\_settings.TiePieLCR\_settings attribute)@\spxentry{impedance\_format\_unit2}\spxextra{TiePieLCR\_settings.TiePieLCR\_settings attribute}}

\begin{fulllineitems}
\phantomsection\label{\detokenize{index:TiePieLCR_settings.TiePieLCR_settings.impedance_format_unit2}}\pysigline{\sphinxbfcode{\sphinxupquote{impedance\_format\_unit2}}\sphinxbfcode{\sphinxupquote{\DUrole{w}{  }\DUrole{p}{=}\DUrole{w}{  }{[}\textquotesingle{}Ω\textquotesingle{}, \textquotesingle{}F\textquotesingle{}, \textquotesingle{}F\textquotesingle{}, \textquotesingle{}rad\textquotesingle{}{]}}}}
\end{fulllineitems}

\index{impedance\_formats (TiePieLCR\_settings.TiePieLCR\_settings attribute)@\spxentry{impedance\_formats}\spxextra{TiePieLCR\_settings.TiePieLCR\_settings attribute}}

\begin{fulllineitems}
\phantomsection\label{\detokenize{index:TiePieLCR_settings.TiePieLCR_settings.impedance_formats}}\pysigline{\sphinxbfcode{\sphinxupquote{impedance\_formats}}\sphinxbfcode{\sphinxupquote{\DUrole{w}{  }\DUrole{p}{=}\DUrole{w}{  }{[}\textquotesingle{}XY\textquotesingle{}, \textquotesingle{}RpCp\textquotesingle{}, \textquotesingle{}RsCs\textquotesingle{}, \textquotesingle{}ZPhi\textquotesingle{}{]}}}}
\end{fulllineitems}

\index{inst (TiePieLCR\_settings.TiePieLCR\_settings attribute)@\spxentry{inst}\spxextra{TiePieLCR\_settings.TiePieLCR\_settings attribute}}

\begin{fulllineitems}
\phantomsection\label{\detokenize{index:TiePieLCR_settings.TiePieLCR_settings.inst}}\pysigline{\sphinxbfcode{\sphinxupquote{inst}}\sphinxbfcode{\sphinxupquote{\DUrole{w}{  }\DUrole{p}{=}\DUrole{w}{  }1}}}
\end{fulllineitems}

\index{integration\_time (TiePieLCR\_settings.TiePieLCR\_settings attribute)@\spxentry{integration\_time}\spxextra{TiePieLCR\_settings.TiePieLCR\_settings attribute}}

\begin{fulllineitems}
\phantomsection\label{\detokenize{index:TiePieLCR_settings.TiePieLCR_settings.integration_time}}\pysigline{\sphinxbfcode{\sphinxupquote{integration\_time}}\sphinxbfcode{\sphinxupquote{\DUrole{w}{  }\DUrole{p}{=}\DUrole{w}{  }1}}}
\end{fulllineitems}

\index{lcm() (TiePieLCR\_settings.TiePieLCR\_settings static method)@\spxentry{lcm()}\spxextra{TiePieLCR\_settings.TiePieLCR\_settings static method}}

\begin{fulllineitems}
\phantomsection\label{\detokenize{index:TiePieLCR_settings.TiePieLCR_settings.lcm}}\pysiglinewithargsret{\sphinxbfcode{\sphinxupquote{static\DUrole{w}{  }}}\sphinxbfcode{\sphinxupquote{lcm}}}{\emph{\DUrole{n}{L}}}{}
\end{fulllineitems}

\index{load\_settings() (TiePieLCR\_settings.TiePieLCR\_settings method)@\spxentry{load\_settings()}\spxextra{TiePieLCR\_settings.TiePieLCR\_settings method}}

\begin{fulllineitems}
\phantomsection\label{\detokenize{index:TiePieLCR_settings.TiePieLCR_settings.load_settings}}\pysiglinewithargsret{\sphinxbfcode{\sphinxupquote{load\_settings}}}{\emph{\DUrole{n}{filename}}}{}
\sphinxAtStartPar
Loads a settings file and applies the settings to this object
\begin{quote}\begin{description}
\item[{Parameters}] \leavevmode
\sphinxAtStartPar
\sphinxstyleliteralstrong{\sphinxupquote{filename}} (\sphinxstyleliteralemphasis{\sphinxupquote{String}}) \textendash{} The filename

\item[{Returns}] \leavevmode
\sphinxAtStartPar
True if the file exists, False otherwise

\item[{Return type}] \leavevmode
\sphinxAtStartPar
Boolean

\end{description}\end{quote}

\end{fulllineitems}

\index{max\_sample\_frequency (TiePieLCR\_settings.TiePieLCR\_settings attribute)@\spxentry{max\_sample\_frequency}\spxextra{TiePieLCR\_settings.TiePieLCR\_settings attribute}}

\begin{fulllineitems}
\phantomsection\label{\detokenize{index:TiePieLCR_settings.TiePieLCR_settings.max_sample_frequency}}\pysigline{\sphinxbfcode{\sphinxupquote{max\_sample\_frequency}}\sphinxbfcode{\sphinxupquote{\DUrole{w}{  }\DUrole{p}{=}\DUrole{w}{  }6250000.0}}}
\end{fulllineitems}

\index{multisine\_update\_required (TiePieLCR\_settings.TiePieLCR\_settings attribute)@\spxentry{multisine\_update\_required}\spxextra{TiePieLCR\_settings.TiePieLCR\_settings attribute}}

\begin{fulllineitems}
\phantomsection\label{\detokenize{index:TiePieLCR_settings.TiePieLCR_settings.multisine_update_required}}\pysigline{\sphinxbfcode{\sphinxupquote{multisine\_update\_required}}\sphinxbfcode{\sphinxupquote{\DUrole{w}{  }\DUrole{p}{=}\DUrole{w}{  }False}}}
\end{fulllineitems}

\index{offset\_bandwidth (TiePieLCR\_settings.TiePieLCR\_settings attribute)@\spxentry{offset\_bandwidth}\spxextra{TiePieLCR\_settings.TiePieLCR\_settings attribute}}

\begin{fulllineitems}
\phantomsection\label{\detokenize{index:TiePieLCR_settings.TiePieLCR_settings.offset_bandwidth}}\pysigline{\sphinxbfcode{\sphinxupquote{offset\_bandwidth}}\sphinxbfcode{\sphinxupquote{\DUrole{w}{  }\DUrole{p}{=}\DUrole{w}{  }20}}}
\end{fulllineitems}

\index{offset\_integration\_time (TiePieLCR\_settings.TiePieLCR\_settings attribute)@\spxentry{offset\_integration\_time}\spxextra{TiePieLCR\_settings.TiePieLCR\_settings attribute}}

\begin{fulllineitems}
\phantomsection\label{\detokenize{index:TiePieLCR_settings.TiePieLCR_settings.offset_integration_time}}\pysigline{\sphinxbfcode{\sphinxupquote{offset\_integration\_time}}\sphinxbfcode{\sphinxupquote{\DUrole{w}{  }\DUrole{p}{=}\DUrole{w}{  }1}}}
\end{fulllineitems}

\index{optimise\_crest() (TiePieLCR\_settings.TiePieLCR\_settings method)@\spxentry{optimise\_crest()}\spxextra{TiePieLCR\_settings.TiePieLCR\_settings method}}

\begin{fulllineitems}
\phantomsection\label{\detokenize{index:TiePieLCR_settings.TiePieLCR_settings.optimise_crest}}\pysiglinewithargsret{\sphinxbfcode{\sphinxupquote{optimise\_crest}}}{}{}
\end{fulllineitems}

\index{output\_block\_size (TiePieLCR\_settings.TiePieLCR\_settings attribute)@\spxentry{output\_block\_size}\spxextra{TiePieLCR\_settings.TiePieLCR\_settings attribute}}

\begin{fulllineitems}
\phantomsection\label{\detokenize{index:TiePieLCR_settings.TiePieLCR_settings.output_block_size}}\pysigline{\sphinxbfcode{\sphinxupquote{output\_block\_size}}\sphinxbfcode{\sphinxupquote{\DUrole{w}{  }\DUrole{p}{=}\DUrole{w}{  }0}}}
\end{fulllineitems}

\index{output\_oversample\_ratio (TiePieLCR\_settings.TiePieLCR\_settings attribute)@\spxentry{output\_oversample\_ratio}\spxextra{TiePieLCR\_settings.TiePieLCR\_settings attribute}}

\begin{fulllineitems}
\phantomsection\label{\detokenize{index:TiePieLCR_settings.TiePieLCR_settings.output_oversample_ratio}}\pysigline{\sphinxbfcode{\sphinxupquote{output\_oversample\_ratio}}\sphinxbfcode{\sphinxupquote{\DUrole{w}{  }\DUrole{p}{=}\DUrole{w}{  }2}}}
\end{fulllineitems}

\index{plot\_periods (TiePieLCR\_settings.TiePieLCR\_settings attribute)@\spxentry{plot\_periods}\spxextra{TiePieLCR\_settings.TiePieLCR\_settings attribute}}

\begin{fulllineitems}
\phantomsection\label{\detokenize{index:TiePieLCR_settings.TiePieLCR_settings.plot_periods}}\pysigline{\sphinxbfcode{\sphinxupquote{plot\_periods}}\sphinxbfcode{\sphinxupquote{\DUrole{w}{  }\DUrole{p}{=}\DUrole{w}{  }3}}}
\end{fulllineitems}

\index{plot\_points (TiePieLCR\_settings.TiePieLCR\_settings attribute)@\spxentry{plot\_points}\spxextra{TiePieLCR\_settings.TiePieLCR\_settings attribute}}

\begin{fulllineitems}
\phantomsection\label{\detokenize{index:TiePieLCR_settings.TiePieLCR_settings.plot_points}}\pysigline{\sphinxbfcode{\sphinxupquote{plot\_points}}\sphinxbfcode{\sphinxupquote{\DUrole{w}{  }\DUrole{p}{=}\DUrole{w}{  }500}}}
\end{fulllineitems}

\index{plot\_time (TiePieLCR\_settings.TiePieLCR\_settings attribute)@\spxentry{plot\_time}\spxextra{TiePieLCR\_settings.TiePieLCR\_settings attribute}}

\begin{fulllineitems}
\phantomsection\label{\detokenize{index:TiePieLCR_settings.TiePieLCR_settings.plot_time}}\pysigline{\sphinxbfcode{\sphinxupquote{plot\_time}}\sphinxbfcode{\sphinxupquote{\DUrole{w}{  }\DUrole{p}{=}\DUrole{w}{  }5}}}
\end{fulllineitems}

\index{protection\_C (TiePieLCR\_settings.TiePieLCR\_settings attribute)@\spxentry{protection\_C}\spxextra{TiePieLCR\_settings.TiePieLCR\_settings attribute}}

\begin{fulllineitems}
\phantomsection\label{\detokenize{index:TiePieLCR_settings.TiePieLCR_settings.protection_C}}\pysigline{\sphinxbfcode{\sphinxupquote{protection\_C}}\sphinxbfcode{\sphinxupquote{\DUrole{w}{  }\DUrole{p}{=}\DUrole{w}{  }{[}0, 0, 3.9e\sphinxhyphen{}12, 3.9e\sphinxhyphen{}12, 0, 0{]}}}}
\end{fulllineitems}

\index{protection\_R (TiePieLCR\_settings.TiePieLCR\_settings attribute)@\spxentry{protection\_R}\spxextra{TiePieLCR\_settings.TiePieLCR\_settings attribute}}

\begin{fulllineitems}
\phantomsection\label{\detokenize{index:TiePieLCR_settings.TiePieLCR_settings.protection_R}}\pysigline{\sphinxbfcode{\sphinxupquote{protection\_R}}\sphinxbfcode{\sphinxupquote{\DUrole{w}{  }\DUrole{p}{=}\DUrole{w}{  }{[}0, 0, 2610.0, 2610.0, 0, 0{]}}}}
\end{fulllineitems}

\index{real\_time\_mode (TiePieLCR\_settings.TiePieLCR\_settings attribute)@\spxentry{real\_time\_mode}\spxextra{TiePieLCR\_settings.TiePieLCR\_settings attribute}}

\begin{fulllineitems}
\phantomsection\label{\detokenize{index:TiePieLCR_settings.TiePieLCR_settings.real_time_mode}}\pysigline{\sphinxbfcode{\sphinxupquote{real\_time\_mode}}\sphinxbfcode{\sphinxupquote{\DUrole{w}{  }\DUrole{p}{=}\DUrole{w}{  }False}}}
\end{fulllineitems}

\index{reference\_C (TiePieLCR\_settings.TiePieLCR\_settings attribute)@\spxentry{reference\_C}\spxextra{TiePieLCR\_settings.TiePieLCR\_settings attribute}}

\begin{fulllineitems}
\phantomsection\label{\detokenize{index:TiePieLCR_settings.TiePieLCR_settings.reference_C}}\pysigline{\sphinxbfcode{\sphinxupquote{reference\_C}}\sphinxbfcode{\sphinxupquote{\DUrole{w}{  }\DUrole{p}{=}\DUrole{w}{  }{[}0, 4e\sphinxhyphen{}12, 4e\sphinxhyphen{}12, 3.9e\sphinxhyphen{}10, 4e\sphinxhyphen{}12, 4e\sphinxhyphen{}12, 0{]}}}}
\end{fulllineitems}

\index{reference\_R (TiePieLCR\_settings.TiePieLCR\_settings attribute)@\spxentry{reference\_R}\spxextra{TiePieLCR\_settings.TiePieLCR\_settings attribute}}

\begin{fulllineitems}
\phantomsection\label{\detokenize{index:TiePieLCR_settings.TiePieLCR_settings.reference_R}}\pysigline{\sphinxbfcode{\sphinxupquote{reference\_R}}\sphinxbfcode{\sphinxupquote{\DUrole{w}{  }\DUrole{p}{=}\DUrole{w}{  }{[}1, 2610.0, 200000.0, 90000000.0, 10000000000.0, 2700.0, 1{]}}}}
\end{fulllineitems}

\index{reference\_gain\_list (TiePieLCR\_settings.TiePieLCR\_settings attribute)@\spxentry{reference\_gain\_list}\spxextra{TiePieLCR\_settings.TiePieLCR\_settings attribute}}

\begin{fulllineitems}
\phantomsection\label{\detokenize{index:TiePieLCR_settings.TiePieLCR_settings.reference_gain_list}}\pysigline{\sphinxbfcode{\sphinxupquote{reference\_gain\_list}}\sphinxbfcode{\sphinxupquote{\DUrole{w}{  }\DUrole{p}{=}\DUrole{w}{  }{[}1, \sphinxhyphen{}0.0003831417624521073, \sphinxhyphen{}5e\sphinxhyphen{}06, \sphinxhyphen{}3.9e\sphinxhyphen{}10, \sphinxhyphen{}3.9e\sphinxhyphen{}12, 0.0003831417624521073, 1{]}}}}
\end{fulllineitems}

\index{reference\_name\_list (TiePieLCR\_settings.TiePieLCR\_settings attribute)@\spxentry{reference\_name\_list}\spxextra{TiePieLCR\_settings.TiePieLCR\_settings attribute}}

\begin{fulllineitems}
\phantomsection\label{\detokenize{index:TiePieLCR_settings.TiePieLCR_settings.reference_name_list}}\pysigline{\sphinxbfcode{\sphinxupquote{reference\_name\_list}}\sphinxbfcode{\sphinxupquote{\DUrole{w}{  }\DUrole{p}{=}\DUrole{w}{  }{[}\textquotesingle{}None\textquotesingle{}, \textquotesingle{}Lcur 370 μA/V\textquotesingle{}, \textquotesingle{}Lcur 5 μA/V\textquotesingle{}, \textquotesingle{}Lcur 390 pC/V\textquotesingle{}, \textquotesingle{}Lcur 3.9 pC/V\textquotesingle{}, \textquotesingle{}HcurI\textquotesingle{}, \textquotesingle{}HcurV\textquotesingle{}{]}}}}
\end{fulllineitems}

\index{reference\_offset\_unit\_list (TiePieLCR\_settings.TiePieLCR\_settings attribute)@\spxentry{reference\_offset\_unit\_list}\spxextra{TiePieLCR\_settings.TiePieLCR\_settings attribute}}

\begin{fulllineitems}
\phantomsection\label{\detokenize{index:TiePieLCR_settings.TiePieLCR_settings.reference_offset_unit_list}}\pysigline{\sphinxbfcode{\sphinxupquote{reference\_offset\_unit\_list}}\sphinxbfcode{\sphinxupquote{\DUrole{w}{  }\DUrole{p}{=}\DUrole{w}{  }{[}\textquotesingle{}\textquotesingle{}, \textquotesingle{}A\textquotesingle{}, \textquotesingle{}A\textquotesingle{}, \textquotesingle{}A\textquotesingle{}, \textquotesingle{}A\textquotesingle{}, \textquotesingle{}A\textquotesingle{}, \textquotesingle{}V\textquotesingle{}{]}}}}
\end{fulllineitems}

\index{reference\_setting (TiePieLCR\_settings.TiePieLCR\_settings attribute)@\spxentry{reference\_setting}\spxextra{TiePieLCR\_settings.TiePieLCR\_settings attribute}}

\begin{fulllineitems}
\phantomsection\label{\detokenize{index:TiePieLCR_settings.TiePieLCR_settings.reference_setting}}\pysigline{\sphinxbfcode{\sphinxupquote{reference\_setting}}\sphinxbfcode{\sphinxupquote{\DUrole{w}{  }\DUrole{p}{=}\DUrole{w}{  }1}}}
\end{fulllineitems}

\index{reference\_unit\_list (TiePieLCR\_settings.TiePieLCR\_settings attribute)@\spxentry{reference\_unit\_list}\spxextra{TiePieLCR\_settings.TiePieLCR\_settings attribute}}

\begin{fulllineitems}
\phantomsection\label{\detokenize{index:TiePieLCR_settings.TiePieLCR_settings.reference_unit_list}}\pysigline{\sphinxbfcode{\sphinxupquote{reference\_unit\_list}}\sphinxbfcode{\sphinxupquote{\DUrole{w}{  }\DUrole{p}{=}\DUrole{w}{  }{[}\textquotesingle{}\textquotesingle{}, \textquotesingle{}A\textquotesingle{}, \textquotesingle{}A\textquotesingle{}, \textquotesingle{}C\textquotesingle{}, \textquotesingle{}C\textquotesingle{}, \textquotesingle{}A\textquotesingle{}, \textquotesingle{}V\textquotesingle{}{]}}}}
\end{fulllineitems}

\index{reference\_update\_required (TiePieLCR\_settings.TiePieLCR\_settings attribute)@\spxentry{reference\_update\_required}\spxextra{TiePieLCR\_settings.TiePieLCR\_settings attribute}}

\begin{fulllineitems}
\phantomsection\label{\detokenize{index:TiePieLCR_settings.TiePieLCR_settings.reference_update_required}}\pysigline{\sphinxbfcode{\sphinxupquote{reference\_update\_required}}\sphinxbfcode{\sphinxupquote{\DUrole{w}{  }\DUrole{p}{=}\DUrole{w}{  }False}}}
\end{fulllineitems}

\index{reset() (TiePieLCR\_settings.TiePieLCR\_settings method)@\spxentry{reset()}\spxextra{TiePieLCR\_settings.TiePieLCR\_settings method}}

\begin{fulllineitems}
\phantomsection\label{\detokenize{index:TiePieLCR_settings.TiePieLCR_settings.reset}}\pysiglinewithargsret{\sphinxbfcode{\sphinxupquote{reset}}}{}{}
\sphinxAtStartPar
When the settings are loaded in the TiePieLCR parts interface of the interface might be reset. By running this function before any changes are made to the settings object, only the parts that really need to be reset are reset.
\begin{quote}\begin{description}
\item[{Returns}] \leavevmode
\sphinxAtStartPar
Nothing

\item[{Return type}] \leavevmode
\sphinxAtStartPar
None

\end{description}\end{quote}

\end{fulllineitems}

\index{restart\_required (TiePieLCR\_settings.TiePieLCR\_settings attribute)@\spxentry{restart\_required}\spxextra{TiePieLCR\_settings.TiePieLCR\_settings attribute}}

\begin{fulllineitems}
\phantomsection\label{\detokenize{index:TiePieLCR_settings.TiePieLCR_settings.restart_required}}\pysigline{\sphinxbfcode{\sphinxupquote{restart\_required}}\sphinxbfcode{\sphinxupquote{\DUrole{w}{  }\DUrole{p}{=}\DUrole{w}{  }False}}}
\end{fulllineitems}

\index{sample\_frequency (TiePieLCR\_settings.TiePieLCR\_settings attribute)@\spxentry{sample\_frequency}\spxextra{TiePieLCR\_settings.TiePieLCR\_settings attribute}}

\begin{fulllineitems}
\phantomsection\label{\detokenize{index:TiePieLCR_settings.TiePieLCR_settings.sample_frequency}}\pysigline{\sphinxbfcode{\sphinxupquote{sample\_frequency}}\sphinxbfcode{\sphinxupquote{\DUrole{w}{  }\DUrole{p}{=}\DUrole{w}{  }6250000.0}}}
\end{fulllineitems}

\index{save\_memory (TiePieLCR\_settings.TiePieLCR\_settings attribute)@\spxentry{save\_memory}\spxextra{TiePieLCR\_settings.TiePieLCR\_settings attribute}}

\begin{fulllineitems}
\phantomsection\label{\detokenize{index:TiePieLCR_settings.TiePieLCR_settings.save_memory}}\pysigline{\sphinxbfcode{\sphinxupquote{save\_memory}}\sphinxbfcode{\sphinxupquote{\DUrole{w}{  }\DUrole{p}{=}\DUrole{w}{  }10000000.0}}}
\end{fulllineitems}

\index{save\_settings() (TiePieLCR\_settings.TiePieLCR\_settings method)@\spxentry{save\_settings()}\spxextra{TiePieLCR\_settings.TiePieLCR\_settings method}}

\begin{fulllineitems}
\phantomsection\label{\detokenize{index:TiePieLCR_settings.TiePieLCR_settings.save_settings}}\pysiglinewithargsret{\sphinxbfcode{\sphinxupquote{save\_settings}}}{\emph{\DUrole{n}{filename}}}{}
\sphinxAtStartPar
Saves a settings file and with the settings of this object
\begin{quote}\begin{description}
\item[{Parameters}] \leavevmode
\sphinxAtStartPar
\sphinxstyleliteralstrong{\sphinxupquote{filename}} (\sphinxstyleliteralemphasis{\sphinxupquote{String}}) \textendash{} The filename

\item[{Returns}] \leavevmode
\sphinxAtStartPar
Nothing

\item[{Return type}] \leavevmode
\sphinxAtStartPar
none

\end{description}\end{quote}

\end{fulllineitems}

\index{scope\_auto\_ranging (TiePieLCR\_settings.TiePieLCR\_settings attribute)@\spxentry{scope\_auto\_ranging}\spxextra{TiePieLCR\_settings.TiePieLCR\_settings attribute}}

\begin{fulllineitems}
\phantomsection\label{\detokenize{index:TiePieLCR_settings.TiePieLCR_settings.scope_auto_ranging}}\pysigline{\sphinxbfcode{\sphinxupquote{scope\_auto\_ranging}}\sphinxbfcode{\sphinxupquote{\DUrole{w}{  }\DUrole{p}{=}\DUrole{w}{  }{[}True, True{]}}}}
\end{fulllineitems}

\index{scope\_coupling\_name\_list (TiePieLCR\_settings.TiePieLCR\_settings attribute)@\spxentry{scope\_coupling\_name\_list}\spxextra{TiePieLCR\_settings.TiePieLCR\_settings attribute}}

\begin{fulllineitems}
\phantomsection\label{\detokenize{index:TiePieLCR_settings.TiePieLCR_settings.scope_coupling_name_list}}\pysigline{\sphinxbfcode{\sphinxupquote{scope\_coupling\_name\_list}}\sphinxbfcode{\sphinxupquote{\DUrole{w}{  }\DUrole{p}{=}\DUrole{w}{  }{[}\textquotesingle{}AC\textquotesingle{}, \textquotesingle{}DC\textquotesingle{}{]}}}}
\end{fulllineitems}

\index{scope\_couplings (TiePieLCR\_settings.TiePieLCR\_settings attribute)@\spxentry{scope\_couplings}\spxextra{TiePieLCR\_settings.TiePieLCR\_settings attribute}}

\begin{fulllineitems}
\phantomsection\label{\detokenize{index:TiePieLCR_settings.TiePieLCR_settings.scope_couplings}}\pysigline{\sphinxbfcode{\sphinxupquote{scope\_couplings}}\sphinxbfcode{\sphinxupquote{\DUrole{w}{  }\DUrole{p}{=}\DUrole{w}{  }{[}1, 1{]}}}}
\end{fulllineitems}

\index{scope\_couplings\_list (TiePieLCR\_settings.TiePieLCR\_settings attribute)@\spxentry{scope\_couplings\_list}\spxextra{TiePieLCR\_settings.TiePieLCR\_settings attribute}}

\begin{fulllineitems}
\phantomsection\label{\detokenize{index:TiePieLCR_settings.TiePieLCR_settings.scope_couplings_list}}\pysigline{\sphinxbfcode{\sphinxupquote{scope\_couplings\_list}}\sphinxbfcode{\sphinxupquote{\DUrole{w}{  }\DUrole{p}{=}\DUrole{w}{  }{[}2, 1{]}}}}
\end{fulllineitems}

\index{scope\_range\_list (TiePieLCR\_settings.TiePieLCR\_settings attribute)@\spxentry{scope\_range\_list}\spxextra{TiePieLCR\_settings.TiePieLCR\_settings attribute}}

\begin{fulllineitems}
\phantomsection\label{\detokenize{index:TiePieLCR_settings.TiePieLCR_settings.scope_range_list}}\pysigline{\sphinxbfcode{\sphinxupquote{scope\_range\_list}}\sphinxbfcode{\sphinxupquote{\DUrole{w}{  }\DUrole{p}{=}\DUrole{w}{  }{[}0.2, 0.4, 0.8, 2, 4{]}}}}
\end{fulllineitems}

\index{scope\_range\_name\_list (TiePieLCR\_settings.TiePieLCR\_settings attribute)@\spxentry{scope\_range\_name\_list}\spxextra{TiePieLCR\_settings.TiePieLCR\_settings attribute}}

\begin{fulllineitems}
\phantomsection\label{\detokenize{index:TiePieLCR_settings.TiePieLCR_settings.scope_range_name_list}}\pysigline{\sphinxbfcode{\sphinxupquote{scope\_range\_name\_list}}\sphinxbfcode{\sphinxupquote{\DUrole{w}{  }\DUrole{p}{=}\DUrole{w}{  }{[}\textquotesingle{}0.2\textquotesingle{}, \textquotesingle{}0.4\textquotesingle{}, \textquotesingle{}0.8\textquotesingle{}, \textquotesingle{}2\textquotesingle{}, \textquotesingle{}4\textquotesingle{}{]}}}}
\end{fulllineitems}

\index{scope\_ranges (TiePieLCR\_settings.TiePieLCR\_settings attribute)@\spxentry{scope\_ranges}\spxextra{TiePieLCR\_settings.TiePieLCR\_settings attribute}}

\begin{fulllineitems}
\phantomsection\label{\detokenize{index:TiePieLCR_settings.TiePieLCR_settings.scope_ranges}}\pysigline{\sphinxbfcode{\sphinxupquote{scope\_ranges}}\sphinxbfcode{\sphinxupquote{\DUrole{w}{  }\DUrole{p}{=}\DUrole{w}{  }{[}2, 2{]}}}}
\end{fulllineitems}

\index{set\_LCR\_gain() (TiePieLCR\_settings.TiePieLCR\_settings method)@\spxentry{set\_LCR\_gain()}\spxextra{TiePieLCR\_settings.TiePieLCR\_settings method}}

\begin{fulllineitems}
\phantomsection\label{\detokenize{index:TiePieLCR_settings.TiePieLCR_settings.set_LCR_gain}}\pysiglinewithargsret{\sphinxbfcode{\sphinxupquote{set\_LCR\_gain}}}{\emph{\DUrole{n}{new\_gain}}}{}
\sphinxAtStartPar
Set the instrumentation amplifier gain to one of the options in the list of possible gains. The list of possible gain can be obtained using {\hyperref[\detokenize{index:TiePieLCR_settings.TiePieLCR_settings.get_gain_name_list}]{\sphinxcrossref{\sphinxcode{\sphinxupquote{TiePieLCR\_settings.TiePieLCR\_settings.get\_gain\_name\_list()}}}}}
\begin{quote}\begin{description}
\item[{Parameters}] \leavevmode
\sphinxAtStartPar
\sphinxstyleliteralstrong{\sphinxupquote{new\_gain}} (\sphinxstyleliteralemphasis{\sphinxupquote{Int}}) \textendash{} The index in the list

\item[{Returns}] \leavevmode
\sphinxAtStartPar
Nothing

\item[{Return type}] \leavevmode
\sphinxAtStartPar
None

\end{description}\end{quote}

\end{fulllineitems}

\index{set\_amplitude() (TiePieLCR\_settings.TiePieLCR\_settings method)@\spxentry{set\_amplitude()}\spxextra{TiePieLCR\_settings.TiePieLCR\_settings method}}

\begin{fulllineitems}
\phantomsection\label{\detokenize{index:TiePieLCR_settings.TiePieLCR_settings.set_amplitude}}\pysiglinewithargsret{\sphinxbfcode{\sphinxupquote{set\_amplitude}}}{\emph{\DUrole{n}{amplitude}}}{}
\sphinxAtStartPar
Sets the maximum amplitude of the multisine that is used as an excitation signal.
\begin{quote}\begin{description}
\item[{Parameters}] \leavevmode
\sphinxAtStartPar
\sphinxstyleliteralstrong{\sphinxupquote{amplitude}} (\sphinxstyleliteralemphasis{\sphinxupquote{Float}}) \textendash{} The maximum amplitude

\item[{Returns}] \leavevmode
\sphinxAtStartPar
False if the total excitation signals get’s too large, True otherwise

\item[{Return type}] \leavevmode
\sphinxAtStartPar
Boolean

\end{description}\end{quote}

\end{fulllineitems}

\index{set\_demodulation\_bandwidth() (TiePieLCR\_settings.TiePieLCR\_settings method)@\spxentry{set\_demodulation\_bandwidth()}\spxextra{TiePieLCR\_settings.TiePieLCR\_settings method}}

\begin{fulllineitems}
\phantomsection\label{\detokenize{index:TiePieLCR_settings.TiePieLCR_settings.set_demodulation_bandwidth}}\pysiglinewithargsret{\sphinxbfcode{\sphinxupquote{set\_demodulation\_bandwidth}}}{\emph{\DUrole{n}{bandwidth}}}{}
\sphinxAtStartPar
Set the bandwidth with which the current and the voltage are demodulated and how fast the bottom two graphs in the interface will respond.
\begin{quote}\begin{description}
\item[{Parameters}] \leavevmode
\sphinxAtStartPar
\sphinxstyleliteralstrong{\sphinxupquote{bandwidth}} (\sphinxstyleliteralemphasis{\sphinxupquote{Float}}) \textendash{} The bandwidth

\item[{Returns}] \leavevmode
\sphinxAtStartPar
A dictionary that contains the keyword ‘error’ with an error when an error occured and is empty otherwise

\item[{Return type}] \leavevmode
\sphinxAtStartPar
Dict

\end{description}\end{quote}

\end{fulllineitems}

\index{set\_demodulation\_params() (TiePieLCR\_settings.TiePieLCR\_settings method)@\spxentry{set\_demodulation\_params()}\spxextra{TiePieLCR\_settings.TiePieLCR\_settings method}}

\begin{fulllineitems}
\phantomsection\label{\detokenize{index:TiePieLCR_settings.TiePieLCR_settings.set_demodulation_params}}\pysiglinewithargsret{\sphinxbfcode{\sphinxupquote{set\_demodulation\_params}}}{\emph{\DUrole{n}{freq\_list}}, \emph{\DUrole{n}{tiepie\_list}}}{}
\end{fulllineitems}

\index{set\_impedance\_format() (TiePieLCR\_settings.TiePieLCR\_settings method)@\spxentry{set\_impedance\_format()}\spxextra{TiePieLCR\_settings.TiePieLCR\_settings method}}

\begin{fulllineitems}
\phantomsection\label{\detokenize{index:TiePieLCR_settings.TiePieLCR_settings.set_impedance_format}}\pysiglinewithargsret{\sphinxbfcode{\sphinxupquote{set\_impedance\_format}}}{\emph{\DUrole{n}{value}}}{}
\sphinxAtStartPar
Set the complex impedance measured by the LCR can be represented in different ways:
\begin{itemize}
\item {} 
\sphinxAtStartPar
XY: As a complex and an imaginary part

\item {} 
\sphinxAtStartPar
RpCp: As a capacitor and a resistor in parallel

\item {} 
\sphinxAtStartPar
RsCs: As a capacitor and a reisistor in series

\item {} 
\sphinxAtStartPar
ZPhi: As an absolute value and a phase

\end{itemize}
\begin{quote}\begin{description}
\item[{Parameters}] \leavevmode
\sphinxAtStartPar
\sphinxstyleliteralstrong{\sphinxupquote{value}} (\sphinxstyleliteralemphasis{\sphinxupquote{String}}) \textendash{} The used format as a string i.e. ‘XY’ or ‘RpCp’

\item[{Returns}] \leavevmode
\sphinxAtStartPar
Nothing

\item[{Return type}] \leavevmode
\sphinxAtStartPar
None

\end{description}\end{quote}

\end{fulllineitems}

\index{set\_integration\_time() (TiePieLCR\_settings.TiePieLCR\_settings method)@\spxentry{set\_integration\_time()}\spxextra{TiePieLCR\_settings.TiePieLCR\_settings method}}

\begin{fulllineitems}
\phantomsection\label{\detokenize{index:TiePieLCR_settings.TiePieLCR_settings.set_integration_time}}\pysiglinewithargsret{\sphinxbfcode{\sphinxupquote{set\_integration\_time}}}{\emph{\DUrole{n}{integration\_time}}}{}
\sphinxAtStartPar
Set the time over which the demodulation signals will be avaraged to compute the offset displayed in the interface and the offset obtain using \sphinxcode{\sphinxupquote{TiePieLCR\_api.TiePieLCR\_api.get\_impedance()}}.
\begin{quote}\begin{description}
\item[{Parameters}] \leavevmode
\sphinxAtStartPar
\sphinxstyleliteralstrong{\sphinxupquote{integration\_time}} (\sphinxstyleliteralemphasis{\sphinxupquote{Float}}) \textendash{} The amount of time

\item[{Returns}] \leavevmode
\sphinxAtStartPar
Nothing

\item[{Return type}] \leavevmode
\sphinxAtStartPar
None

\end{description}\end{quote}

\end{fulllineitems}

\index{set\_maximum\_plot\_frequency() (TiePieLCR\_settings.TiePieLCR\_settings method)@\spxentry{set\_maximum\_plot\_frequency()}\spxextra{TiePieLCR\_settings.TiePieLCR\_settings method}}

\begin{fulllineitems}
\phantomsection\label{\detokenize{index:TiePieLCR_settings.TiePieLCR_settings.set_maximum_plot_frequency}}\pysiglinewithargsret{\sphinxbfcode{\sphinxupquote{set\_maximum\_plot\_frequency}}}{\emph{\DUrole{n}{frequency}}}{}
\sphinxAtStartPar
Set the minimum frequency that is shown in the frequency plots on the top right.
\begin{quote}\begin{description}
\item[{Returns}] \leavevmode
\sphinxAtStartPar
True if larger as the current minimum frequency, False otherwise.

\item[{Return type}] \leavevmode
\sphinxAtStartPar
Boolean

\end{description}\end{quote}

\end{fulllineitems}

\index{set\_minimum\_plot\_frequency() (TiePieLCR\_settings.TiePieLCR\_settings method)@\spxentry{set\_minimum\_plot\_frequency()}\spxextra{TiePieLCR\_settings.TiePieLCR\_settings method}}

\begin{fulllineitems}
\phantomsection\label{\detokenize{index:TiePieLCR_settings.TiePieLCR_settings.set_minimum_plot_frequency}}\pysiglinewithargsret{\sphinxbfcode{\sphinxupquote{set\_minimum\_plot\_frequency}}}{\emph{\DUrole{n}{frequency}}}{}
\sphinxAtStartPar
Set the minimum frequency that is shown in the frequency plots on the top right.
\begin{quote}\begin{description}
\item[{Returns}] \leavevmode
\sphinxAtStartPar
True if smaller as the current maximum frequency, False otherwise.

\item[{Return type}] \leavevmode
\sphinxAtStartPar
Boolean

\end{description}\end{quote}

\end{fulllineitems}

\index{set\_multisine() (TiePieLCR\_settings.TiePieLCR\_settings method)@\spxentry{set\_multisine()}\spxextra{TiePieLCR\_settings.TiePieLCR\_settings method}}

\begin{fulllineitems}
\phantomsection\label{\detokenize{index:TiePieLCR_settings.TiePieLCR_settings.set_multisine}}\pysiglinewithargsret{\sphinxbfcode{\sphinxupquote{set\_multisine}}}{\emph{\DUrole{n}{freq\_list}}, \emph{\DUrole{n}{weight\_list}}}{}
\sphinxAtStartPar
Sets the frequencies and their weights in the multisine that is used as an excitation signal. The weights are defined relative to the maximum amplitude of the signal.
\begin{quote}\begin{description}
\item[{Parameters}] \leavevmode\begin{itemize}
\item {} 
\sphinxAtStartPar
\sphinxstyleliteralstrong{\sphinxupquote{freq\_list}} (\sphinxstyleliteralemphasis{\sphinxupquote{List of floats}}) \textendash{} A list of frequencies to be used in the multisine

\item {} 
\sphinxAtStartPar
\sphinxstyleliteralstrong{\sphinxupquote{weight\_list}} (\sphinxstyleliteralemphasis{\sphinxupquote{List of weights}}) \textendash{} A list of weight to be used in the multisine. The weights are relative to the maximum amplitude of the signal as defined by {\hyperref[\detokenize{index:TiePieLCR_settings.TiePieLCR_settings.set_amplitude}]{\sphinxcrossref{\sphinxcode{\sphinxupquote{TiePieLCR\_settings.TiePieLCR\_settings.set\_amplitude()}}}}}.

\end{itemize}

\item[{Returns}] \leavevmode
\sphinxAtStartPar
The sample frequency

\item[{Return type}] \leavevmode
\sphinxAtStartPar
Float

\end{description}\end{quote}

\end{fulllineitems}

\index{set\_offset() (TiePieLCR\_settings.TiePieLCR\_settings method)@\spxentry{set\_offset()}\spxextra{TiePieLCR\_settings.TiePieLCR\_settings method}}

\begin{fulllineitems}
\phantomsection\label{\detokenize{index:TiePieLCR_settings.TiePieLCR_settings.set_offset}}\pysiglinewithargsret{\sphinxbfcode{\sphinxupquote{set\_offset}}}{\emph{\DUrole{n}{offset}}}{}
\sphinxAtStartPar
Sets the offset of the multisine that is used as an excitation signal.
\begin{quote}\begin{description}
\item[{Parameters}] \leavevmode
\sphinxAtStartPar
\sphinxstyleliteralstrong{\sphinxupquote{amplitude}} (\sphinxstyleliteralemphasis{\sphinxupquote{Float}}) \textendash{} The offset

\item[{Returns}] \leavevmode
\sphinxAtStartPar
False if the total excitation signals get’s too large, True otherwise

\item[{Return type}] \leavevmode
\sphinxAtStartPar
Boolean

\end{description}\end{quote}

\end{fulllineitems}

\index{set\_offset\_bandwidth() (TiePieLCR\_settings.TiePieLCR\_settings method)@\spxentry{set\_offset\_bandwidth()}\spxextra{TiePieLCR\_settings.TiePieLCR\_settings method}}

\begin{fulllineitems}
\phantomsection\label{\detokenize{index:TiePieLCR_settings.TiePieLCR_settings.set_offset_bandwidth}}\pysiglinewithargsret{\sphinxbfcode{\sphinxupquote{set\_offset\_bandwidth}}}{\emph{\DUrole{n}{new\_offset\_bandwidth}}}{}
\sphinxAtStartPar
Set the bandwidth with which the offset is calculated and what will be the bandwidth of the offset signals in the mat file.
\begin{quote}\begin{description}
\item[{Parameters}] \leavevmode
\sphinxAtStartPar
\sphinxstyleliteralstrong{\sphinxupquote{bandwidth}} (\sphinxstyleliteralemphasis{\sphinxupquote{Float}}) \textendash{} The bandwidth

\item[{Returns}] \leavevmode
\sphinxAtStartPar
Nothing

\item[{Return type}] \leavevmode
\sphinxAtStartPar
None

\end{description}\end{quote}

\end{fulllineitems}

\index{set\_offset\_integration\_time() (TiePieLCR\_settings.TiePieLCR\_settings method)@\spxentry{set\_offset\_integration\_time()}\spxextra{TiePieLCR\_settings.TiePieLCR\_settings method}}

\begin{fulllineitems}
\phantomsection\label{\detokenize{index:TiePieLCR_settings.TiePieLCR_settings.set_offset_integration_time}}\pysiglinewithargsret{\sphinxbfcode{\sphinxupquote{set\_offset\_integration\_time}}}{\emph{\DUrole{n}{new\_offset\_integration}}}{}
\sphinxAtStartPar
Over how much time the the offset signals are averaged to come to the displayed impedances.
\begin{quote}\begin{description}
\item[{Parameters}] \leavevmode
\sphinxAtStartPar
\sphinxstyleliteralstrong{\sphinxupquote{bandwidth}} (\sphinxstyleliteralemphasis{\sphinxupquote{Float}}) \textendash{} The amount of time

\item[{Returns}] \leavevmode
\sphinxAtStartPar
Nothing

\item[{Return type}] \leavevmode
\sphinxAtStartPar
None

\end{description}\end{quote}

\end{fulllineitems}

\index{set\_plot\_periods() (TiePieLCR\_settings.TiePieLCR\_settings method)@\spxentry{set\_plot\_periods()}\spxextra{TiePieLCR\_settings.TiePieLCR\_settings method}}

\begin{fulllineitems}
\phantomsection\label{\detokenize{index:TiePieLCR_settings.TiePieLCR_settings.set_plot_periods}}\pysiglinewithargsret{\sphinxbfcode{\sphinxupquote{set\_plot\_periods}}}{\emph{\DUrole{n}{periods}}}{}
\sphinxAtStartPar
Set the number of repetitions that are shown in the time plot in the left top of the interface.
\begin{quote}\begin{description}
\item[{Parameters}] \leavevmode
\sphinxAtStartPar
\sphinxstyleliteralstrong{\sphinxupquote{periods}} (\sphinxstyleliteralemphasis{\sphinxupquote{float}}) \textendash{} The number of periods

\item[{Returns}] \leavevmode
\sphinxAtStartPar
Nothing

\item[{Return type}] \leavevmode
\sphinxAtStartPar
None

\end{description}\end{quote}

\end{fulllineitems}

\index{set\_plot\_time() (TiePieLCR\_settings.TiePieLCR\_settings method)@\spxentry{set\_plot\_time()}\spxextra{TiePieLCR\_settings.TiePieLCR\_settings method}}

\begin{fulllineitems}
\phantomsection\label{\detokenize{index:TiePieLCR_settings.TiePieLCR_settings.set_plot_time}}\pysiglinewithargsret{\sphinxbfcode{\sphinxupquote{set\_plot\_time}}}{\emph{\DUrole{n}{plot\_time}}}{}
\sphinxAtStartPar
Set the amount of time shown on the x\sphinxhyphen{}axis of the bottom two graphs.
\begin{quote}\begin{description}
\item[{Parameters}] \leavevmode
\sphinxAtStartPar
\sphinxstyleliteralstrong{\sphinxupquote{plot\_time}} (\sphinxstyleliteralemphasis{\sphinxupquote{Float}}) \textendash{} The amount of time

\item[{Returns}] \leavevmode
\sphinxAtStartPar
Nothing

\item[{Return type}] \leavevmode
\sphinxAtStartPar
None

\end{description}\end{quote}

\end{fulllineitems}

\index{set\_reference() (TiePieLCR\_settings.TiePieLCR\_settings method)@\spxentry{set\_reference()}\spxextra{TiePieLCR\_settings.TiePieLCR\_settings method}}

\begin{fulllineitems}
\phantomsection\label{\detokenize{index:TiePieLCR_settings.TiePieLCR_settings.set_reference}}\pysiglinewithargsret{\sphinxbfcode{\sphinxupquote{set\_reference}}}{\emph{\DUrole{n}{new\_reference}}}{}
\sphinxAtStartPar
Set the transimpedance amplifier to one of the options in the list of possible options. The list of possible transimpedance amplifiers can be obtained using {\hyperref[\detokenize{index:TiePieLCR_settings.TiePieLCR_settings.get_reference_name_list}]{\sphinxcrossref{\sphinxcode{\sphinxupquote{TiePieLCR\_settings.TiePieLCR\_settings.get\_reference\_name\_list()}}}}}
\begin{quote}\begin{description}
\item[{Parameters}] \leavevmode
\sphinxAtStartPar
\sphinxstyleliteralstrong{\sphinxupquote{new\_gain}} (\sphinxstyleliteralemphasis{\sphinxupquote{Int}}) \textendash{} The index in the list

\item[{Returns}] \leavevmode
\sphinxAtStartPar
Nothing

\item[{Return type}] \leavevmode
\sphinxAtStartPar
None

\end{description}\end{quote}

\end{fulllineitems}

\index{set\_reference\_scope\_coupling() (TiePieLCR\_settings.TiePieLCR\_settings method)@\spxentry{set\_reference\_scope\_coupling()}\spxextra{TiePieLCR\_settings.TiePieLCR\_settings method}}

\begin{fulllineitems}
\phantomsection\label{\detokenize{index:TiePieLCR_settings.TiePieLCR_settings.set_reference_scope_coupling}}\pysiglinewithargsret{\sphinxbfcode{\sphinxupquote{set\_reference\_scope\_coupling}}}{\emph{\DUrole{n}{coupling}}}{}
\sphinxAtStartPar
Set the index of the coupling of the channel of the scope that is used to measure the reference
\begin{quote}\begin{description}
\item[{Parameters}] \leavevmode
\sphinxAtStartPar
\sphinxstyleliteralstrong{\sphinxupquote{coupling}} (\sphinxstyleliteralemphasis{\sphinxupquote{Int}}) \textendash{} Index in the list returned by {\hyperref[\detokenize{index:TiePieLCR_settings.TiePieLCR_settings.get_scope_coupling_name_list}]{\sphinxcrossref{\sphinxcode{\sphinxupquote{TiePieLCR\_settings.TiePieLCR\_settings.get\_scope\_coupling\_name\_list()}}}}}

\item[{Returns}] \leavevmode
\sphinxAtStartPar
Nothing

\item[{Return type}] \leavevmode
\sphinxAtStartPar
none

\end{description}\end{quote}

\end{fulllineitems}

\index{set\_reference\_scope\_range() (TiePieLCR\_settings.TiePieLCR\_settings method)@\spxentry{set\_reference\_scope\_range()}\spxextra{TiePieLCR\_settings.TiePieLCR\_settings method}}

\begin{fulllineitems}
\phantomsection\label{\detokenize{index:TiePieLCR_settings.TiePieLCR_settings.set_reference_scope_range}}\pysiglinewithargsret{\sphinxbfcode{\sphinxupquote{set\_reference\_scope\_range}}}{\emph{\DUrole{n}{range}}}{}
\sphinxAtStartPar
Set which element of the scope range list is selected for the reference channel. The scope range list can be obtained using {\hyperref[\detokenize{index:TiePieLCR_settings.TiePieLCR_settings.get_scope_range_list}]{\sphinxcrossref{\sphinxcode{\sphinxupquote{TiePieLCR\_settings.TiePieLCR\_settings.get\_scope\_range\_list()}}}}}.
\begin{quote}\begin{description}
\item[{Returns}] \leavevmode
\sphinxAtStartPar
Nothing

\item[{Return type}] \leavevmode
\sphinxAtStartPar
None

\end{description}\end{quote}

\end{fulllineitems}

\index{set\_sample\_frequency() (TiePieLCR\_settings.TiePieLCR\_settings method)@\spxentry{set\_sample\_frequency()}\spxextra{TiePieLCR\_settings.TiePieLCR\_settings method}}

\begin{fulllineitems}
\phantomsection\label{\detokenize{index:TiePieLCR_settings.TiePieLCR_settings.set_sample_frequency}}\pysiglinewithargsret{\sphinxbfcode{\sphinxupquote{set\_sample\_frequency}}}{\emph{\DUrole{n}{value}}}{}
\sphinxAtStartPar
Set the sampling frequency of the scope. A higher value results in less noise, however it might be reduced if the LCR stop and gives an error complaining that your PC is not fast enough.

\sphinxAtStartPar
The sampling frequency should be smaller as 6250000 Hz and be an integer multiple of the number of sub blocks.
\begin{quote}\begin{description}
\item[{Returns}] \leavevmode
\sphinxAtStartPar
A dictionary that contains the keyword ‘error’ when an error occured and is empty otherwise

\item[{Return type}] \leavevmode
\sphinxAtStartPar
Dict

\end{description}\end{quote}

\end{fulllineitems}

\index{set\_signal\_scope\_coupling() (TiePieLCR\_settings.TiePieLCR\_settings method)@\spxentry{set\_signal\_scope\_coupling()}\spxextra{TiePieLCR\_settings.TiePieLCR\_settings method}}

\begin{fulllineitems}
\phantomsection\label{\detokenize{index:TiePieLCR_settings.TiePieLCR_settings.set_signal_scope_coupling}}\pysiglinewithargsret{\sphinxbfcode{\sphinxupquote{set\_signal\_scope\_coupling}}}{\emph{\DUrole{n}{coupling}}}{}
\sphinxAtStartPar
Set the index of the coupling of the channel of the scope that is used to measure the signal
\begin{quote}\begin{description}
\item[{Parameters}] \leavevmode
\sphinxAtStartPar
\sphinxstyleliteralstrong{\sphinxupquote{coupling}} (\sphinxstyleliteralemphasis{\sphinxupquote{Int}}) \textendash{} Index in the list returned by {\hyperref[\detokenize{index:TiePieLCR_settings.TiePieLCR_settings.get_scope_coupling_name_list}]{\sphinxcrossref{\sphinxcode{\sphinxupquote{TiePieLCR\_settings.TiePieLCR\_settings.get\_scope\_coupling\_name\_list()}}}}}

\item[{Returns}] \leavevmode
\sphinxAtStartPar
Nothing

\item[{Return type}] \leavevmode
\sphinxAtStartPar
none

\end{description}\end{quote}

\end{fulllineitems}

\index{set\_signal\_scope\_range() (TiePieLCR\_settings.TiePieLCR\_settings method)@\spxentry{set\_signal\_scope\_range()}\spxextra{TiePieLCR\_settings.TiePieLCR\_settings method}}

\begin{fulllineitems}
\phantomsection\label{\detokenize{index:TiePieLCR_settings.TiePieLCR_settings.set_signal_scope_range}}\pysiglinewithargsret{\sphinxbfcode{\sphinxupquote{set\_signal\_scope\_range}}}{\emph{\DUrole{n}{range}}}{}
\sphinxAtStartPar
Set which element of the scope range list is selected for the signal channel. The scope range list can be obtained using {\hyperref[\detokenize{index:TiePieLCR_settings.TiePieLCR_settings.get_scope_range_list}]{\sphinxcrossref{\sphinxcode{\sphinxupquote{TiePieLCR\_settings.TiePieLCR\_settings.get\_scope\_range\_list()}}}}}.
\begin{quote}\begin{description}
\item[{Returns}] \leavevmode
\sphinxAtStartPar
Nothing

\item[{Return type}] \leavevmode
\sphinxAtStartPar
None

\end{description}\end{quote}

\end{fulllineitems}

\index{set\_sub\_blocks() (TiePieLCR\_settings.TiePieLCR\_settings method)@\spxentry{set\_sub\_blocks()}\spxextra{TiePieLCR\_settings.TiePieLCR\_settings method}}

\begin{fulllineitems}
\phantomsection\label{\detokenize{index:TiePieLCR_settings.TiePieLCR_settings.set_sub_blocks}}\pysiglinewithargsret{\sphinxbfcode{\sphinxupquote{set\_sub\_blocks}}}{\emph{\DUrole{n}{sub\_block\_index}}}{}
\sphinxAtStartPar
Sets in how many blocks the data retrieved from the scope is split up before doing the fft. Since the time it takes to do an fft increases with a power of 1.4 with the amount of points, a higher number of sub\sphinxhyphen{}blocks will reduce the computation load. A higher number sub\sphinxhyphen{}blocks also moves the frequency at which noise at the edge of each block will appear. A higher number of sub\sphinxhyphen{}blocks makes very lower frequency measurements more computationally intensive though.
\begin{quote}\begin{description}
\item[{Parameters}] \leavevmode
\sphinxAtStartPar
\sphinxstyleliteralstrong{\sphinxupquote{sub\_block\_index}} (\sphinxstyleliteralemphasis{\sphinxupquote{Int}}) \textendash{} Number in the list obtained through {\hyperref[\detokenize{index:TiePieLCR_settings.TiePieLCR_settings.get_sub_blocks_list}]{\sphinxcrossref{\sphinxcode{\sphinxupquote{TiePieLCR\_settings.TiePieLCR\_settings.get\_sub\_blocks\_list()}}}}} that should be used

\item[{Returns}] \leavevmode
\sphinxAtStartPar
A dictionary that contains the keyword ‘error’ with an error when an error occured and is empty otherwise

\item[{Return type}] \leavevmode
\sphinxAtStartPar
Dict

\end{description}\end{quote}

\end{fulllineitems}

\index{set\_update\_freq() (TiePieLCR\_settings.TiePieLCR\_settings method)@\spxentry{set\_update\_freq()}\spxextra{TiePieLCR\_settings.TiePieLCR\_settings method}}

\begin{fulllineitems}
\phantomsection\label{\detokenize{index:TiePieLCR_settings.TiePieLCR_settings.set_update_freq}}\pysiglinewithargsret{\sphinxbfcode{\sphinxupquote{set\_update\_freq}}}{\emph{\DUrole{n}{update\_freq\_index}}}{}
\sphinxAtStartPar
Set how often data is retrieved from the scope, the gui is updated and how often the value obtained using \sphinxcode{\sphinxupquote{TiePieLCR\_api.TiePieLCR\_api.get\_impedance()}} is updated. Only reduce this if your computer is really really slow..

\sphinxAtStartPar
The update sample frequency divided by the update frequency should be an integer multiple of the number of sub blocks.
\begin{quote}\begin{description}
\item[{Parameters}] \leavevmode
\sphinxAtStartPar
\sphinxstyleliteralstrong{\sphinxupquote{update\_freq\_index}} (\sphinxstyleliteralemphasis{\sphinxupquote{Int}}) \textendash{} Frequency in the list obtained through \sphinxcode{\sphinxupquote{TiePieLCR\_settings.TiePieLCR\_settings.get\_update\_freq\_list()}} that should be used

\item[{Returns}] \leavevmode
\sphinxAtStartPar
A dictionary that contains the keyword ‘error’ with an error when an error occured and is empty otherwise

\item[{Return type}] \leavevmode
\sphinxAtStartPar
Dict

\end{description}\end{quote}

\end{fulllineitems}

\index{settings\_dict (TiePieLCR\_settings.TiePieLCR\_settings attribute)@\spxentry{settings\_dict}\spxextra{TiePieLCR\_settings.TiePieLCR\_settings attribute}}

\begin{fulllineitems}
\phantomsection\label{\detokenize{index:TiePieLCR_settings.TiePieLCR_settings.settings_dict}}\pysigline{\sphinxbfcode{\sphinxupquote{settings\_dict}}\sphinxbfcode{\sphinxupquote{\DUrole{w}{  }\DUrole{p}{=}\DUrole{w}{  }\{\}}}}
\end{fulllineitems}

\index{side\_lob\_n (TiePieLCR\_settings.TiePieLCR\_settings attribute)@\spxentry{side\_lob\_n}\spxextra{TiePieLCR\_settings.TiePieLCR\_settings attribute}}

\begin{fulllineitems}
\phantomsection\label{\detokenize{index:TiePieLCR_settings.TiePieLCR_settings.side_lob_n}}\pysigline{\sphinxbfcode{\sphinxupquote{side\_lob\_n}}\sphinxbfcode{\sphinxupquote{\DUrole{w}{  }\DUrole{p}{=}\DUrole{w}{  }7}}}
\end{fulllineitems}

\index{sub\_blocks\_list (TiePieLCR\_settings.TiePieLCR\_settings attribute)@\spxentry{sub\_blocks\_list}\spxextra{TiePieLCR\_settings.TiePieLCR\_settings attribute}}

\begin{fulllineitems}
\phantomsection\label{\detokenize{index:TiePieLCR_settings.TiePieLCR_settings.sub_blocks_list}}\pysigline{\sphinxbfcode{\sphinxupquote{sub\_blocks\_list}}\sphinxbfcode{\sphinxupquote{\DUrole{w}{  }\DUrole{p}{=}\DUrole{w}{  }{[}1, 2, 5, 10, 20, 50{]}}}}
\end{fulllineitems}

\index{sub\_blocks\_name\_list (TiePieLCR\_settings.TiePieLCR\_settings attribute)@\spxentry{sub\_blocks\_name\_list}\spxextra{TiePieLCR\_settings.TiePieLCR\_settings attribute}}

\begin{fulllineitems}
\phantomsection\label{\detokenize{index:TiePieLCR_settings.TiePieLCR_settings.sub_blocks_name_list}}\pysigline{\sphinxbfcode{\sphinxupquote{sub\_blocks\_name\_list}}\sphinxbfcode{\sphinxupquote{\DUrole{w}{  }\DUrole{p}{=}\DUrole{w}{  }{[}\textquotesingle{}1\textquotesingle{}, \textquotesingle{}2\textquotesingle{}, \textquotesingle{}5\textquotesingle{}, \textquotesingle{}10\textquotesingle{}, \textquotesingle{}20\textquotesingle{}, \textquotesingle{}50\textquotesingle{}{]}}}}
\end{fulllineitems}

\index{sub\_blocks\_setting (TiePieLCR\_settings.TiePieLCR\_settings attribute)@\spxentry{sub\_blocks\_setting}\spxextra{TiePieLCR\_settings.TiePieLCR\_settings attribute}}

\begin{fulllineitems}
\phantomsection\label{\detokenize{index:TiePieLCR_settings.TiePieLCR_settings.sub_blocks_setting}}\pysigline{\sphinxbfcode{\sphinxupquote{sub\_blocks\_setting}}\sphinxbfcode{\sphinxupquote{\DUrole{w}{  }\DUrole{p}{=}\DUrole{w}{  }1}}}
\end{fulllineitems}

\index{update\_freq\_list (TiePieLCR\_settings.TiePieLCR\_settings attribute)@\spxentry{update\_freq\_list}\spxextra{TiePieLCR\_settings.TiePieLCR\_settings attribute}}

\begin{fulllineitems}
\phantomsection\label{\detokenize{index:TiePieLCR_settings.TiePieLCR_settings.update_freq_list}}\pysigline{\sphinxbfcode{\sphinxupquote{update\_freq\_list}}\sphinxbfcode{\sphinxupquote{\DUrole{w}{  }\DUrole{p}{=}\DUrole{w}{  }{[}1, 2, 5, 10, 25{]}}}}
\end{fulllineitems}

\index{update\_freq\_name\_list (TiePieLCR\_settings.TiePieLCR\_settings attribute)@\spxentry{update\_freq\_name\_list}\spxextra{TiePieLCR\_settings.TiePieLCR\_settings attribute}}

\begin{fulllineitems}
\phantomsection\label{\detokenize{index:TiePieLCR_settings.TiePieLCR_settings.update_freq_name_list}}\pysigline{\sphinxbfcode{\sphinxupquote{update\_freq\_name\_list}}\sphinxbfcode{\sphinxupquote{\DUrole{w}{  }\DUrole{p}{=}\DUrole{w}{  }{[}\textquotesingle{}1 Hz\textquotesingle{}, \textquotesingle{}2 Hz\textquotesingle{}, \textquotesingle{}5 Hz\textquotesingle{}, \textquotesingle{}10 Hz\textquotesingle{}, \textquotesingle{}25 Hz\textquotesingle{}{]}}}}
\end{fulllineitems}

\index{update\_freq\_setting (TiePieLCR\_settings.TiePieLCR\_settings attribute)@\spxentry{update\_freq\_setting}\spxextra{TiePieLCR\_settings.TiePieLCR\_settings attribute}}

\begin{fulllineitems}
\phantomsection\label{\detokenize{index:TiePieLCR_settings.TiePieLCR_settings.update_freq_setting}}\pysigline{\sphinxbfcode{\sphinxupquote{update\_freq\_setting}}\sphinxbfcode{\sphinxupquote{\DUrole{w}{  }\DUrole{p}{=}\DUrole{w}{  }3}}}
\end{fulllineitems}

\index{version (TiePieLCR\_settings.TiePieLCR\_settings attribute)@\spxentry{version}\spxextra{TiePieLCR\_settings.TiePieLCR\_settings attribute}}

\begin{fulllineitems}
\phantomsection\label{\detokenize{index:TiePieLCR_settings.TiePieLCR_settings.version}}\pysigline{\sphinxbfcode{\sphinxupquote{version}}\sphinxbfcode{\sphinxupquote{\DUrole{w}{  }\DUrole{p}{=}\DUrole{w}{  }\textquotesingle{}V1.0.0\textquotesingle{}}}}
\end{fulllineitems}


\end{fulllineitems}



\chapter{impedance class}
\label{\detokenize{index:module-impedance}}\label{\detokenize{index:impedance-class}}\index{module@\spxentry{module}!impedance@\spxentry{impedance}}\index{impedance@\spxentry{impedance}!module@\spxentry{module}}\phantomsection\label{\detokenize{index:module-4}}\index{module@\spxentry{module}!impedance@\spxentry{impedance}}\index{impedance@\spxentry{impedance}!module@\spxentry{module}}\index{impedance (class in impedance)@\spxentry{impedance}\spxextra{class in impedance}}

\begin{fulllineitems}
\phantomsection\label{\detokenize{index:impedance.impedance}}\pysigline{\sphinxbfcode{\sphinxupquote{class\DUrole{w}{  }}}\sphinxcode{\sphinxupquote{impedance.}}\sphinxbfcode{\sphinxupquote{impedance}}}
\sphinxAtStartPar
Bases: \sphinxcode{\sphinxupquote{object}}
\index{error1 (impedance.impedance attribute)@\spxentry{error1}\spxextra{impedance.impedance attribute}}

\begin{fulllineitems}
\phantomsection\label{\detokenize{index:impedance.impedance.error1}}\pysigline{\sphinxbfcode{\sphinxupquote{error1}}\sphinxbfcode{\sphinxupquote{\DUrole{w}{  }\DUrole{p}{=}\DUrole{w}{  }0}}}
\end{fulllineitems}

\index{error2 (impedance.impedance attribute)@\spxentry{error2}\spxextra{impedance.impedance attribute}}

\begin{fulllineitems}
\phantomsection\label{\detokenize{index:impedance.impedance.error2}}\pysigline{\sphinxbfcode{\sphinxupquote{error2}}\sphinxbfcode{\sphinxupquote{\DUrole{w}{  }\DUrole{p}{=}\DUrole{w}{  }0}}}
\end{fulllineitems}

\index{ref\_clipped (impedance.impedance attribute)@\spxentry{ref\_clipped}\spxextra{impedance.impedance attribute}}

\begin{fulllineitems}
\phantomsection\label{\detokenize{index:impedance.impedance.ref_clipped}}\pysigline{\sphinxbfcode{\sphinxupquote{ref\_clipped}}\sphinxbfcode{\sphinxupquote{\DUrole{w}{  }\DUrole{p}{=}\DUrole{w}{  }False}}}
\end{fulllineitems}

\index{ref\_offset (impedance.impedance attribute)@\spxentry{ref\_offset}\spxextra{impedance.impedance attribute}}

\begin{fulllineitems}
\phantomsection\label{\detokenize{index:impedance.impedance.ref_offset}}\pysigline{\sphinxbfcode{\sphinxupquote{ref\_offset}}\sphinxbfcode{\sphinxupquote{\DUrole{w}{  }\DUrole{p}{=}\DUrole{w}{  }0}}}
\end{fulllineitems}

\index{set\_clipping() (impedance.impedance method)@\spxentry{set\_clipping()}\spxextra{impedance.impedance method}}

\begin{fulllineitems}
\phantomsection\label{\detokenize{index:impedance.impedance.set_clipping}}\pysiglinewithargsret{\sphinxbfcode{\sphinxupquote{set\_clipping}}}{\emph{\DUrole{n}{ref\_clipping}}, \emph{\DUrole{n}{sig\_clipping}}}{}
\sphinxAtStartPar
Set wether or not the reference and signal clipped (=reached the maximum voltage that can be measured) during the integration time.
\begin{quote}\begin{description}
\item[{Parameters}] \leavevmode\begin{itemize}
\item {} 
\sphinxAtStartPar
\sphinxstyleliteralstrong{\sphinxupquote{ref\_clipping}} (\sphinxstyleliteralemphasis{\sphinxupquote{Boolean}}) \textendash{} The reference clipped

\item {} 
\sphinxAtStartPar
\sphinxstyleliteralstrong{\sphinxupquote{sig\_clipping}} (\sphinxstyleliteralemphasis{\sphinxupquote{Booelan}}) \textendash{} The signal clipped

\end{itemize}

\item[{Returns}] \leavevmode
\sphinxAtStartPar
Nothing

\item[{Return type}] \leavevmode
\sphinxAtStartPar
None

\end{description}\end{quote}

\end{fulllineitems}

\index{set\_errors() (impedance.impedance method)@\spxentry{set\_errors()}\spxextra{impedance.impedance method}}

\begin{fulllineitems}
\phantomsection\label{\detokenize{index:impedance.impedance.set_errors}}\pysiglinewithargsret{\sphinxbfcode{\sphinxupquote{set\_errors}}}{\emph{\DUrole{n}{new\_error1}}, \emph{\DUrole{n}{new\_error2}}}{}
\sphinxAtStartPar
Set the impedance values of the object. What these values represents is determined by \sphinxcode{\sphinxupquote{tiepieLCR\_settings.tiepieLCR\_settings.get\_impedance\_format()}}
\begin{quote}\begin{description}
\item[{Parameters}] \leavevmode\begin{itemize}
\item {} 
\sphinxAtStartPar
\sphinxstyleliteralstrong{\sphinxupquote{new\_error1}} (\sphinxstyleliteralemphasis{\sphinxupquote{Complex double}}) \textendash{} The standard error in the first impedance value

\item {} 
\sphinxAtStartPar
\sphinxstyleliteralstrong{\sphinxupquote{new\_error2}} (\sphinxstyleliteralemphasis{\sphinxupquote{Complex double}}) \textendash{} The standard error in the second impedance value

\end{itemize}

\item[{Returns}] \leavevmode
\sphinxAtStartPar
Nothing

\item[{Return type}] \leavevmode
\sphinxAtStartPar
None

\end{description}\end{quote}

\end{fulllineitems}

\index{set\_offsets() (impedance.impedance method)@\spxentry{set\_offsets()}\spxextra{impedance.impedance method}}

\begin{fulllineitems}
\phantomsection\label{\detokenize{index:impedance.impedance.set_offsets}}\pysiglinewithargsret{\sphinxbfcode{\sphinxupquote{set\_offsets}}}{\emph{\DUrole{n}{new\_ref\_offset}}, \emph{\DUrole{n}{new\_sig\_offset}}}{}
\sphinxAtStartPar
Set the offset values of the object.
\begin{quote}\begin{description}
\item[{Parameters}] \leavevmode\begin{itemize}
\item {} 
\sphinxAtStartPar
\sphinxstyleliteralstrong{\sphinxupquote{new\_ref\_offset}} (\sphinxstyleliteralemphasis{\sphinxupquote{Double}}) \textendash{} The offset in the reference

\item {} 
\sphinxAtStartPar
\sphinxstyleliteralstrong{\sphinxupquote{new\_sig\_offset}} (\sphinxstyleliteralemphasis{\sphinxupquote{Double}}) \textendash{} The offset in the signal

\end{itemize}

\item[{Returns}] \leavevmode
\sphinxAtStartPar
Nothing

\item[{Return type}] \leavevmode
\sphinxAtStartPar
None

\end{description}\end{quote}

\end{fulllineitems}

\index{set\_timestamp() (impedance.impedance method)@\spxentry{set\_timestamp()}\spxextra{impedance.impedance method}}

\begin{fulllineitems}
\phantomsection\label{\detokenize{index:impedance.impedance.set_timestamp}}\pysiglinewithargsret{\sphinxbfcode{\sphinxupquote{set\_timestamp}}}{\emph{\DUrole{n}{stamp}}}{}
\sphinxAtStartPar
set the timestamp of the point in time the impedance was measured. This always is the timestamp of the last sample of the integration time.
\begin{quote}\begin{description}
\item[{Parameters}] \leavevmode
\sphinxAtStartPar
\sphinxstyleliteralstrong{\sphinxupquote{stamp}} (\sphinxstyleliteralemphasis{\sphinxupquote{Float}}) \textendash{} The timestamp

\item[{Returns}] \leavevmode
\sphinxAtStartPar
Nothing

\item[{Return type}] \leavevmode
\sphinxAtStartPar
None

\end{description}\end{quote}

\end{fulllineitems}

\index{set\_values() (impedance.impedance method)@\spxentry{set\_values()}\spxextra{impedance.impedance method}}

\begin{fulllineitems}
\phantomsection\label{\detokenize{index:impedance.impedance.set_values}}\pysiglinewithargsret{\sphinxbfcode{\sphinxupquote{set\_values}}}{\emph{\DUrole{n}{new\_value1}}, \emph{\DUrole{n}{new\_value2}}}{}
\sphinxAtStartPar
Set the impedance values of the object. What these values represents is determined by \sphinxcode{\sphinxupquote{tiepieLCR\_settings.tiepieLCR\_settings.get\_impedance\_format()}}
\begin{quote}\begin{description}
\item[{Parameters}] \leavevmode\begin{itemize}
\item {} 
\sphinxAtStartPar
\sphinxstyleliteralstrong{\sphinxupquote{new\_value1}} (\sphinxstyleliteralemphasis{\sphinxupquote{Complex double}}) \textendash{} The first impedance value.

\item {} 
\sphinxAtStartPar
\sphinxstyleliteralstrong{\sphinxupquote{new\_value2}} (\sphinxstyleliteralemphasis{\sphinxupquote{Complex double}}) \textendash{} The second impedance value.

\end{itemize}

\item[{Returns}] \leavevmode
\sphinxAtStartPar
Nothing

\item[{Return type}] \leavevmode
\sphinxAtStartPar
None

\end{description}\end{quote}

\end{fulllineitems}

\index{sig\_clipped (impedance.impedance attribute)@\spxentry{sig\_clipped}\spxextra{impedance.impedance attribute}}

\begin{fulllineitems}
\phantomsection\label{\detokenize{index:impedance.impedance.sig_clipped}}\pysigline{\sphinxbfcode{\sphinxupquote{sig\_clipped}}\sphinxbfcode{\sphinxupquote{\DUrole{w}{  }\DUrole{p}{=}\DUrole{w}{  }False}}}
\end{fulllineitems}

\index{sig\_offset (impedance.impedance attribute)@\spxentry{sig\_offset}\spxextra{impedance.impedance attribute}}

\begin{fulllineitems}
\phantomsection\label{\detokenize{index:impedance.impedance.sig_offset}}\pysigline{\sphinxbfcode{\sphinxupquote{sig\_offset}}\sphinxbfcode{\sphinxupquote{\DUrole{w}{  }\DUrole{p}{=}\DUrole{w}{  }0}}}
\end{fulllineitems}

\index{timestamp (impedance.impedance attribute)@\spxentry{timestamp}\spxextra{impedance.impedance attribute}}

\begin{fulllineitems}
\phantomsection\label{\detokenize{index:impedance.impedance.timestamp}}\pysigline{\sphinxbfcode{\sphinxupquote{timestamp}}\sphinxbfcode{\sphinxupquote{\DUrole{w}{  }\DUrole{p}{=}\DUrole{w}{  }0}}}
\end{fulllineitems}

\index{valid (impedance.impedance attribute)@\spxentry{valid}\spxextra{impedance.impedance attribute}}

\begin{fulllineitems}
\phantomsection\label{\detokenize{index:impedance.impedance.valid}}\pysigline{\sphinxbfcode{\sphinxupquote{valid}}\sphinxbfcode{\sphinxupquote{\DUrole{w}{  }\DUrole{p}{=}\DUrole{w}{  }False}}}
\end{fulllineitems}

\index{value1 (impedance.impedance attribute)@\spxentry{value1}\spxextra{impedance.impedance attribute}}

\begin{fulllineitems}
\phantomsection\label{\detokenize{index:impedance.impedance.value1}}\pysigline{\sphinxbfcode{\sphinxupquote{value1}}\sphinxbfcode{\sphinxupquote{\DUrole{w}{  }\DUrole{p}{=}\DUrole{w}{  }0}}}
\end{fulllineitems}

\index{value2 (impedance.impedance attribute)@\spxentry{value2}\spxextra{impedance.impedance attribute}}

\begin{fulllineitems}
\phantomsection\label{\detokenize{index:impedance.impedance.value2}}\pysigline{\sphinxbfcode{\sphinxupquote{value2}}\sphinxbfcode{\sphinxupquote{\DUrole{w}{  }\DUrole{p}{=}\DUrole{w}{  }0}}}
\end{fulllineitems}


\end{fulllineitems}



\renewcommand{\indexname}{Python Module Index}
\begin{sphinxtheindex}
\let\bigletter\sphinxstyleindexlettergroup
\bigletter{a}
\item\relax\sphinxstyleindexentry{acquisition}\sphinxstyleindexpageref{index:\detokenize{module-2}}
\item\relax\sphinxstyleindexentry{app}\sphinxstyleindexpageref{index:\detokenize{module-app}}
\indexspace
\bigletter{i}
\item\relax\sphinxstyleindexentry{impedance}\sphinxstyleindexpageref{index:\detokenize{module-4}}
\indexspace
\bigletter{l}
\item\relax\sphinxstyleindexentry{lockin\_tab}\sphinxstyleindexpageref{index:\detokenize{module-0}}
\indexspace
\bigletter{m}
\item\relax\sphinxstyleindexentry{MainWindow}\sphinxstyleindexpageref{index:\detokenize{module-MainWindow}}
\indexspace
\bigletter{t}
\item\relax\sphinxstyleindexentry{TiePieLCR}\sphinxstyleindexpageref{index:\detokenize{module-1}}
\item\relax\sphinxstyleindexentry{TiePieLCR\_settings}\sphinxstyleindexpageref{index:\detokenize{module-3}}
\end{sphinxtheindex}

\renewcommand{\indexname}{Index}
\printindex
\end{document}